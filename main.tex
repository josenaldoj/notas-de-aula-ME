\documentclass[a4paper,12pt, oneside]{book}
\usepackage{main}
\usepackage{bibliography}
\usepackage{titlepage}

\begin{document}
    \nocite{*}
    % ------------------------------------------------------------------------------
    % Maketitle
    % ------------------------------------------------------------------------------
    \thispagestyle{empty}       % Remove page numbering on this page

    \printtitle                 % Print the title data as defined above
        \vfill
    \printauthor                % Print the author data as defined above
    \newpage
    
    \frontmatter % turns off chapter numbering, uses roman numerals for page numbers
    \tableofcontents
    \mainmatter % turns on chapter numbering, resets page numbering and uses arabic numerals for page numbers

    \chapter{Conjuntos}
    \import{src/conjuntos/}{apresentacao.tex}
    \import{src/conjuntos/}{introducao.tex}
    \import{src/conjuntos/}{pertinencia.tex}
    \import{src/conjuntos/}{inclusao.tex}
    \import{src/conjuntos/}{operacoes.tex}
    \import{src/conjuntos/}{logica.tex}
    \import{src/conjuntos/}{exercicios.tex}
    \import{src/conjuntos/}{bibliografia.tex} %@TODO: discutir isso aqui

    % deixar comentado enquanto tonhão não tiver chegado no assunto

    \chapter{Conjuntos Numéricos e Potenciação}
    \label{cap:conjuntos-numericos}    
    \import{src/conjuntos-numericos/}{apresentacao.tex}
    \import{src/conjuntos-numericos/}{conjuntos-numericos.tex}
    \import{src/conjuntos-numericos/}{operacoes.tex}
    \import{src/conjuntos-numericos/}{potenciacao.tex}
    \import{src/conjuntos-numericos/}{exercicios.tex}
    \import{src/conjuntos-numericos/}{bibliografia.tex} %TO-DO: discutir isso aqui

    \chapter{Equações e Inequações}
    \import{src/equacoes-e-inequacoes/}{introducao.tex}
    \import{src/equacoes-e-inequacoes/}{equacao-primeiro-grau.tex}
    \import{src/equacoes-e-inequacoes/}{equacao-segundo-grau.tex}
    \import{src/equacoes-e-inequacoes/}{inequacao-primeiro-grau.tex}
    \import{src/equacoes-e-inequacoes/}{inequacao-segundo-grau.tex}
    \import{src/equacoes-e-inequacoes/}{modulos.tex}
    \import{src/equacoes-e-inequacoes/}{desigualdades-classicas.tex}
    \import{src/equacoes-e-inequacoes/}{exercicios.tex}
    \import{src/equacoes-e-inequacoes/}{bibliografia.tex} %TO-DO: discutir isso aqui

    \begin{comment}
    \chapter{Princípio da Indução Finita}
    \import{src/principio-da-inducao-finita/}{introducao.tex}
    \import{src/principio-da-inducao-finita/}{pif.tex}
    \import{src/principio-da-inducao-finita/}{pfif.tex}
    \import{src/principio-da-inducao-finita/}{cuidados-ao-usar-o-pif.tex}

    \chapter{Funções}
    \import{src/funcoes/}{introducao.tex}
    \import{src/funcoes/}{definicao.tex}
    \import{src/funcoes/}{compostas.tex}
    \import{src/funcoes/}{inversa.tex}
    \import{src/funcoes/}{inj-sobr.tex}
    \import{src/funcoes/}{formulas-e-funcoes.tex}
    \import{src/funcoes/}{cardinalidade.tex}

    \chapter{Progressões}
    \import{src/progressoes/}{progressao-aritmetica.tex}
    \import{src/progressoes/}{somatorio-n-primeiros-pa}
    \import{src/progressoes/}{progressao-geometrica}
    \import{src/progressoes/}{formulas-pg}
    \import{src/progressoes/}{somatorio-n-primeiros-pg}

    \end{comment}

\end{document}
