\documentclass[a4paper,12pt, oneside]{book}
\usepackage{mystyle}

\begin{document}
    \title{Notas de Aula da Disciplina de Matemática Elementar}
    \author{Igor, Josenaldo} %TO-DO: mudar isso aqui

    \frontmatter % turns off chapter numbering, uses roman numerals for page numbers
    \maketitle
    \tableofcontents

    \mainmatter % turns on chapter numbering, resets page numbering and uses arabic numerals for page numbers

    \chapter{Conjuntos}
    \import{aulas/conjuntos/}{apresentacao.tex}
    \import{aulas/conjuntos/}{introducao.tex}
    \import{aulas/conjuntos/}{inclusao.tex}
    \import{aulas/conjuntos/}{igualdade.tex}
    \import{aulas/conjuntos/}{operacoes.tex}
    \import{aulas/conjuntos/}{logica.tex}
    \import{aulas/conjuntos/}{exercicios.tex}
    \import{aulas/conjuntos/}{bibliografia.tex} %TO-DO: discutir isso aqui

    \chapter{Conjuntos Numéricos e Potenciação}    
    \import{aulas/conjuntos_numericos/}{apresentacao.tex}
    \import{aulas/conjuntos_numericos/}{conjuntos_numericos.tex}
    \import{aulas/conjuntos_numericos/}{operacoes.tex}
    \import{aulas/conjuntos_numericos/}{potenciacao.tex}
    \import{aulas/conjuntos_numericos/}{exercicios.tex}
    \import{aulas/conjuntos_numericos/}{bibliografia.tex} %TO-DO: discutir isso aqui

    \chapter{Equações e Inequações}
    \import{aulas/equacoes_e_inequacoes/}{introducao.tex}
    \import{aulas/equacoes_e_inequacoes/}{equacao_primeiro_grau.tex}
    \import{aulas/equacoes_e_inequacoes/}{equacao_segundo_grau.tex}
    \import{aulas/equacoes_e_inequacoes/}{inequacao_primeiro_grau.tex}
    \import{aulas/equacoes_e_inequacoes/}{inequacao_segundo_grau.tex}
    \import{aulas/equacoes_e_inequacoes/}{modulos.tex}
    \import{aulas/equacoes_e_inequacoes/}{desigualdades_classicas.tex}
    \import{aulas/equacoes_e_inequacoes/}{exercicios.tex}
    \import{aulas/equacoes_e_inequacoes/}{bibliografia.tex} %TO-DO: discutir isso aqui

    \chapter{Princípio da Indução Finita}
    \import{aulas/principio-da-inducao-finita/}{introducao.tex}
    \import{aulas/principio-da-inducao-finita/}{pif.tex}
    \import{aulas/principio-da-inducao-finita/}{pfif.tex}
    \import{aulas/principio-da-inducao-finita/}{cuidados-ao-usar-o-pif.tex}

    \chapter{Funções}
    \import{aulas/funcoes/}{introducao.tex}
    \import{aulas/funcoes/}{definicao.tex}
    \import{aulas/funcoes/}{compostas.tex}
    \import{aulas/funcoes/}{inversa.tex}
    \import{aulas/funcoes/}{inj-sobr.tex}
    \import{aulas/funcoes/}{formulas-e-funcoes.tex}
    \import{aulas/funcoes/}{cardinalidade.tex}

    \chapter{Progressões}
    \import{aulas/progressoes/}{progressao-aritmetica.tex}
    \import{aulas/progressoes/}{somatorio-n-primeiros-pa}
    \import{aulas/progressoes/}{progressao-geometrica}
    \import{aulas/progressoes/}{formulas-pg}
    \import{aulas/progressoes/}{somatorio-n-primeiros-pg}

\end{document}
