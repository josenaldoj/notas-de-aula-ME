\documentclass[a4paper,12pt, oneside]{book}
\usepackage{main}
\usepackage{bibliography}
\usepackage{titlepage}

\begin{document}
    \nocite{*}
    % ------------------------------------------------------------------------------
    % Maketitle
    % ------------------------------------------------------------------------------
    \thispagestyle{empty}       % Remove page numbering on this page

    \printtitle                 % Print the title data as defined above
        \vfill
    \printauthor                % Print the author data as defined above
    \newpage
    
    \frontmatter % turns off chapter numbering, uses roman numerals for page numbers
    \tableofcontents
    \mainmatter % turns on chapter numbering, resets page numbering and uses arabic numerals for page numbers
    
    {
        \newcommand{\templatesdir}{src/esqueletos-dos-capitulos/}

        \subimport{\templatesdir}{conjuntos.tex}
        \subimport{\templatesdir}{conjuntos-numericos.tex}
        \subimport{\templatesdir}{equacoes-e-inequacoes.tex}
    }
    
    \begin{comment}
    \chapter{Princípio da Indução Finita}
    \import{src/principio-da-inducao-finita/}{introducao.tex}
    \import{src/principio-da-inducao-finita/}{pif.tex}
    \import{src/principio-da-inducao-finita/}{pfif.tex}
    \import{src/principio-da-inducao-finita/}{cuidados-ao-usar-o-pif.tex}

    \chapter{Funções}
    \import{src/funcoes/}{introducao.tex}
    \import{src/funcoes/}{definicao.tex}
    \import{src/funcoes/}{compostas.tex}
    \import{src/funcoes/}{inversa.tex}
    \import{src/funcoes/}{inj-sobr.tex}
    \import{src/funcoes/}{formulas-e-funcoes.tex}
    \import{src/funcoes/}{cardinalidade.tex}

    \chapter{Progressões}
    \import{src/progressoes/}{progressao-aritmetica.tex}
    \import{src/progressoes/}{somatorio-n-primeiros-pa}
    \import{src/progressoes/}{progressao-geometrica}
    \import{src/progressoes/}{formulas-pg}
    \import{src/progressoes/}{somatorio-n-primeiros-pg}

    \end{comment}

\end{document}
