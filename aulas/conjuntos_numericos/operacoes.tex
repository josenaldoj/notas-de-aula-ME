\section{Operações Básicas}

Definem-se duas operações básicas com os elementos dos conjuntos numéricos: adição e a multiplicação. A subtração e a divisão provêm da adição e da multiplicação, respectivamente. A diferença $a-b$ pode ser vista como $a+(-b)$, e a razão $c/d$, como $c \cdot (1/d)$, onde $a,b,c,d \in \R$ com $d \ne 0$.

Você está bem treinado nas operações com frações? Dê uma treinada na Khan Academy \khan{https://pt.khanacademy.org/math/arithmetic-home/arith-review-fractions}{aqui}!
