\section{Conjuntos Numéricos}

Nesta seção, serão introduzidos os principais conjuntos numéricos com os quais se trabalha na Matemática.

\begin{definition}
\label{def:naturais}
Ao conjunto $\N = \set {0, 1, 2, \dots , n, n+1, \dots}$, damos o nome de \textdef{conjunto dos números naturais}.
\end{definition}

Alguns autores consideram que o conjunto dos números naturais não possui o 0.
Um dos principais argumentos utilizados é o fato de que esses números foram inventados para contar coisas (casas, animais, etc), e, quando fazemos uma contagem, não a iniciamos no 1.
Conforme a Definição \ref{def:naturais}, enxergamos, neste texto, o 0 como um número natural. No entanto, usaremos a notação especial $\N^*$ para se referir ao conjunto $\N \setminus \set 0 = \set{1, 2, \dots, n, n+1, \dots}$.


\begin{definition}
Ao conjunto $\Z =\set {\dots , -m -1, -m, \dots, -1, 0, 1,  \dots , n, n+1, \dots}$ damos o nome de \textdef{conjunto dos números inteiros}. 
\end{definition}

\noindent Usam-se as seguintes notações para se referir a certas ``variações'' do conjunto dos inteiros:

\begin{gather*}
	\Z^\ast = \Z \setminus \set 0 \text{ (inteiros não nulos;)}\\
	\Z_+ = \N \text{ (inteiros não negativos);} \\
	\Z^\ast_+ =\N ^\ast \text{ (inteiros positivos);} \\
	\Z_- =\set {\dots , -m -1, -m, \dots, -1, 0} \text{ (inteiros não positivos);} \\
	\Z_-^\ast =\Z_- \setminus \set 0 \text{ (inteiros negativos).}
\end{gather*}

\begin{definition}
Ao conjunto $\Q = \set{p/q \tq p, q \in \Z \text{ e } q \neq 0}$ damos o nome de \textdef{conjunto dos números racionais}.
\end{definition}

\begin{remark}
A representação decimal de um número racional é finita ou é uma dízima periódica (infinita).
\end{remark}

\begin{exercise}
Reescreva os números $0{,}6$; $1{,}37$; $0{,}222\dots$; $0{,}313131 \dots$ e $1{,}123123123 \dots$ em forma de fração irredutível, ou seja, já simplificada.
\end{exercise}

\begin{definition}
O \textdef{conjunto dos números irracionais} é constituído por todos os números que possuem uma representação decimal infinita e não periódica.
\end{definition}

% Exemplo 5
\begin{example}
$\sqrt 2$, $e$ e $\pi$ são números irracionais.
\end{example}

\begin{proof}
Provemos que $\sqrt2 \notin \Q$. Para tal, suponhamos, por absurdo, que $\sqrt2 \in \Q$. Logo, existem $p, q \in \Z^*_+$ tais que $p/q$ é uma fração irredutível, ou seja, $p$ e $q$ não possuem nenhum fator comum nas suas decomposições em fatores primos. Teremos:
%
\begin{equation*}
\begin{aligned}
\sqrt2 = \dfrac{p}{q} & \implies 2 = \dfrac{p^2}{q^2} \\
& \implies 2q^2 = p^2
\end{aligned}
\end{equation*}
%
Isso é um absurdo pois, enquanto que o número $2q^2$ possui uma quantidade ímpar de fatores 2, o número $p^2$ possui uma quantidade par. Esse fato contraria o Teorema Fundamental da Aritmética, que garante a unicidade da decomposição dos números inteiros em fatores primos. Portanto, $\sqrt2 \notin \Q$.

As provas de que $\pi$ e $e$ são irracionais vão além do escopo desse texto. 
\end{proof}

\noindent Você sabia que existem infinitos ``maiores'' que outros? Qual conjunto você diria que tem mais elementos: racionais ou irracionais? O problema a seguir, proposto pelo matemático alemão David Hilbert em 1924, ilustra a ideia de enumeração de elementos de conjuntos infinitos.

{\itshape O Grande Hotel Georg Cantor tinha uma infinidade de quartos, numerados consecutivamente, um para cada número natural. Todos eram igualmente confortáveis. Num fim de semana prolongado, o hotel estava com seus quartos todos ocupados, quando chega um visitante. A recepcionista vai logo dizendo: 

--- Sinto muito, mas não há vagas. 

Ouvindo isto, o gerente interveio: 

--- Podemos abrigar o cavalheiro sim, senhora. 

E ordenou:

--- Transfira o hóspede do quarto 1 para o quarto 2, passe o do quarto 2 para o quarto 3 e assim por diante. Quem estiver no quarto $n$, mude para o quarto $\textit{n+1}$. Isso manterá todos alojados e deixará disponível o quarto 1 para o recém chegado.

Logo depois, chegou um ônibus com 30 passageiros, todos querendo hospedagem. Como deve proceder a recepcionista para acomodar todos? 

Horas depois, chegou um trem com uma infinidade de passageiros. Como proceder para acomodá-los?}

\begin{onlineact}
	\khan{https://pt.khanacademy.org/math/algebra/rational-and-irrational-numbers/modal/e/recognizing-rational-and-irrational-numbers}
	{Classifique números: racionais e irracionais}.
\end{onlineact}

\begin{onlineact}
	\khan{https://pt.khanacademy.org/math/algebra/rational-and-irrational-numbers/modal/e/recognizing-rational-and-irrational-expressions}
	{Expressões racionais versus irracionais}.
\end{onlineact}

\begin{definition}
À reunião de $\Q$ com o conjunto dos números irracionais, nomeamos de \textdef{conjunto dos números reais}. Denotamo-la por $\R$.
\end{definition}

Usamos os números reais para medir grandezas contínuas. A cada número real está associado um ponto na reta graduada, e vice-versa. 

O conjunto dos números reais é ``denso'' no sentido de que, entre quaisquer dois números reais distintos, há um número racional e um irracional. Já \khan{https://pt.khanacademy.org/math/algebra/rational-and-irrational-numbers/proofs-concerning-irrational-numbers/v/proof-that-there-is-an-irrational-number-between-any-two-rational-numbers}{este vídeo} da Khan Academy mostra que, além disso, entre dois racionais distintos sempre há um número irracional.

Quando se trata de números reais, são frequentes as ocasiões nas quais nossa intuição inicial pode ser falha. Uma delas é a respeito de dízimas periódicas. Você consegue afirmar se a igualdade $0\,999\ldots=1$ é verdadeira?

\begin{definition}
Chamamos $i = \sqrt {-1}$ de \textdef{número imaginário}, e ao conjunto $\C = \set{ a+bi \tq a,b \in \R}$ damos o nome de \textdef{conjunto dos
números complexos}.
\end{definition}

Os conjuntos estudados até aqui estão relacionados por meio da seguinte cadeia de inclusões próprias:
%
\begin{align*}
\N \subset \Z \subset \Q \subset \R \subset \C
\end{align*}

\begin{definition}
Seja $a+bi \in \C$. Nomeamos o número $a-bi$ de \textdef{conjugado} de $a+bi$.
\end{definition}