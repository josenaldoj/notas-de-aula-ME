\section{Princípio da Indução Finita}

\begin{theorem}[Princípio da Indução Finita]
\label{theorem:pif}
Considere $n_0$ um inteiro não negativo. Suponhamos que, para cada inteiro $n \geq n_0$, seja dada uma proposição $p \prn n$. Suponha
que se pode verificar as seguintes propriedades:

\begin{enumerate}[(a)]
  \item $p \prn{n_0}$ é verdadeira;
  \item Se $p \prn n$ é verdadeira, então $p \prn {n+1}$ também
  é verdadeira, para todo $n \geq n_0$.
\end{enumerate}

\noindent Então, $p \prn n$ é verdadeira para qualquer $n \geq n_0$.
\end{theorem}

\begin{remark}
No Teorema \ref{theorem:pif}, a afirmação (a) é chamada de \textdef{base da indução}, e a (b), de \textdef{passo indutivo}. O fato de que $p \prn n$ é verdadeira no item (b) é chamado de \textdef{hipótese de indução}.
\end{remark}

\begin{example}
Demonstre que, para qualquer $n \in \N^*$, é válida a igualdade:
%
\begin{equation*}
2+ 4+ \dots + 2n = n \prn {n+1}.
\end{equation*}
\end{example}

\begin{solution}
Seja $p(n) : 2+4+\dots + 2n = n(n+1)$. Provaremos a validade de $p(n)$ para todo $n \in \nnats$ utilizando o Princípio da Indução Finita.
%
\begin{itemize}
	\item \textit{Caso base} ($n=1$):

	$p(1):2=1(1+1)=2$ é válido.

	\item \textit{Passo de indução}:

	Suponha, como hipótese de indução, que $p(k)$ é válida para algum $k \in \nnats$; ou seja, vale a equação:
	%
	\begin{equation*}
	2+4+\dots+2k=k(k+1)
	\end{equation*}
	%
	Provemos a validade de $p(k+1)$.
	%
	\begin{align*}
	2+4+\dots + 2k+2(k+1) = & k(k+1)+2(k+1) \\
	= & (k+1)(k+2) \text{(HI)} \\
	= & (k+1)\left[(k+1)+1\right]
	\end{align*}
	%
	Com isso, provamos o passo de indução.
\end{itemize}
%
Portanto, pelo Princípio da Indução Finita, a equação $2+4+\dots + 2n=n(n+1)$ é válida para todo $n \in \nnats$.
\end{solution}

\begin{example}
Demonstre que, para qualquer $n \in \N^*$, é válida a igualdade:
%
\begin{equation*}
1+3+\dots +\prn {2n-1} = n^2.
\end{equation*}
\end{example}

\begin{solution}
Aplicaremos o Princípio da Indução Finita para $n \in \nnats$.
%
\begin{itemize}
	\item \textit{Caso base} ($n=1$):

	Como $1=1^2$, o caso base é válido.

	\item \textit{Passo de indução}:

	Como hipótese de indução, suponha que a igualdade é válida para algum $k\in \nnats$, ou seja:
	%
	\begin{equation*}
	1+3+\dots+(2k-1) = k^2.
	\end{equation*}
	%
	Provemos a validade da afirmação para $n := k$. Com efeito,
	%
	\begin{align*}
	1+3+\dots + (2k-1) & = \left[2(k+1)-1\right] \text{(HI)} \\
	& = k^2+2k+1 \\
	& = (k+1)^2
	\end{align*}
\end{itemize}
%
Portanto, a equação a seguir é válida para todo $n \in \nnats$:
%
\begin{equation*}
1+3+\dots + (2n-1) = n^2.
\end{equation*}
\end{solution}

\begin{example}
Mostre que, para todo número $n \in \nnats$, $n>3$, vale que:
%
\begin{equation*}
2^n < n!
\end{equation*}
\end{example}

\begin{solution}
Utilizaremos indução finita em $n \in \nnats$, $n >3$.
%
\begin{itemize}
	\item \textit{Caso base} ($n=4$):
	%
	\begin{equation*}
	2^4 =16 < 24=4!
	\end{equation*}  
	%
	\item \textit{Passo de indução}:

	Suponha, como hipótese de indução, que $2^k < k!$ para algum $k \in \nnats$, $k>3$. Além dessa hipótese, temos que $2 < k+1$ pois $k>3$. Logo,
	%
	\begin{align*}
	0<2<k+1 \text{ e } 0<2^k<k! & \implies 2\cdot 2^k < k!(k+1) \\
	& \iff 2^{k+1} < (k+1)!,
	\end{align*}
	%
	provando a validade da inequação para $n:=k+1$.
\end{itemize}
%
Portanto, $2^n < n!$ para todo $n \in \nnats$, $n>3$.
\end{solution}

\begin{example}
Prove  que, para todo $n \in \nnats$,
%
\begin{equation*}
\underbrace{\sqrt{2+\sqrt{2+\sqrt{2+ \dots + \sqrt 2}}}}_{n  \text{ radicais}} < 2.
\end{equation*}
\end{example}

\begin{solution}
Aplicaremos o Princípio da Indução Finita em $n \in \nnats$.
%
\begin{itemize}
	\item \textit{Caso base} ($n=1$):

	$\sqrt 2 < 2$ é válido.

	\item \textit{Passo de indução}:

	Suponha a validade da inequação para algum $k \in \nnats$, ou seja, 
	%
	\begin{equation*}
	\underbrace{\sqrt{2+\sqrt{2+\sqrt{2+ \dots + \sqrt 2}}}}_{k  \text{ radicais}} < 2.
	\end{equation*}
	%
	Da hipótese de indução, segue que:
	%
	\begin{align*}
	2+\underbrace{\sqrt{2+\sqrt{2+\sqrt{2+ \dots + \sqrt 2}}}}_{k  \text{ radicais}} < 2+2 & \implies \underbrace{\sqrt{2+\sqrt{2+\sqrt{2+ \dots + \sqrt 2}}}}_{k+1  \text{ radicais}} < \sqrt 4 = 2.
	\end{align*}
\end{itemize}
%
Logo, a inequação é válida para $n := k+1$; e, assim, concluímos a validade da inequação para todo $n \in \nnats$.
\end{solution}

\begin{example}
Seja $n \in \N$ tal que $n\ge 3$. Mostre  que podemos cobrir os $n^2$ pontos no reticulado a seguir traçando $2n-2$ segmentos de reta sem tirar o lápis do papel.
%
\begin{equation*}
\underbrace{\begin{array}{ccccc}
                \bullet & \bullet & \bullet & \bullet & \bullet \\
                \bullet & \bullet & \bullet & \bullet & \bullet \\
                \bullet & \bullet & \bullet & \bullet & \bullet \\
                \bullet & \bullet & \bullet & \bullet & \bullet \\
                \bullet & \bullet & \bullet & \bullet & \bullet
              \end{array}
}_{n \times n \text{ pontos}}
\end{equation*}
\end{example}

\begin{example}
Um rei muito rico possui $3^n$ moedas de ouro. No entanto, uma dessas moedas é falsa, e seu peso é menor que o peso das demais. Com uma balança de dois pratos e sem nenhum peso, mostre que é possível encontrar a moeda falsa com apenas $n$ pesagens.
\end{example}

\begin{solution}
Dado que o rei tem $3^n$ moedas, usaremos indução finita em $n \in \nnats$. 

Para $n=1$, ou seja, três moedas, coloca-se uma moeda em cada prato. Se as moedas tiverem pesos iguais, então a moeda falsa é a que não foi colocada na balança. Do contrário, a moeda falsa é a mais leve.

Como hipótese de indução, suponhamos que, para algum $k \in \nnats$, seja possível identificar a moeda falsa dentre $3^k$ moedas realizando-se $k$ pesagens. Suponha que temos $3^{k+1}$ moedas. 

Separando as moedas em três grupos com $3^k$ moedas cada, coloca-se um grupo em cada prato da balança. Assim, analogamente ao caso $n = 1$, identificamos o grupo com a moeda falsa. Tal grupo tem $3^k$ moedas e, pela hipótese de indução, pode-se identificar a moeda falsa com mais $k$  pesagens, totalizando $k+1$ pesagens.

Assim, para cada $n \in \nnats$, é possível identificar a moeda falsa dentre $3^n$ moedas com apenas $n$ pesagens.
\end{solution}

\begin{theorem}
Para quaisquer $a_1, a_2, \dots , a_n \in \R_+$, vale:
%
\begin{equation*}
    \sqrt[n]{a_1\dots a_n} \le \frac {a_1 + \dots + a_n} n.
\end{equation*}
\end{theorem}