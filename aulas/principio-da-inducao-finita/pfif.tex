\section{Princípio Forte da Indução Finita}

\begin{theorem}[Princípio Forte da Indução Finita]
Considere $n_0$ um inteiro não negativo. Suponhamos que, para cada inteiro $n \ge n_0$, seja dada uma proposição $p \prn n$. Suponha que se pode verificar as seguintes propriedades:
%
\begin{enumerate}[(a)]
  \item $p \prn{n_0}$ é verdadeira;
  \item Se para cada inteiro não negativo $k$, com $n_0 \le k \le n$, temos que
   $p \prn k$ é verdadeira, então $p \prn {n+1}$ também
  é verdadeira.
\end{enumerate}
%
Então, $p \prn n$ é verdadeira para qualquer $n \ge n_0$.
\end{theorem}

\begin{theorem}[Teorema Fundamental da Aritmética]
Todo número natural $N$ maior que 1 pode ser escrito como um produto
%
\begin{equation}
\label{theorem:tfa}
N = p_1 \cdot p_2 \cdot p_3 \dots p_m,
\end{equation}
%
onde $m \ge 1$ é um número natural e os $p_i$, $1 \le i \le m$, são números primos. Além disso, a fatoração exibida na Equação \ref{theorem:tfa} é única se exigirmos que $p_1 \le p_2 \le \dots \le p_m$.
\end{theorem}

\begin{example}
Critique a argumentação a seguir.

\textit{Quer-se provar que todo número natural é pequeno. Evidentemente, 1 é um número pequeno. Além disso, se $n$ for pequeno, $n+1$ também o será, pois não se torna grande um número pequeno simplesmente somando-lhe uma unidade. Logo, por indução, todo número natural é pequeno}.
\end{example}

\begin{example}
Considere a seguinte afirmação, evidentemente falsa:

\textit{Em um conjunto qualquer de $n$ bolas, todas as bolas possuem a mesma cor}.

Analise a seguinte demonstração por indução para a afirmação anterior e aponte o problema da demonstração.

{\it Para $n=1$, nossa proposição é verdadeira pois em qualquer conjunto com uma bola, todas as bolas têm a mesma cor, já que só existe uma bola. 

Assumamos, por hipótese de indução, que a afirmação é verdadeira para $n$ e provemos que a afirmação é verdadeira para
$n+1$.

Ora, seja $A = \set {b_1, \dots, b_n, b_{n+1}}$ o conjunto  com $n+1$ bolas referido. Considere os subconjuntos $B$ e $C$ de $A$ com $n$ elementos, construídos como:
%
\begin{equation*}
B = \set {b_1, b_2, \dots, b_n} \text{ e } C= \set{ b_2, \dots, b_{n+1}}.
\end{equation*}
%
De fato, ambos os conjuntos têm $n$ elementos. Assim, as bolas $b_1, b_2, \dots , b_n$ têm a mesma cor. Do mesmo modo, as bolas do conjunto $C$ também têm a mesma cor. Em particular, as bolas $b_n$ e $b_{n+1}$ têm a mesma cor (ambas estão em $C$). Assim, todas as $n+1$ bolas têm a mesma cor.}
\end{example}