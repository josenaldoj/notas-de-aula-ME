\section{Conjuntos e Lógica}

\noindent
Em Matemática, a Teoria de Conjuntos está intimamente relacionada à Lógica.
Como evidência disso, existem diversas equivalências entre relações e operadores de conjuntos e conectivos lógicos.
Apresentar-se-ão quatro delas, mas antes vamos entender como usamos o símbolo $ \implies$ no nosso curso.

	\begin{remark}
	Em toda esta seção, considere $P$ e $Q$ propriedades aplicáveis aos elementos de $\U$.
	Considere, também, $A = \set {x \tq x \text{ satisfaz } P}$ e $B= \set {x \tq x \text{ satisfaz } Q}$.
	\end{remark}

\begin{comment}
Em Lógica Proposicional, a implicação $\prn \to $ é um operador lógico. 
Para que uma implicação seja aplicada ela precisa de duas proposições $P$ e $Q$, 
onde cada uma pode ser verdadeira ou falsa.
A validade da implicação $P \to Q$ é definida através da seguinte tabela, que chamamos de \textdef{tabela verdade}.

<INSERIR A TABELA VERDADE DE $P \to Q$ AQUI.

Em nosso texto, interpretamos o $P$ como a hipótese do meu problema e $Q$ como a tese em uma implicação.
Essa implicação aparece textualmente de algumas maneiras, a mais comum delas é ``Se $P$, então $Q$'' 
e, em símbolos, escrevemos $P \implies Q$.

Quando queremos provar a validade desse tipo de afirmação, 
tomamos nossa hipótese $P$ como uma propriedade verdadeira, 
o que faz considerarmos somente as duas primeiras linhas da tabela verdade de $P \to Q$.
Dessa maneira, a afirmação ``Se $P$, então $Q$'' é verdadeira se conseguirmos provar que a tese $Q$ também é verdadeira.

Quando já está provado que ``Se $P$, então $Q$'' é verdadeira, 
utilizamos esse resultado para concluir que $Q$ é verdadeira quando temos que $P$ é verdadeira, 
pois, novamente olhando somente para as duas primeiras linhas de $P \to Q$, 
a única opção que temos, quando $P \to Q$ e $P$ são verdadeiras, é que $Q$ também seja verdadeira.
\end{comment}

\begin{equivalence}[Inclusão e implicação] 
$$A \subset B \text{ é equivalente a } P \implies Q$$
\end{equivalence}

\begin{equivalence}[Igualdade e bi-implicação] 
	$$A=B \text{ é equivalente a } P \iff Q$$
\end{equivalence}

\begin{example}
Analise as implicações abaixo:
\begin{equation*}
\begin{aligned}
x^2+1=0 & \implies \left(x^2 +1 \right) \left( x^2-1 \right) = 0
\cdot \left( x^2-1 \right) \\
& \implies x^4 - 1 = 0 \\
& \implies x^4 = 1 \\
& \implies x \in \set {-1, 1}
\end{aligned}
\end{equation*}
%
Isso quer dizer que o conjunto solução de $x^2 +1 = 0$ é $\set{-1,
1}$?
\end{example}

\begin{solution}
O conjunto-solução de $x^2 + 1 = 0$ é $S = \set{x \in \R \tq x^2 + 1 = 0} = \emptyset$, o que implica que $S \ne \set{-1,1}$.
\end{solution}

\begin{equivalence}[Complementar e negação] 
$A^C$ é equivalente a $\neg P$.
\end{equivalence}
%TO-DO: change i. and i..
Podemos combinar os itens i. e i.. da Proposição \ref{prop-complementar} e obter que:
%
\begin{align*}
P \implies Q \text{ se, e somente se, } \neg Q \implies \neg P.	
\end{align*}
%
Chamamos $Q \implies P$ de \textdef{recíproca} de $P \implies Q$, e $P \land \neg Q$ de \textdef{negação}\newline de $P \implies Q$. É dado um exemplo no Exercício \ref{exe:escrever-reciprocas}.

\begin{example}
Observe as afirmações:

\begin{enumerate}
	\item Todo número primo maior do que 2 é ímpar;
	\item Todo número par maior do que 2 é composto.
\end{enumerate}
%
Essas afirmações dizem exatamente a mesma coisa, ou seja, exprimem a mesma ideia; só que com diferentes termos. Podemos reescrevê-las na forma de implicações vendo claramente que uma é contrapositiva da outra, e todas estão sob a hipótese de que $n \in \N$, com $n > 2$:
%
\begin{nofleqn}{align*}
n \text{ primo 	} & \implies n \text{ ímpar}\\
\neg (\text{ }n \text{ ímpar } )& \implies \neg (\text{ }n \text{ primo } )\\
n \text{ par } & \implies n \text{ composto}
\end{nofleqn}
\end{example}

\begin{equivalence}[União e disjunção]
$A \cup B$ é equivalente a $P \lor Q$ ($P \text{ ou } Q$).
\end{equivalence}

\begin{equivalence}[Interseção e conjunção]
$A \cup B$ é equivalente a $P \land Q$ ($P \text{ e } Q$).
\end{equivalence}

\begin{remark}
O conectivo lógico \textit{ou} tem significado diferente do usado normalmente no português. Na linguagem coloquial, usamos $P$ \textit{ou} $Q$ sem permitir que sejam as duas coisas ao mesmo tempo. Analise a seguinte história:\newline

\textit{Um obstetra que também era matemático acabara de realizar um parto quando o pai perguntou: ``É menino ou menina, doutor?''. E ele respondeu: ``sim''}.
\end{remark}

As equivalências entre as relações e os operadores da Teoria dos Conjuntos e conectivos da Lógica são resumidas na Tabela \ref{tbl:equiv-conj-logc}.

\begin{table}[h]
\begin{center}
\caption{Equivalências entre as relações e operadores de conjuntos e conectivos lógicos.}
\label{tbl:equiv-conj-logc}
\begin{tabular}{|c|c|}
	\hline
	Operação/relação em Conjuntos & Fórmula de Lógica \\\hline
	$A=B$ & $P \iff Q$ \\ \hline
	$A \subset B$ & $P \implies Q$ \\ \hline
	$A^C$ & $\neg P$ \\ \hline
	$A \cup B$ & $P \lor Q$ \\ \hline
	$A \cap B$ & $P \land Q$ \\
	\hline
\end{tabular}
\end{center}
\end{table}

\begin{problem}
A polícia prende quatro homens, um dos quais cometeu um furto. Eles fazem as seguintes declarações:
%
\begin{itemize}[leftmargin=*]
  \item Arnaldo: Bernaldo fez o furto.
  \item Bernaldo: Cernaldo fez o furto.
  \item Dernaldo: eu não fiz o furto.
  \item Cernaldo: Bernaldo mente ao dizer que eu fiz o furto.
\end{itemize}
%
Se sabemos que só uma destas declarações é a verdadeira, quem é culpado pelo furto?
\end{problem}
