\section{Exercícios}

\begin{exercise}
\label{exe:vazio-notacao}
	De que outras formas podemos representar o \emptysetref{conjunto vazio} utilizando as duas notações de definição de conjuntos que conhecemos?
\end{exercise}


\begin{exercise}
	Decida quais das afirmações a seguir estão corretas. Justifique suas respostas.
	\begin{enumerate}[a.]
		\item $\emptyset \in \emptyset$	
		\item $\emptyset \subset \emptyset$
		\item $\emptyset \in \set{\emptyset}$
		\item $\emptyset \subset \set{\emptyset}$
	\end{enumerate}
\end{exercise}


% \exercise Demonstre as seguintes propriedades sobre os conjuntos arbitrários $A$, $B$ e $C$. Utilize apenas as notações de conjuntos, isto é, sem usar a lógica diretamente. (Dica: ao demonstrar uma propriedade você tem o direito de usá-la para demosntrar as outras)
%	\begin{enumerate}
%		\item $A \subseteq A$
%		\item $A \subseteq A \cup B$
%		\item $A \cup (B \cap C) = (A \cup B) \cap (A \cup C)$
%		\item $A \cap (B \cup C) = (A \cap B) \cup (A \cap C)$
%		\item $(A \cup B)^C = A^C \cap B^C$
%		\item $(A \cap B)^C = A^C \cup B^C$
%		\item $(A^C)^C = A$	
%	\end{enumerate}
