\section{Pertinência}

Na Seção \ref{sec:intro}, entendemos o que é um conjunto e como podemos defini-lo.
Agora, teremos o poder de relacionar objetos a conjuntos perguntando se um dado objeto está em um conjunto.
Como já falamos sobre, só temos uma resposta possível dentre sim e não.
Dados um objeto $a$ e um conjunto $A$, utilizamos a notação $a \in A$ para dizer que $a$ pertence ao conjunto $A$, e $a \notin A$ significa que $a$ não pertence a $A$.

\begin{example}
Considere $PP$ e $V$ conforme definido nos Exemplos~\ref{ex-vogais} e~\ref{ex-primos-pares}, respectivamente. Temos que $\texttt{e} \in \mathcal{V}$ e $3 \notin PP$.
\end{example}

\begin{example}
Considere $L$ como sendo o conjunto de todos as letras do alfabeto latino.
Sabendo que as vogais do conjunto $V$ fazem parte desse alfabeto, podemos concluir que $\texttt{a}, \texttt{e}, \texttt{i}, \texttt{o}, \texttt{u} \in L$.
Também sabemos que a letra $\texttt{j}$ faz parte do alfabeto, isso é, $\texttt{j} \in L$.
Contudo, como $\texttt{j}$ não é uma vogal, podemos concluir que $\texttt{j} \notin V$.
\end{example}

Ainda na Seção \ref{sec:intro} \nameref{sec:intro}, viu-se que dois conjuntos iguais $A$ e $B$ possuem, necessariamente, os mesmos elementos, o que implica que são diferentes quando um possui algum elemento que o outro não possui.
Com a notação de pertinência, isso é o mesmo que dizer que existe $x \in A$ tal que $x \notin B$ ou que existe $x \in B$ tal que $x \notin A$.

\begin{example}
Os conjuntos $\N$ (números naturais) e $\Z$ (números inteiros), abordados com mais detalhe no Capítulo \ref{cap:conjuntos-numericos}, são diferentes.
Note que $-1 \in \Z$ mas $-1 \notin \N$, por exemplo, o que prova que esses conjuntos são diferentes.
\end{example}

As noções de conjuntos e elementos não são mutuamente exclusivas como, por exemplo, a paridade nos números naturais. 
Nos Exemplos \ref{exe:conjuntos-de-conjuntos-explicito} e \ref{exe:conjuntos-de-conjuntos-implicito}, serão vistos conjuntos cujos elementos também são conjuntos!

\begin{example}
\label{exe:conjuntos-de-conjuntos-explicito}
Considere o conjunto $A = \set{\set{1,2},\set{2}, 1}$.
    Observe que existem elementos em $A$ que são conjuntos.
    Note que:
    \begin{enumerate}
        \item $\set{1,2} \in A$
        \item $\set{2} \in A$
        \item $1 \in A$
        \item $2 \notin A$
    \end{enumerate}
\end{example}

\begin{example}
\label{exe:conjuntos-de-conjuntos-implicito}
Os habitantes de um país pode ser visto como um conjunto de pessoas.
Por sua vez, cada pessoa pode ser vista como um conjunto de células.
Mais tarde, na definição~\ref{def:powerset}, veremos um famoso conjunto formado por conjuntos.
\end{example}


Para dizer se um dado objeto deve receber o título de ``elemento'', é preciso olhar para o objeto com o qual ele está se relacionando. 
No caso do Exemplo \ref{exe:conjuntos-de-conjuntos-explicito}, o conjunto $\set{1,2}$ é elemento do conjunto $A$, mas quando o interesse se volta à sua relação com o número $2$, não faz sentido se referir a $\set{1,2}$ usando simplesmente o nome ``elemento''.

A Definição \ref{def:emptyset} irá introduzir outro importante tipo de conjunto.

\begin{definition}[O Conjunto Vazio]
\label{def:emptyset}
O conjunto que não possui elementos é chamado de \textdef{conjunto vazio} e é representado por $\emptyset$.
Em outras palavras, para qualquer que seja o objeto $x$, temos $x \notin \emptyset$.
\end{definition}

Agora, com a Definição~\ref{def:emptyset}, o Exercício~\homeworkref{exe:vazio-notacao} já pode ser feito.

\begin{example}
Quais outros conjuntos você conhece? Que tal pensar sobre o conjunto $A = \set{x \tq x \notin A}$?
\end{example}

