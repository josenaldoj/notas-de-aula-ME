\section{Pertinência}

Na seção anterior entendemos o que é um conjunto, e como podemos definí-los.
Agora, teremos o poder de relacionar objetos à conjuntos, e receber uma resposta que pode ser verdadeira ou falsa.
A idéia é poder perguntar se um dado objeto, faz parte de tal conjunto.

\begin{definition}[Relação de Pertinência]
\label{def:in}
Dado um conjunto arbitrário $A$ e um objeto $x$.
Poderemos sempre perguntar se $x$ é um elemento de $A$, se for o caso, represetaremos por $x \in A$, e se não for o caso, por $x \notin A$.
\end{definition}

\begin{remark}
Conjuntos também podem pertencer a outros conjuntos.
\end{remark}

\begin{example}
Considere $PP$ e $V$ conforme definido nos Exemplos \ref{ex-primos-pares} e \ref{ex-vogais}, respectivamente. Temos que $\texttt{e} \in \mathcal{V}$ e $3 \notin PP$.
\end{example}

\begin{definition}[O Conjunto Vazio]
\label{def:emptyset}
	O conjunto que não possui elementos é chamado de \textdef{conjunto vazio} e é representado por $\emptyset$.
	Em outras palavras, para qualquer objeto $x$ que seja, podemos concluir que $x \notin \emptyset$.
\end{definition}

\begin{homework}
	Resolva o \homeworkref{exe:vazio-notacao}.
\end{homework}

\begin{example}
Quais outros conjuntos você conhece? Que tal pensar sobre o conjunto $A = \set{x \tq x \notin A}$?
\end{example}

