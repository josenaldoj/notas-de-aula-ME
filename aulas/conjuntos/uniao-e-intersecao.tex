\subsection{União e Interseção}

\begin{definition}[União e Interseção]
Dados os conjuntos $A$ e $B$, definem-se:

\begin{enumerate}
	\item A \textdef{união} $A \cup B$ como sendo o conjunto formado pelos elementos que pertencem a pelo menos um dos conjuntos $A$ e $B$. Ou seja,
		$$ A \cup B = \set{x \tq x \in A \text{ ou } x \in B} . $$
	\item A \textdef{interseção} $A \cap B$ como sendo o conjunto formado pelos elementos que pertencem a ambos $A$ e $B$. Ou seja,
		$$ A \cap B = \set{x \tq x \in A \text{ e } x \in B} . $$
\end{enumerate}
\end{definition}

\begin{example}
Sejam $A = \set{1, 2, 3}$ e $ B = \set{2,5}$. Determine $A \cup B$ e $A \cap B$.
\end{example}

\begin{solution}
\begin{align*}
	&A \cup B = \set{1,2,3,5};\\
	&A \cap B = \set{2}.\\
\end{align*}
\end{solution}

\begin{proposition}[Propriedades da união e interseção]
\label{prop:uniao-e-intersecao}
Para todo conjunto $A$, $B$ e $C$, tem-se:
\begin{enumerate}[i)]
\item \textdef{Comutatividade}: 
	\begin{enumerate}[a)]
		\item $A \cup B = B \cup A$;
		\item $A \cap B = B \cap A$.
	\end{enumerate}
\item \textdef{Associatividade}:
	\begin{enumerate}[a)]
		\item $\left(A \cup B \right) \cup C = A \cup \left( B \cup C \right)$;
		\item $\left(A \cap B \right) \cap C = A \cap \left( B \cap C \right)$.
	\end{enumerate}
\item % @TODO nomear esta propriedade - by TONHAUNM: não vejo necessidade
	\begin{enumerate}[a)]
		\item $A \subset \prn{A \cup B}$;
		\item $\prn{A \cap B} \subset A$.
	\end{enumerate}
\item \textdef{Distributividade}, de uma em relação à outra:
	\begin{enumerate}[a)]
		\item $A \cap \left( B \cup C \right) = \left(A \cap B \right) \cup \left( A \cap C \right)$;
		\item $A \cup \left( B \cap C \right) = \left(A \cup B \right) \cap \left( A \cup C  \right)$.
	\end{enumerate}
\item
	\textdef{Leis de DeMorgan}:
	\begin{enumerate}[a)]
		\item $\left( A \cup B \right)^C = A^C \cap B^C$;
		\item $\left(A \cap B \right)^C = A^C \cup B^C$.
	\end{enumerate}
\end{enumerate}
\end{proposition}

% Proposição 21
\begin{proof}
\begin{enumerate}[i)]
\item[]
\item Exercício.
\item
	\begin{enumerate}[a)]
		\item 
		Exercício.

		\item
		Provemos que $A \cap (B \cap C) \subset (A \cap B) \cap C$. 
		Seja $x \in A \cap (B \cap C)$, isso é, $x \in A$ e $x \in B \cap C$.
		De $x \in B \cap C$, temos $x \in B$ e $x \in C$.
		Como $x \in A$ e $x \in B$, segue que $x \in A \cap B$.
		Além disso, $x \in C$.
		Então, $x \in A \cap (B \cap C)$.
		Logo, $A \cap (B \cap C) \subset (A \cap B) \cap C$. 

		A prova de que $A \cap (B \cap C) \supset (A \cap B) \cap C$, necessária para concluir a igualdade desejada, fica como exercício para o leitor.

	\end{enumerate}

\item 	
	\begin{enumerate}[a)]
		\item Seja $x \in A$. Pela definição de união, segue que $x \in A \cup B$. Portanto, $A \subset \prn{A \cup B}$;
		\item Seja $x \in \prn{A \cap B}$. Pela definição de interseção, segue que $x\in A$ e $x \in B$. 
		Em particular, já temos que $x \in A$. Portanto, $\prn{A \cap B} \subset A$.
	\end{enumerate}
	Em decorrência dessa propriedade, vamos tratar como imediatos que: 
	se $x \in A$, então $A \subset \prn{A \cup B}$; 
	e se $x \in \prn{A \cap B}$, então $x\in A$ (ou, caso convenha, $x\in B$).
\item Exercício.
\item Exercício.
\end{enumerate}
\end{proof}

\begin{onlineact}[\khan{https://pt.khanacademy.org/math/statistics-probability/probability-library/basic-set-ops/e/basic_set_notation}{Notação Básica de Conjunto}]
Veja o desempenho na Missão O Mundo da Matemática - Probabilidade.
\end{onlineact}


