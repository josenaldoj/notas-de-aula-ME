\section{Inclusão}

\begin{definition}[Relação de Inclusão]
\label{def:subset}
Considere dois conjuntos $A$ e $B$.
Se for o caso que todo elemento de $A$ é, também, elemento de $B$, diz-se que $A$ é um \textdef{subconjunto} de $B$, que $A$ está \textdef{contido} em $B$, ou que $A$ é \textdef{parte} de $B$.
Para indicar esse fato, usa-se a notação $A \subset B$.

\label{def:notsubset}
Quando $A$ \textit{não} está contido em $B$,  utiliza-se a notação  $A \notsubset B$.
Na prática, isso significa que existe pelo menos um elemento que está em $A$ mas não está em $B$.
Em outras palavras, existe um elemento $x$ tal que $x \in A$ mas $x \notin B$.
\end{definition}

\begin{remark}
Também podemos escrever $B \supset A$ quando for o caso que $A \subset B$. Para essa situação, dizemos que $B$ \textdef{contém} $A$.
\end{remark}

\begin{example}
Considere $T$ o conjunto de todos os triângulos e $P$ o conjunto dos polígonos no plano. Como todo triângulo é um polígono podemos concluir que $T \subset P$.
Observe também que $P \notsubset T$. Para poder concluir isso precisamos encontrar um elemento de $P$ que não seja um elemento de $T$. Ora, basta considerar um quadrado $q$ com lados de tamanho 1, como todo quadrado é um polígono, temos que $q \in P$, mas quadrados não são triangulos, então $q \notin T$.
\end{example}

\begin{example}
Na Geometria, uma reta, um plano e o espaço são conjuntos. Seus
elementos são pontos. Quando dizemos que uma reta $r$ está no plano $\Pi$, estamos afirmando que $r$ está contida em $\Pi$ ou, equivalentemente, que $r$ é um subconjunto de $\Pi$, pois todos os pontos que pertencem a $r$ pertencem, também, a $\Pi$. Nesse caso, deve-se escrever $ r \subset \Pi$. Porém, não é correto dizer que $r$ pertence a $\Pi$, nem escrever $r \in \Pi$. Os
elementos do conjunto $\Pi$ são pontos, não retas.
\end{example}

\begin{example}
Considere os conjuntos $N = \set{1, 2, 3, 4, 5, 6}$, $I = \set{1, 3, 5}$ e $P = \set{0, 2, 4, 6}$.
Analisando o cenário, podemos concluir que:
\begin{enumerate}
\item
$I \subset N$. Observe que todos os elementos de $I$ também são elementos de $N$.
\item
$P \notsubset N$. Observe que nem todos os elementos de $P$ são elementos de $N$, pois $0 \in P$ mas $0 \notin N$.
\end{enumerate}
\end{example}

\begin{proposition}[Inclusão universal do $\emptyset$]
\label{prop:emptyset1}
Seja $A$ um conjunto arbitrário. Pode-se concluir que $\emptyset \subset A$.
Em outras palavras, o \emptysetref{conjunto vazio} é subconjunto de todos os conjuntos.
\end{proposition}

\begin{proof}
Para chegarmos num absurdo, considere um conjunto $A$ tal que $\emptyset \notsubset A$.
Logo, podemos concluir que existe um elemento $x$ tal que $x \in \emptyset$ mas $x \notin A$, pela \subsetref{definição de inclusão}.
Mas, já sabemos que $x \notin \emptyset$ pela \emptysetref{definição do conjunto vazio}.
O que nos leva a um absurdo, pois não pode acontecer que $x \in \emptyset$ e $x \notin \emptyset$ ao mesmo tempo.
Portanto, podemos concluir que $\emptyset \subset A$.
\end{proof}

\begin{remark}
Ao manter a arbitrariedade de um conjunto, qualquer conclusão relacionada a este conjunto valerá para todos os conjuntos.
\end{remark}

\begin{definition}[Inclusão Própria]
Dizemos que $A$ é um \textdef{subconjunto próprio} de $B$ quando $A \subset B$ mas $A \neq B$.
Quando isso ocorre utiliza-se a notação $A \subsetneq B$.
\end{definition}

\begin{proposition}[Propriedades da inclusão]
Considere $A$, $B$ e $C$ conjuntos arbitrários. Logo, são válidas as propriedades a seguir:

\begin{enumerate}[i)]
\item
	\label{inclusao:reflexividade}
	\textdef{Reflexividade}: $A \subset A$;
\item
	\label{inclusao:antissimetria}
	\textdef{Antissimetria}: Se $A \subset B$ e $B \subset A$, então $A = B$;
\item
	\label{inclusao:transitividade}
	\textdef{Transitividade}: Se $A \subset B$ e $B \subset C$, então $A \subset C$.
\end{enumerate}
\end{proposition}

\begin{proof}
	Apenas dos itens \ref{inclusao:reflexividade} e \ref{inclusao:transitividade}.

	\ref{inclusao:reflexividade}
	Seja $x \in A$ um elemento arbitrário.
	Ora, como já temos que $x \in A$ podemos concluir que $A \subset A$.

	\ref{inclusao:antissimetria}
	Sejam $A$ e $B$ conjuntos tais que $A \subset B$ e $B \subset A$.
	Suponha, por contradição, que $A \ne B$, ou seja, existe $x \in A$ tal que $x \notin B$ (1) ou existe $x \in B$ tal que $x \notin A$ (2).
	Ora, (1) é o mesmo que $A \notsubset B$, contradizendo $A \subset B$.
	Analogamente, (2) contradiz $B \subset A$.
	Portanto, $A = B$.

	\ref{inclusao:transitividade}
	Sejam $A$, $B$ e $C$ conjuntos tais que $A \subset B$ e $B \subset C$.
	Agora basta demonstrar que $A \subset C$.
	Para isso, considere $x \in A$ um elemento arbitrário e mostremos que $x \in C$.
	Como temos que $A \subset B$, podemos concluir que $x \in B$.
	E, como $x \in B$ e $B \subset C$, segue que $x \in C$.
	Portanto, $A \subset C$.
\end{proof}

\begin{definition}[Conjunto das Partes]
\label{def:powerset}
Dado um conjunto $A$, chamamos de \textdef{conjunto das partes} de $A$ o conjunto formado por todos os seus subconjuntos, e denotamo-lo $\powerset{A}$.
\end{definition}

Em outras palavras, podemos definir uma condição necessária e sucifiente para um elemento pertencer a $\powerset{A}$, para isso, considere X um objeto arbitrário, logo:
$$ X \in \powerset{A} \sse X \subset A $$

\begin{example}
\label{exem-powerset-basico}
Dado $A = \set{1,2,3}$, determine $\powerset A$.
\end{example}

\begin{solution}

$\powerset A=\{\emptyset,\{1\}, \{2\}, \{3\}, \{1,2\}, \{2,3\}, \{1,3\},A\}$.
\end{solution}

\begin{remark}
Como o Exemplo~\ref{exem-powerset-basico} ilustra, o conjunto das partes de um determinado conjunto tem a particularidade de que todos os seus elementos também são conjuntos.
\end{remark}
