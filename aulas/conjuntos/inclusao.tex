\section{Inclusão}

\begin{definition}
Sejam $A$ e $B$ conjuntos. Se todo elemento de $A$ for, também, elemento de $B$, diz-se que $A$ é um \textdef{subconjunto} de $B$, que $A$ está \textdef{contido} em $B$, ou que $A$ é \textdef{parte} de $B$. Para indicar esse fato, usa-se a notação $A \subset B$.
\end{definition}

\begin{remark}
Quando $A$ não é um subconjunto de $B$, escreve-se $A \notsubset B$. Em outras palavras, existe pelo menos um elemento $a$ tal que $a \in A$ e $a \notin B$.
\end{remark}

\begin{definition}
Quando $A \subset B$, dizemos que $B$ \textdef{contém} $A$, e escrevemos $B \supset A$.
\end{definition}

\begin{example}
Sejam $T$ o conjunto de todos os triângulos e $P$ o conjunto dos polígonos no plano. Todo triângulo é um polígono, logo, $T \subset P$.
\end{example}

\begin{example}
Na Geometria, uma reta, um plano e o espaço são conjuntos. Seus
elementos são pontos. Quando dizemos que uma reta $r$ está no plano $\Pi$, estamos afirmando que $r$ está contida em $\Pi$ ou, equivalentemente, que $r$ é um subconjunto de $\Pi$, pois todos os pontos que pertencem a $r$ pertencem, também, a $\Pi$. Nesse caso, deve-se escrever $ r \subset \Pi$. Porém, não é correto dizer que $r$ pertence a $\Pi$, nem escrever $r \in \Pi$. Os
elementos do conjunto $\Pi$ são pontos, não retas.
\end{example}

\begin{proposition}[Inclusão universal do $\emptyset$]
Para todo conjunto $A$, vale $\emptyset \subset A$.
\end{proposition}

\begin{proof}
Seja $A$ um conjunto. Suponha, por absurdo, que $\emptyset \notin A$. Logo, existe $x \in \emptyset$ tal que $x \notin A$. Isso é um absurdo pois $\emptyset$ não possui elementos. Portanto, $\emptyset \subset A$.
\end{proof}

\begin{definition}
Dizemos que $A$ é um \textdef{subconjunto próprio} de $B$ quando $A \subset B$ e $A \neq B$. Quando isso ocorre ultiliza-se a notação $A \subsetneq B$.
\end{definition}

\begin{proposition}[Propriedades da inclusão]
Sejam $A$, $B$ e $C$ conjuntos. Tem-se:
%
\begin{enumerate}
\item \textdef{Reflexividade}: $A \subset A$;
\item \textdef{Antissimetria}: Se $A \subset B$ e $B \subset A$, então $A = B$;
\item \textdef{Transitividade}: $A \subset B$ e $B \subset C$, então $A \subset C$.
\end{enumerate}
\end{proposition}

\begin{proof}
\begin{enumerate}
	\item[] 
	\item Seja $x \in A$. Então, temos $x \in A$. Portanto, $A \subset A$.
	\item Sejam $A$ e $B$ conjuntos tais que $A \subset B$ e $B \subset A$. Suponha, por contradição, que $A \ne B$, ou seja, existe $x \in A$ tal que $x \notin B$ (1) ou existe $x \in B$ tal que $x \notin A$ (2). Ora, (1) é o mesmo que $A \notsubset B$, contradizendo $A \subset B$. Analogamente, (2) contradiz $B \subset A$. Portanto, $A = B$.
	\item Sejam $A$, $B$ e $C$ conjuntos tais que $A \subset B$ e $B \subset C$. Seja $x \in A$. Então $x \in B$ pois $A \subset B$. Como $x \in B$, segue que $x \in C$, pois $B \subset C$. Portanto, $A \subset C$.
\end{enumerate}
\end{proof}

\begin{definition}
Dado um conjunto $A$, chamamos de \textdef{conjunto das partes} de $A$ o conjunto formado por todos os seus subconjuntos, e denotamo-lo $\mathcal{P}(A)$.
\end{definition}

\begin{example}
Dado $A = \set{1,2,3}$, determine $\powerset A$.
\end{example}

\begin{solution}
$\powerset A=\{\emptyset,\{1\}, \{2\}, \{3\}, \{1,2\}, \{2,3\}, \{1,3\},A\}$.
\end{solution}
