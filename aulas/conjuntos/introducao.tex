\section{Introdução}
Um \textdef{conjunto} é definido por seus elementos (e nada mais). Isso nos traz imediatamente que dois conjuntos são \textdef{iguais} se, e somente se, possuem os mesmos elementos.

Dados um conjunto $A$ e um objeto qualquer $b$, há somente uma pergunta cabível para nós: $b$ é um elemento do conjunto $A$? Tal pergunta só admite ``sim'' ou ``não'' como resposta. Isso se dá porque, na Matemática, qualquer afirmação é verdadeira ou é falsa, sem possibilidade de uma terceira opção ou de ser as duas coisas ao mesmo tempo. 

O caráter binário e exclusivo do valor-verdade de afirmações faz parecer que a Matemática é infalível se usada corretamente, mas ela não é. O matemático austríaco Kurt Gödel provou, em 1931, que todo sistema formal é falho no sentido de que vai possuir verdades que não podem ser provadas -- os chamados paradoxos. Antes de assistir ao vídeo \href{https://youtu.be/UI1xR_AECrU}{{\tt Este vídeo está mentindo}}, reflita se você vai acreditar nele ou não.

\begin{example}
\label{ex-primos-pares}
O conjunto $PP$ dos números primos pares pode ser representado por $PP = \set{x \tq x \text{ é primo e par }} = \set{2}$. Nunca escreva $PP = \set{\text{ números primos pares }}$.
\end{example}

\begin{example}
\label{ex-vogais}
Temos $V=\set{\texttt{a}, \texttt{e}, \texttt{i}, \texttt{o}, \texttt{u}}$ como sendo o conjunto das vogais.
\end{example}

\begin{remark}
Quando um elemento pertence a um determinado conjunto, usamos o símbolo $\in$, e, quando não pertence, usamos $\notin$.
\end{remark}

\begin{example}
Considere $PP$ e $V$ conforme definido nos Exemplos \ref{ex-primos-pares} e \ref{ex-vogais}, respectivamente. Temos que $\texttt{e} \in \mathcal{V}$ e $3 \notin PP$.
\end{example}

\begin{definition}
O conjunto que não possui elementos é chamado de \textdef{conjunto vazio} e é representado por $\emptyset$.
\end{definition}

\begin{example}
Quais outros conjuntos você conhece? Que tal pensar sobre o conjunto $A = \set{x \tq x \notin A}$?
\end{example}
