\section{Introdução}
\label{sec:intro}

Um \textdef{conjunto} é definido por seus elementos (e nada mais). Isso nos traz imediatamente que dois conjuntos são \textdef{iguais} se, e somente se, possuem os mesmos elementos.

Dados um conjunto $A$ e um objeto qualquer $b$, há somente uma pergunta cabível para nós: $b$ é um elemento do conjunto $A$? Tal pergunta só admite ``sim'' ou ``não'' como resposta. Isso se dá porque, na Matemática, qualquer afirmação é verdadeira ou é falsa, sem possibilidade de uma terceira opção ou de ser as duas coisas ao mesmo tempo. 

O caráter binário e exclusivo do valor-verdade de afirmações faz parecer que a Matemática é infalível se usada corretamente, mas ela não é. O matemático austríaco Kurt Gödel provou, em 1931, que todo sistema formal é falho no sentido de que vai possuir verdades que não podem ser provadas -- os chamados paradoxos. Antes de assistir ao vídeo \href{https://youtu.be/UI1xR_AECrU}{{\tt Este vídeo está mentindo}}, reflita se você vai acreditar nele ou não.

\begin{remark}
Podemos definir conjuntos de duas formas. Ambas são representadas nos exemplos a seguir.
\end{remark}

\begin{example}
\label{ex-vogais}
Se quisermos expressar qual seria o conjunto de todas as vogais do nosso alfabeto, precisaríamos de alguma notação para representá-lo. 
Temos $V=\set{\texttt{a}, \texttt{e}, \texttt{i}, \texttt{o}, \texttt{u}}$ como sendo o conjunto das vogais. Tal notação lista explicitamente os membros de seu conjunto.
\end{example}

\begin{example}
\label{ex-primos-pares}
O conjunto $PP$ dos números primos pares pode ser representado por $PP = \set{x \tq x \text{ é primo e par}} = \set{2}$. Tal notação é lida por ``o conjunto de todos os $x$'s tais que $x$ é primo e par''. Nunca escreva $PP = \set{\text{ números primos pares }}$.
\end{example}

% @TODO mais um exemlo
%\begin{example}
%
%\end{example}
