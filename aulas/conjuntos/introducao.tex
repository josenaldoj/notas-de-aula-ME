\section{Introdução}
\label{sec:intro}

Um \textdef{conjunto} é definido por seus elementos, e nada mais.
Desse fato decorre, imediatamente, que dois conjuntos são \textdef{iguais} quando possuem os mesmos elementos, e apenas nessa situação. 

Dados um conjunto $A$ e um objeto qualquer $b$, é natural perguntar se $b$ é um elemento do conjunto $A$.
Tal questionamento admite apenas ``sim'' ou ``não'' como candidatos a resposta.
Isso se dá porque, na Matemática, qualquer afirmação é verdadeira ou é falsa, não havendo possibilidade de ser as duas coisas simultaneamente nem de, ainda, haver uma terceira opção. 

O caráter binário e exclusivo do valor-verdade das afirmações faz parecer que a Matemática é infalível se usada corretamente, o que não se verifica de fato.
O matemático austríaco Kurt Gödel provou, em 1931, que todo sistema formal é falho pois possui verdades que não podem ser provadas -- os chamados paradoxos.
Antes de assistir ao vídeo \href{https://youtu.be/UI1xR_AECrU}{{\tt Este vídeo está mentindo}}, reflita se você vai acreditar nele ou não.

A descrição dos elementos de um conjunto -- necessária para defini-lo, conforme já comentado -- pode ser feita textualmente.
Contudo, nestas notas serão utilizadas duas formas matemáticas comuns de se especificar tal descrição, apresentadas nos Exemplos \ref{ex-vogais} e \ref{ex-primos-pares}. 

\begin{example}
\label{ex-vogais}
Se quisermos expressar qual seria o conjunto de todas as vogais do nosso alfabeto, precisaríamos de alguma notação para representá-lo. 
Temos $V=\set{\texttt{a}, \texttt{e}, \texttt{i}, \texttt{o}, \texttt{u}}$ como sendo o conjunto das vogais. Tal notação lista explicitamente os membros de seu conjunto.
\end{example}

\begin{example}
\label{ex-primos-pares}
O conjunto $PP$ dos números primos pares pode ser representado por $PP = \set{x \tq x \text{ é primo e par}} = \set{2}$.
Tal notação é lida por ``o conjunto de todos os $x$'s tais que $x$ é primo e par''.
Nunca escreva $PP = \set{\text{ números primos pares }}$.
\end{example}

% @TODO mais um exemlo
%\begin{example}
%
%\end{example}
