\section{Inequação do 1° grau}

\begin{definition}
Uma \textdef{inequação do primeiro grau} é uma relação de uma das seguintes formas:

\begin{align*}
& ax+b <0;\\
& ax+b >0;\\
& ax+b \le 0;\\
& ax+b \ge 0;
\end{align*}
%
onde $a, b \in \R$ e $ a \neq 0$. Lemos os símbolos da seguinte maneira: $<$ (menor que), $>$ (maior que), $\leq$ (menor ou igual
que) e $\geq$ (maior ou igual que).
\end{definition}

\begin{remark}
O \textdef{conjunto solução} de uma inequação do primeiro grau é o conjunto $S$ de números reais que satisfazem a inequação, isto é, o conjunto de números que, quando substituídos na inequação, tornam a desigualdade verdadeira.
\end{remark}

\begin{proposition}
Sejam $a, b, c, d \in \R$; $n \in \N^*$. Os seguintes valem:
%
\begin{enumerate}[i.]
  \item Invariância por adição de números reais: $a < b \implies a+c < b+c$;
  \item Invariância por multiplicação de números reais positivos:
  $a < b ; c>0 \implies a \cdot c < b \cdot c$;
  \item Mudança por multiplicação de números reais
  negativos: $a < b ; c<0 \implies a \cdot c > b \cdot c$;
  \item Se $a < b$, então $\frac 1 a > \frac 1 b$, para $a, b \neq
  0$;
  \item Se $a,b \geq 0$ e $c>0$, segue que $a < b \implies a^c < b^c$;
  \item Se $a,b < 0$ e $n$ par, segue que $a < b \implies a^n > b^n$;
  \item Se $a,b < 0$ e $n$ ímpar, segue que $a < b \implies a^n <
  b^n$;
  \item Se $a< b$ e $c< d$, então $a+c < b+d$;
  \item Para $a, b, c, d \in \R_+$. Se $a< b$ e $c< d$, então $ac < bd$.
\end{enumerate}
%
Os resultados são análogos para os tipos $>$, $\leq$ e $\geq$.
\end{proposition}

\begin{example}
Qual o conjunto solução da inequação $8x - 4 \ge 0$?
\end{example}

% Exemplo 15
\begin{solution}
Note que:
%
\begin{align*}
	8x-4 \ge 0 & \iff 8x \ge 4\\
	& \iff x \ge \dfrac{1}{2}
\end{align*}
%
Logo, o conjunto-solução da equação é $S = \set{x \in \R \tq x \ge 1/2}$. A seguir, é exibida uma resolução alternativa da inequação:
%
\begin{align*}
	8x-4 \ge 0 & \iff -4 \ge -8x\\
	& \iff \dfrac{-4}{-8} \le x \\
	& \iff \dfrac{1}{2} \le x 
\end{align*}
\end{solution}	

\begin{example}
Antes de fazer os cálculos, diga qual dos números $a = 3456784 \cdot 3456786 + 3456785$ e $b = 3456785^2 - 3456788$ é maior.
\end{example}

% Exemplo 16
\begin{solution}
Suponha que $a > b$. Faça $x=3456784$. Teremos:
%
\begin{align*}
	x(x+2)+x+1>(x+1)^2 -(x+4)& \iff x(x+1+1)+x+1>(x+1)^2-\prn{x+4}\\
	& \iff x\prn{x+1}+x+\prn{x+1}>\prn{x+1}^2-x-4\\
	& \iff \prn{x+1}\prn{x+1}+x>\prn{x+1}^2-x-4\\
	& \iff \prn{x+1}^2>\prn{x+1}^2-2x-4\\
	& \iff 0>-2x-4\\
	& \iff 4>-2x\\
	& \iff -2<x,
\end{align*}
%
\noindent o que é uma verdade pois $x=3456784$. Logo, $a>b$.
\end{solution}


\begin{onlineact}[\khan{https://pt.khanacademy.org/math/cc-sixth-grade-math/cc-6th-equations-and-inequalities/cc-6th-inequalities/e/inequalities-in-one-variable-1}{Problemas com Inequações}]
Veja o desempenho na Missão 7\tdeg{} Ano -- Introdução às Equações e Inequações.
\end{onlineact}