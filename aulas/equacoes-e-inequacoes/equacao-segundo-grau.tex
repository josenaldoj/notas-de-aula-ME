\section{Equação do 2° grau}

\begin{definition}
\label{def:eq-2-grau}
A \textdef{equação do segundo grau} com coeficientes $a$, $b$ e $c$ é uma equação da forma 
%
\begin{align*}
ax^2 + bx + c = 0,
\end{align*}
%
\noindent onde $a, b, c \in \R$, $a \neq 0$ e $x$ é uma variável real a ser determinada.
\end{definition}

\begin{example}
\label{ex:sol-2-grau}
Encontre as soluções de uma equação do segundo grau.
\end{example}

\begin{solution}
Da Definição \ref{def:eq-2-grau}, sabemos que uma equação do segundo grau tem a seguinte forma:
%
\begin{align*}
	ax^2 + bx + c = 0,
\end{align*}
%
onde $a,b,c \in \R$, com $a \ne 0$. Manipulemos a equação para encontrar o valor de $x$:
%
\begin{align*}
ax^2 + bx + c = 0 & \iff x^2+\frac{b}{a}x +\frac{c}{a}=0 \\ 
	& \iff x^2+2\frac{b}{2a}x+\prn{\frac{b}{2a}}^2-\prn{\frac{b}{2a}}^2+\frac{c}{a}=0\\ 
	& \iff \prn{x+\frac{b}{2a}}^2=\frac{b^2}{4a^2}-\frac{c}{a}=\frac{b^2 -4ac}{4a^2}\\ 
	& \iff \abs{x+\frac{b}{2a}} = \sqrt{\frac{b^2-4ac}{4a^2}}\\ 
	& \iff x+\frac{b}{2a}=\pm\sqrt{\frac{b^2-4ac}{4a^2}}=\frac{\pm\sqrt{b^2-4ac}}{\pm2a}=\pm\frac{\sqrt{b^2-4ac}}{2a}\\ 
	& \iff x=\frac{-b\pm\sqrt{b^2-4ac}}{2a}\\ 
\end{align*}
%	
Portanto, quando $b^2-4ac \ge 0$, o conjunto-solução $S$ da equação será:
%
\begin{align*}
 S = \set{\dfrac{-b-\sqrt{b^2-4ac}}{2a}, \dfrac{-b+\sqrt{b^2-4ac}}{2a}}
\end{align*}
\end{solution}

\begin{onlineact}[\khan{https://pt.khanacademy.org/math/algebra/quadratics/quadratics-square-root/e/quadratics-by-taking-square-roots-with-steps}{Equações do segundo grau com cálculo de
raízes quadradas: com etapas}]
\end{onlineact}

\begin{onlineact}[\khan{https://pt.khanacademy.org/math/algebra/quadratics/solving-quadratics-by-completing-the-square/e/completing_the_square_2}{Método de completar quadrados}]
Veja o desempenho na Missão Álgebra I -- Equações do segundo grau.
\end{onlineact}

\begin{definition}
Chamamos de \textdef{discriminante} da equação do segundo grau a expressão $b^2-4ac$ e denotamos pela letra grega maiúscula $\Delta$ (lê-se delta).
\end{definition}

\begin{remark}
O número de soluções de uma equação do segundo grau é totalmente determinado pelo sinal do seu discriminante, de forma tal que:
%
\begin{itemize}
  \item Se $\Delta > 0$, existem duas soluções reais;
  \item Se $\Delta = 0 $, existe uma solução real ($x_1 = x_2 =
  -b/2a)$;
  \item Se $\Delta < 0$, não existe solução real.
\end{itemize}
\end{remark}

\begin{example}
Sabendo que $x$ é um número real que satisfaz a equação: 
%
\begin{align*}
x = 1 + \frac 1 {1 +\frac 1 x},
\end{align*}
%
\noindent determine os valores possíveis de $x$.

\end{example}

\begin{solution}
Manipulemos a equação:
%
\begin{align*}
x = 1+\dfrac{1}{1+\dfrac{1}{x}} & \iff x\prn{1+\dfrac{1}{x}} = \prn{1+\dfrac{1}{x}}+1\\
	& \iff x+1 = \dfrac{1}{x}+2\\
	& \iff x^2+x=1+2x\\
	& \iff x^2-x-1=0\\
	& \iff x=\dfrac{1\pm\sqrt5}{2}
\end{align*}
%
Logo, as soluções são $\prn{1-\sqrt5}/2$ e $\prn{1+\sqrt5}/2$.
\end{solution}

\begin{remark}
O número $\phi = \frac{\prn {1+\sqrt 5}}2$ é conhecido como razão áurea, número de ouro, proporção divina, entre outras denominações.
Veja o episódio A Proporção Divina \link{https://www.youtube.com/watch?v=mfL6-g5mQw4}{parte 01} e \link{https://www.youtube.com/watch?v=xtsTXAwWF20&}{parte 02} do programa português Isto É Matemática.
\end{remark}

\begin{onlineact}[\khan{https://pt.khanacademy.org/math/algebra/quadratics/solving-quadratics-using-the-quadratic-formula/e/quadratic_equation}{Fórmula de Bhaskara}]
Veja o desempenho na Missão Álgebra I -- Equações do segundo grau.
\end{onlineact}

\begin{theorem}
Os números $\alpha$ e $\beta$ são as raízes da equação do segundo grau:
%
\begin{align*}
ax^2+ bx+c=0
\end{align*}
se, e somente se,
\begin{align*}
\alpha + \beta = \frac {-b} a \; \text{ e } \; \alpha \beta = \frac c a.
\end{align*}
\end{theorem}

\begin{proof}
Sejam $\alpha$ e $\beta$ raízes da equação do 2\tdeg{} grau $ax^2+bx+c=0$. Do Exemplo \ref{ex:sol-2-grau}, temos:
%
\begin{align*}
	& \alpha+\beta=\dfrac{-b+\sqrt\Delta}{2a}+\dfrac{-b-\sqrt\Delta}{2a}=\dfrac{-2b}{2a}=\dfrac{-b}{a}\\
	& \alpha\beta=\dfrac{c}{a} \text{ (exercício)}
\end{align*}
\end{proof}