\section{Progressão Geométrica}

\begin{example}
Uma pessoa, começando com R\$ $64{,}00$, faz seis apostas consecutivas, em cada uma das quais arrisca perder ou ganhar a metade do que possui na ocasião. Se ela ganha três e perde três dessas apostas, pode-se afirmar que ela:
%
\begin{enumerate}[a)]
  \item Ganha dinheiro;
  \item Não ganha nem perde dinheiro;
  \item Perde R\$ $27{,}00$;
  \item Perde R\$ $37{,}00$;
  \item Ganha ou perde dinheiro, dependendo da ordem em que ocorreram suas vitórias e derrotas.
\end{enumerate}
\end{example}

\begin{solution}
Se a pessoa perdesse as três primeiras apostas e, depois, ganhasse as outras três, ela ficaria, em reais, com:
%
\begin{align*}
64 \cdot \frac 1 2 \cdot \frac 1 2 \cdot \frac 1 2 \cdot \frac 3 2\cdot \frac 3 2\cdot \frac 3 2 &= 64 \cdot \frac {27} {64} \\ &= 27
\end{align*}
%
Note que, como a multiplicação de números reais é comutativa, a ordem das apostas não importa. Assim, a pessoa perdeu R\$ $37{,}00$.
\end{solution}