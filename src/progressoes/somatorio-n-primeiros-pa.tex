\section{Somatório dos $n$ primeiros termos de uma PA}

\begin{proposition}
\label{proposition:soma-n-termos-pa}
A soma dos $n$ primeiros termos da PA $(a_1, a_2, a_3, \dots)$ é:
%
\begin{equation*}
S_n = \frac {\prn {a_1+a_n}n} 2.
\end{equation*}
\end{proposition}

\begin{corollary}
Nas condições da Proposição \ref{proposition:soma-n-termos-pa}, tem-se que:
%
\begin{equation*}
S_n = \frac r 2 \cdot n^2 + \prn{a_1 - \frac r 2}n.
\end{equation*}
\end{corollary}

\begin{remark}
Todo polinômio de segundo grau em $n$ que não possua termo independente nulo é o somatório de alguma PA. De fato, tendo $P(n) = an^2 + bn$, basta tomar $r = 2a$ e $a_1 = a+b$. Verifique!
\end{remark}