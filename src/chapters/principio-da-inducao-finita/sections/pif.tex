\section{Princípio da Indução Finita}

O Princípio da indução finita é um Teorema muito usado. Nosso objetivo não é demonstrá-lo. Queremos usá-lo em situações e problemas mais elementares para possibilitar o entendimento e uso básico dessa poderosa ferramenta.

\begin{theorem}[Princípio da Indução Finita]
\label{theorem:pif}
Considere $n_0$ um inteiro não negativo. Suponhamos que, para cada inteiro $n \geq n_0$, seja dada uma proposição $p \prn n$. Suponha
que se pode verificar as seguintes propriedades:

\begin{enumerate}[(a)]
  \item $p \prn{n_0}$ é verdadeira;
  \item Se $p \prn n$ é verdadeira, então $p \prn {n+1}$ também
  é verdadeira, para todo $n \geq n_0$.
\end{enumerate}

\noindent Então, $p \prn n$ é verdadeira para qualquer $n \geq n_0$.
\end{theorem}

\begin{remark}
No Teorema \ref{theorem:pif}, a afirmação (a) é chamada de \textdef{base da indução}, e a (b), de \textdef{passo indutivo}. O fato de que $p \prn n$ é verdadeira no item (b) é chamado de \textdef{hipótese de indução}.
\end{remark}

\begin{example}
Demonstre que, para qualquer $n \in \N^*$, é válida a igualdade:
%
$$2+ 4+ \dots + 2n = n \prn {n+1}.$$
\end{example}


\begin{solution}
	A fim de provar que a soma dos $n$ primeiros números pares é igual a $n(n+1)$, ou seja, $2+4+\dots + 2n = n(n+1)$ para todo $n \in \nnats$, aplicaremos indução em $n$.
	
	\case{Caso base $[n=1]$}
	Observe que $2 = 1(1+1)$, ou seja, a soma do primeiro número par (somente o 2) é igual a $1(1+1)$. Logo, o caso base é válido.
	
	\case{Passo indutivo}
	Suponha, como Hipótese de Indução (HI), que para algum $n \in \nnats$, vale a equação:
	$$2+4+\dots+2n=n(n+1).$$
	Provemos então que:
	$$2+4+\dots + 2n + 2(n+1) = (n+1)[(n+1)+1].$$
	De fato, segue da (HI) que:
	\begin{align*}
	\underbrace{2+4+\dots+2n}_{\text{HI}} +2(n+1)	&= \underbrace{n(n+1)}_{\text{HI}} + 2(n+1) \\
	&= (n+1)(n+2) \\
	&= (n+1)[(n+1)+1].
	\end{align*}
	Com isso, concluímos que o passo indutivo é satisfeito.
	Portanto, pelo Princípio da Indução Finita, a equação $2+4+\dots + 2n=n(n+1)$ é válida para todo $n \in \nnats$.
	\end{solution}


\begin{example}
Demonstre que, para qualquer $n \in \N^*$, é válida a igualdade:
$$1+3+\dots +\prn {2n-1} = n^2.$$
\end{example}

\begin{solution}
Demonstremos que a igualdade vale aplicando Indução em $n$.

\case{Caso base $[n=1]$}
Como $1=1^2$, o caso base é válido.

\case{Passo indutivo}
Assuma, como Hipótese de Indução (HI), que a igualdade vale para algum $n\in \nnats$, ou seja:
$$1+3+\dots+(2n-1) = n^2.$$
Provemos a validade da equação:
$$1+3+\dots+(2n-1)+[2(n+1)-1] = (n+1)^2.$$
Calculando o lado esquerdo da igualdade, obtemos:
\begin{align*}
\underbrace{1+3+\dots+(2n-1)}_{\text{HI}} + [2(n+1)-1] & = \underbrace{n^2}_{\text{HI}} + 2n + 1 \\
& = (n + 1)^2
\end{align*}
Com isso, provamos que o passo indutivo é válido.
Portanto $p(n)$ vale para todo $n \in \nnats$.
\end{solution}

\begin{example}
Mostre que, para todo número $n \in \nnats$ tal que $n>3$ vale:
$$2^n < n!$$
\end{example}

\begin{solution}
Considere $n \in \nnats$ tal que $n >3$. Provemos que $2^n < n!$ aplicando Indução em $n$.

\case{Caso base $[n=4]$} Temos $2^4 = 16$ e $4! = 24$. Como $16 < 24$, o caso base é satisfeito.

\case{Passo Indutivo}
Suponha, como Hipótese de Indução, que vale $2^n < n!$ para algum $n \in \nnats$ tal que $n > 3$.

Provemos que vale $2^{n+1} < (n+1)!$.
Inicialmente, observe que $2 < n + 1$, visto que $2 < 3 < n < n + 1$. Também temos que $2$ e $2^n$ são positivos. Sendo assim, podemos concluir que:
\begin{align*}
2^n < n! \text{ e } 2 < n+1 & \implies 2^n \cdot 2 < n!(n+1) \\
& \implies 2^{n+1} < (n+1)!.
\end{align*}
Assim provando que vale o passo indutivo.
Portanto, $2^n < n!$ para qualquer $n \in \nnats$ tal que $n>3$.
\end{solution}

\begin{example}
Prove  que, para todo $n \in \nnats$,
%
\begin{equation*}
\underbrace{\sqrt{2+\sqrt{2+\sqrt{2+ \dots + \sqrt 2}}}}_{n  \text{ radicais}} < 2.
\end{equation*}
\end{example}

\begin{solution}
Aplicaremos o Princípio da Indução Finita em $n \in \nnats$.
%
\begin{itemize}
	\item \textit{Caso base} ($n=1$):

	$\sqrt 2 < 2$ é válido.

	\item \textit{Passo de indução}:

	Suponha a validade da inequação para algum $k \in \nnats$, ou seja, 
	%
	\begin{equation*}
	\underbrace{\sqrt{2+\sqrt{2+\sqrt{2+ \dots + \sqrt 2}}}}_{k  \text{ radicais}} < 2.
	\end{equation*}
	%
	Da hipótese de indução, segue que:
	%
	\begin{align*}
	2+\underbrace{\sqrt{2+\sqrt{2+\sqrt{2+ \dots + \sqrt 2}}}}_{k  \text{ radicais}} < 2+2 & \implies \underbrace{\sqrt{2+\sqrt{2+\sqrt{2+ \dots + \sqrt 2}}}}_{k+1  \text{ radicais}} < \sqrt 4 = 2.
	\end{align*}
\end{itemize}
%
Logo, a inequação é válida para $n = k+1$; e, assim, concluímos a validade da inequação para todo $n \in \nnats$.
\end{solution}

\begin{example}
Seja $n \in \N$ tal que $n\ge 3$. Mostre  que podemos cobrir os $n^2$ pontos no reticulado a seguir traçando $2n-2$ segmentos de reta sem tirar o lápis do papel.
%
\begin{equation*}
\underbrace{\begin{array}{ccccc}
                \bullet & \bullet & \bullet & \bullet & \bullet \\
                \bullet & \bullet & \bullet & \bullet & \bullet \\
                \bullet & \bullet & \bullet & \bullet & \bullet \\
                \bullet & \bullet & \bullet & \bullet & \bullet \\
                \bullet & \bullet & \bullet & \bullet & \bullet
              \end{array}
}_{n \times n \text{ pontos}}
\end{equation*}
\end{example}

\begin{example}
Um rei muito rico possui $3^n$ moedas de ouro, onde $n \in \nnats$. No entanto, uma dessas moedas é falsa, e seu peso é menor que o peso das demais. Com uma balança de dois pratos e sem nenhum peso, mostre que é possível encontrar a moeda falsa com apenas $n$ pesagens.
\end{example}

\begin{solution}
Dado que o rei tem $3^n$ moedas, usaremos indução finita em $n \in \nnats$. 

\begin{itemize}
	\item \textit{Caso base} ($n=1$): Para $n=1$, ou seja, três moedas, coloca-se uma moeda em cada prato. Se as moedas tiverem pesos iguais, então a moeda falsa é a que não foi colocada na balança. Do contrário, a moeda falsa é a mais leve.

	\item \textit{Passo de indução} Como hipótese de indução, suponhamos que, para algum $n \in \nnats$, seja possível identificar a moeda falsa dentre $3^n$ moedas realizando-se $n$ pesagens. Suponha que temos $3^{n+1}$ moedas. 

	Separando as moedas em três grupos com $3^n$ moedas cada, coloca-se um grupo em cada prato da balança. Assim, analogamente ao caso $n = 1$, identificamos o grupo com a moeda falsa. Tal grupo tem $3^n$ moedas e, pela hipótese de indução, pode-se identificar a moeda falsa com mais $n$  pesagens, totalizando $n+1$ pesagens. Concluímos assim a veracidade do passo de indução.

		Portanto, para cada $n \in \nnats$, é possível identificar a moeda falsa dentre $3^n$ moedas com apenas $n$ pesagens.
\end{itemize}
\end{solution}

\begin{theorem}
Para quaisquer $a_1, a_2, \dots , a_n \in \R_+$, vale:
%
\begin{equation*}
    \sqrt[n]{a_1\dots a_n} \le \frac {a_1 + \dots + a_n} n.
\end{equation*}
\end{theorem}

\begin{proof}
	Provemos, inicialmente, que a desigualdade é válida quando $n$ é uma potência de 2.
	Para tal, seja $n = 2^m$, com $m \in \N$.
	Usaremos o Princípio da Indução Finita em $m$.

	\begin{itemize}
		\item \emph{Caso base} ($m=0$). Para $m=0$, temos $n=1$. 
		Seja $a_1 \in \R_+$. 
		Note que $\sqrt[1]{a_1}=a_1\le \frac{a_1}{1}$.
		Assim, a desigualdade é válida para $m=0$.

		\item \emph{Passo de Indução}. Como hipótese de indução (HI), suponha, para algum $m=k\in \N$ --- ou seja, $n=2^k$ ---, que vale:
		$$
		\sqrt[2^k]{a_1 \cdot \dots \cdot a_{2^k}}\le \frac{a_1 + \dots + a_{2^k}}{2^k}
		$$

		Agora, provemos que a desigualdade é válida para $m=k+1$ --- isso é, $n = 2^{k+1}$---. De fato,
		%
		\begin{align*}
		\sqrt[2^{k+1}]{a_1 \cdot \ldots \cdot a_{2^{k+1}}} &= \sqrt[2]{\sqrt[2^k]{a_1 \cdot \ldots \cdot a_{2^{k+1}}}}&  \\
		& = \sqrt[2]{ \sqrt[2^k]{a_1 \cdot \ldots \cdot a_{2^k}} \cdot \sqrt[2^k]{a_{2^k+1} \cdot \ldots \cdot a_{2^{k+1}}}}& \\
		& \le \frac{\sqrt[2^k]{a_1 \cdot \ldots \cdot a_{2^k}} + \sqrt[2^k]{a_{2^k+1} \cdot \ldots \cdot a_{2^{k+1}}}  }{2} & \text{(Teorema \ref{theo:desigualdade-medias-dois-termos})}\\
		& \le \frac{ \frac{a_1 + \ldots + a_{2^k}}{2^k} + \frac{a_{2^k+1}+\ldots+a_{2^{k+1}}}{2^k}    }{2} & \text{(HI)}\\
		& = \frac{\frac{a_1 + \ldots + a_{2^{k+1}}}{2^k}}{2} & \\
		& = \frac{a_1 + \ldots + a_{2^{k+1}}}{2^{k+1}}  &
		\end{align*}
		
	\end{itemize}

	Dessa forma, a desigualdade vale quando $n$ é uma potência de 2.

	Provemos, agora, o caso geral.
	Para tanto, seja $n \in \N$ arbitrário.
	Tomemos o menor $m \in \N$ tal que $n \le 2^m$.
	Além disso, definamos $L = \sqrt[n]{a_1 \cdot \ldots \cdot a_n}$.
	Pelo que acabamos de provar, é válido que:

	$$
	\frac{a_1 + \ldots + a_n + \overbrace{L+\ldots + L}^{2^m - n \text{ vezes}}}{2^m} \ge \sqrt[2^m]{a_1 \cdot \ldots +\cdot a_n \cdot L^{2^m-n}}
	$$

	No entanto, note que:
	%
	\begin{align*}
		\sqrt[2^m]{a_1 \cdot \ldots +\cdot a_n \cdot L^{2^m-n}} & = \sqrt[2^m]{L^n \cdot L^{2^m-n}} \\ & = \sqrt[2^m]{L^{2^m}} \\ & = L
	\end{align*}
	
	Assim,
	%
	\begin{align*}
		\frac{a_1 + \ldots + a_n + (2^m - n)L}{2^m} \ge L & \implies a_1 + \ldots + a_n \ge 2^m L - (2^m - n)L = n \cdot L\\
		& \implies \frac{a_1 + \ldots + a_n}{n} \ge \sqrt[n]{a_1 \cdot \ldots \cdot a_n}
	\end{align*}

	Portanto, a desigualdade é válida para todo $n \in N$.
\end{proof}