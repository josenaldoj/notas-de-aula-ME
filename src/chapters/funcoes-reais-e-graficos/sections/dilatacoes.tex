\subsubsection{Dilatação}

\begin{example}
Compare os gráficos das funções reais $f, g , h: \R \to \R$ tais que
$f(x) = \sen x$, \\ $g(x) = \frac 1 2 \cdot f(x)  = \frac 1 2 \cdot \sen x $, \\
$h(x)= f(2 \cdot x)= \sen (2 \cdot x)$.
\end{example}

\begin{example}
Compare os gráficos das funções reais $f, g , h: \R \to \R$ tais que
$f(x) = \sen x$, \\ $g(x) = -1 \cdot f(x)  = -1 \cdot \sen x $ , \\
$h(x)= f(-1 \cdot x)= \sen (-1 \cdot x)$.
\end{example}

Dessa forma, se a função real $g$ é tal que $g(x) = c \cdot f(d
\cdot x)$, então o gráfico de $g$ pode ser obtido, do gráfico de
$f$, através de uma dilatação horizontal determinada pelo parâmetro
$d$, e uma dilatação vertical determinada pelo parâmetro $c$. Se o
parâmetro for negativo, haverá, também, uma reflexão.
\begin{itemize}
  \item A dilatação vertical será:
        \begin{itemize}
          \item Um esticamento se $c>1$;
          \item Um encolhimento se $0<c<1$;
          \item Um esticamento composto com reflexão em relação ao eixo $x$ se $c<-1$;
          \item Um encolhimento composto com reflexão em relação ao eixo $x$ se
          $-1<c<0$.
        \end{itemize}
  \item A dilatação horizontal será:
        \begin{itemize}
          \item Um encolhimento se $d>1$;
          \item Um esticamento se $0<d<1$;
          \item Um encolhimento composto com reflexão em relação ao eixo $y$ se $d<-1$;
          \item Um esticamento composto com reflexão em relação ao eixo $y$ se
          $-1<d<0$.
        \end{itemize}
\end{itemize}

\begin{onlineact}
    \khan{https://pt.khanacademy.org/math/algebra2/manipulating-functions/shifting-functions/e/shift-functions}
    {Deslocamento de Funções}.
\end{onlineact}

\begin{onlineact}
    \khan{https://pt.khanacademy.org/math/algebra2/manipulating-functions/stretching-functions/e/shifting_and_reflecting_functions}
    {Como Transformar Funções}.
\end{onlineact}