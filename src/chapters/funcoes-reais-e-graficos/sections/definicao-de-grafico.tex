\subsection{Definição}

\begin{definition}
    O \emph{gráfico} de uma função real $f : D \subset \R \to \R$ é o seguinte subconjunto do plano
cartesiano $\R^2$: $$G(f) = \set{(x,y) \in \R^2 \tq x \in D , y =
f(x)}.$$
\end{definition}

\begin{remark}
    O gráfico de uma função pode ser visto como o lugar geométrico
dos pontos cujas coordenadas satisfazem sua lei de associação.
\end{remark}

\begin{example}
    Esboce o gráfico da função real
    $$\begin{array}{cccl}
    f : & \R^\ast & \to     & \R \\
        &  x & \mapsto & \begin{cases}
                            +1,  &  \ \text{ se } x >0 \\
                            -1, &  \ \text{ se } x <0
                            \end{cases}
    \end{array}.$$
\end{example}

\begin{solution}
    Na Imagem~\ref{img:grafico-funcao-zeroum}, é mostrado o gráfico de $f$.

    \begin{figure}[ht]
    \centering
        \importtikz{grafico-funcao-zeroum}
    \caption{Gráfico da função $f$.}
    \label{img:grafico-funcao-zeroum}
    \end{figure}

    Note que o valor de $f(x)$ não é definido para $x=0$.
    Para deixar claro que o gráfico não possui nenhum ponto com coordenada $x$ com valor $0$, foram colocadas, na imagem, circunferências --- comumente chamadas de bolas abertas --- nos pontos $(0,1)$ e $(0,-1)$, indicando que eles não pertencem ao gráfico de $f$.
\end{solution}