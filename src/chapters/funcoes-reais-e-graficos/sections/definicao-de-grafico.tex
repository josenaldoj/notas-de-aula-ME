\subsection{Definição}

\begin{definition}
    O \textdef{gráfico} de uma função real $f : D \subset \R \to \R$ é o seguinte subconjunto do plano
cartesiano $\R^2$: $$G(f) = \set{(x,y) \in \R^2 \tq x \in D , y =
f(x)}.$$
\end{definition}

\begin{remark}
    O gráfico de uma função pode ser visto como o lugar geométrico
dos pontos cujas coordenadas satisfazem sua lei de associação.
\end{remark}

\begin{example}
    Esboce o gráfico da função real
    $$\begin{array}{cccl}
    f : & \R^\ast & \to     & \R \\
        &  x & \mapsto & \begin{cases}
                            +1,  &  \ \text{ se } x >0 \\
                            -1, &  \ \text{ se } x <0
                            \end{cases}
    \end{array}.$$
\end{example}

\begin{solution}
    Segue conforme a imagem a seguir:

    \begin{center}
        

\tikzset{every picture/.style={line width=0.75pt}} %set default line width to 0.75pt        

\begin{tikzpicture}[x=0.75pt,y=0.75pt,yscale=-1,xscale=1]
%uncomment if require: \path (0,328); %set diagram left start at 0, and has height of 328

%Straight Lines [id:da06701252801465996] 
\draw    (340,300) -- (340,22) (336,240) -- (344,240)(336,180) -- (344,180)(336,120) -- (344,120)(336,60) -- (344,60) ;
\draw [shift={(340,20)}, rotate = 450] [fill={rgb, 255:red, 0; green, 0; blue, 0 }  ][line width=0.75]  [draw opacity=0] (8.93,-4.29) -- (0,0) -- (8.93,4.29) -- cycle    ;

%Straight Lines [id:da13690129851328403] 
\draw    (160,180) -- (518,180) (220,176) -- (220,184)(280,176) -- (280,184)(340,176) -- (340,184)(400,176) -- (400,184)(460,176) -- (460,184) ;
\draw [shift={(520,180)}, rotate = 180] [fill={rgb, 255:red, 0; green, 0; blue, 0 }  ][line width=0.75]  [draw opacity=0] (8.93,-4.29) -- (0,0) -- (8.93,4.29) -- cycle    ;

%Straight Lines [id:da8459758682723727] 
\draw [line width=0.75]    (337.65,240) -- (160,240) ;

\draw [shift={(340,240)}, rotate = 180] [color={rgb, 255:red, 0; green, 0; blue, 0 }  ][line width=0.75]      (0, 0) circle [x radius= 3.35, y radius= 3.35]   ;
%Straight Lines [id:da15248083747856067] 
\draw [line width=0.75]    (342.35,120) -- (520,120) ;

\draw [shift={(340,120)}, rotate = 0] [color={rgb, 255:red, 0; green, 0; blue, 0 }  ][line width=0.75]      (0, 0) circle [x radius= 3.35, y radius= 3.35]   ;

% Text Node
\draw (393,191) node [scale=0.9,color={rgb, 255:red, 0; green, 0; blue, 0 }  ,opacity=1 ]  {$1$};
% Text Node
\draw (269.5,191) node [scale=0.9,color={rgb, 255:red, 0; green, 0; blue, 0 }  ,opacity=1 ]  {$-1$};
% Text Node
\draw (333,130.5) node [scale=0.9]  {$1$};
% Text Node
\draw (327.5,252.5) node [scale=0.9]  {$-1$};
% Text Node
\draw (527,189) node [scale=0.9]  {$x$};
% Text Node
\draw (333,20.5) node [scale=0.9]  {$y$};
% Text Node
\draw (333,69.5) node [scale=0.9]  {$2$};


\end{tikzpicture}

    \end{center}
\end{solution}