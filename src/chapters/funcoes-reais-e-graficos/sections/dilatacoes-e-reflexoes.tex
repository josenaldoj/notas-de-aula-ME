\subsubsection{Dilatação e Reflexão}

Enquanto a translação de um gráfico é gerada pela adição de um valor a certa coordenada de todos os pontos, a dilatação e a reflexão consistem na multiplicação de um mesmo valor à coordenada $x$ ou $y$ de todos os pontos.
A dilatação modifica a forma do gráfico podendo deixá-lo mais ``achatado'' ou ``esticado'', e a reflexão espelha o gráfico.
Ambas as transformações são dadas pelos mesmos parâmetros.
Contudo, enquanto a primeira depende da magnitude (módulo) dos parâmetros, a segunda se dá pelos seus sinais.

Se uma função real $g$ é tal que $g(x) = c \cdot f(d\cdot x)$, em que $f$ é outra função real e $c,d\in \R$, então o gráfico de $g$ pode ser obtido, do gráfico de $f$, através de uma dilatação horizontal de $\modu d$ vezes e uma dilatação vertical de $\modu c$ vezes. 
Mais especificamente, os valores de $c$ e $d$ influenciam nas dilatações vertical e horizontal, respectivamente, do gráfico de $f$ da seguinte forma:
%
\begin{itemize}
          \item $\modu c>1$: esticamento vertical;
          \item $0<\modu c<1$: encolhimento vertical;
          \item $\modu d>1$: encolhimento horizontal;
          \item $0<\modu d<1$: esticamento horizontal.
\end{itemize}
%
Ademais, os valores de $c$ e $d$ influenciam nas reflexões do gráfico de $f$ em relação aos eixos $x$ e $y$, respectivamente, da seguinte forma:
%
\begin{itemize}
  \item $c > 0$: não há reflexões;
  \item $c < 0$: reflexão em relação ao eixo $x$;
  \item $d > 0$: não há reflexões;
  \item $d < 0$: reflexão em relação ao eixo $y$.
\end{itemize}

As transformações de translação e reflexão podem ocorrer ao mesmo tempo.
Assim, se $c<-1$, por exemplo, o gráfico de $g$ é obtível a partir do de $f$ por meio de um esticamento vertical composto com uma reflexão em relação ao eixo $x$.

\begin{example}
  Sejam $f, g , h: \R \to \R$ funções reais tais que $f(x) = \sen x$, $g(x) = \frac 1 2 \cdot f(x)  = \frac 1 2 \cdot \sen x $ e $h(x)= f(2 \cdot x)= \sen (2x)$.
  Compare os gráficos de $h$ e $g$ ao gráfico de $f$ com relação a dilatações.
\end{example}

\begin{solution}
Na Imagem~\ref{img:dilatacao-exemplo1-g}, são mostrados os gráficos de $f$ e $g$.
%
  \begin{figure}[H]
    \centering
    \importtikz{dilatacao-exemplo1-g}
    \caption{Gráficos das funções $f$ e $g$.}
    \label{img:dilatacao-exemplo1-g}
  \end{figure}
  %
\noindent Uma vez que $g(x) = \frac 1 2 \cdot f(x)$ e $0 < \frac 1 2 < 1$, o gráfico de $g$ é obtível encolhendo verticalmente o gráfico de $f$ pela metade.

Na Imagem~\ref{img:dilatacao-exemplo1-h}, são mostrados os gráficos de $f$ e $h$.
%
  \begin{figure}[H]
    \centering
    \importtikz{dilatacao-exemplo1-h}
    \caption{Gráficos das funções $f$ e $h$.}
    \label{img:dilatacao-exemplo1-h}
  \end{figure}
  %
\noindent Uma vez que $h(x) = f(2 \cdot x)$ e $2>0$, o gráfico de $h$ é obtível encolhendo horizontalmente o gráfico de $f$.
\end{solution}

\begin{example}
Sejam $f, g , h: \R \to \R$ funções reais tais que $f(x) = \sen x$, $g(x) = -1 \cdot f(x)  = -1 \cdot \sen x $ e $h(x)= f(-1 \cdot x)= \sen (-1 \cdot x)$.
Compare os gráficos de $h$ e $g$ ao gráfico de $f$ com relação a dilatações e reflexões.
\end{example}

\begin{solution}
Na Imagem~\ref{img:dilatacao-exemplo2-g}, são mostrados os gráficos de $f$ e $g$.
%
  \begin{figure}[H]
    \centering
    \importtikz{dilatacao-exemplo2-g}
    \caption{Gráficos das funções $f$ e $g$.}
    \label{img:dilatacao-exemplo2-g}
  \end{figure}
%
\noindent Uma vez que $g(x) = -1 \cdot f(x)$ e $-1 < 0$, o gráfico de $g$ é obtível por meio de uma reflexão do gráfico de $f$ em relação ao eixo $x$.

Na Imagem~\ref{img:dilatacao-exemplo2-h}, são mostrados os gráficos de $f$ e $h$.
%
  \begin{figure}[H]
    \centering
    \importtikz{dilatacao-exemplo2-h}
    \caption{Gráficos das funções $f$ e $h$.}
    \label{img:dilatacao-exemplo2-h}
  \end{figure}
  %
\noindent Uma vez que $h(x)= f(-1 \cdot x)$ e $-1 < 0$, o gráfico de $h$ é obtível por meio de uma reflexão do gráfico de $f$ em relação ao eixo $y$.
Note que o gráfico de $h$ é igual ao de $g$. 
O porquê disso será explicado quando estudarmos as funções trigonométricas.
\end{solution}

  

\begin{onlineact}
    \khan{https://pt.khanacademy.org/math/algebra2/manipulating-functions/shifting-functions/e/shift-functions}
    {Deslocamento de Funções}.
\end{onlineact}

\begin{onlineact}
    \khan{https://pt.khanacademy.org/math/algebra2/manipulating-functions/stretching-functions/e/shifting_and_reflecting_functions}
    {Como Transformar Funções}.
\end{onlineact}