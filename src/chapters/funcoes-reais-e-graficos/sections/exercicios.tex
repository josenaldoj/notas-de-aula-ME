\section{Exercícios}

\begin{exercise}
Sejam $f: \R \to \R $. Determine se as afirmações abaixo são
verdadeiras ou falsas, justificando suas respostas. As funções que
forem usadas como contraexemplo podem ser exibidas somente com o
esboço de seu gráfico.
\begin{enumerate}[(a)]
  \item Se $f$ é limitada superiormente, então $f$ tem pelo menos um máximo absoluto;
  \item Se $f$ é limitada superiormente, então $f$ tem pelo menos um máximo local;
  \item Se $f$ tem um máximo local, então $f$ tem um máximo absoluto;
  \item Todo máximo local de $f$ é máximo absoluto;
  \item Todo máximo absoluto de $f$ é máximo local;
  \item Se $x_0$ é o ponto de extremo local de $f$, então é ponto de
  extremo local de $f^2$, onde $(f^2)(x) = f(x) \cdot f(x)$;
  \item Se $x_0$ é o ponto de extremo local de $f^2$, então é ponto de
  extremo local de $f$.
\end{enumerate}
\end{exercise}

\begin{exercise}
Sejam $f: \R \to \R $ e $g: \R \to \R$. Determine se as
afirmações abaixo são verdadeiras ou falsas, justificando suas
respostas. As funções que forem usadas como contraexemplo podem ser
exibidas somente com o esboço de seu gráfico.
\begin{enumerate}[(a)]
  \item Se $f$ e $g$ são crescentes, então a composta $f \circ g$ é uma função crescente;
  \item Se $f$ e $g$ são crescentes, então o produto $f\cdot g$ é
  uma função crescente, onde $(f \cdot g)(x) = f(x) \cdot g(x)$;
  \item Se $f$ é crescente em $A \subset \R$ e em $B \subset \R$, então $f$ é crescente em $A \cup B \subset \R$.
\end{enumerate}
\end{exercise}

\begin{exercise}
Mostre que a função inversa de uma função crescente é também uma
função crescente. E a função inversa de uma função decrescente é
decrescente.
\end{exercise}