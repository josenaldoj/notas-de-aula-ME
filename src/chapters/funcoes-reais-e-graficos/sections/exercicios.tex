\section{Exercícios}

\begin{exercise}
  Considere a função $g: [0 ; 5] \to \R$ definida por: $$g(x) =
                                \begin{cases}
                                4x-x^2 & \ \text{ se } \ x< 3 \\
                                x-2 & \ \text{ se } \  x \geq 3 \\
                                \end{cases}.$$
Determine as soluções de:
\begin{enumerate}[a)]
  \item $g(x) = -1$;
  \item $g(x) = 0$;
  \item $g(x) = 3$;
  \item $g(x) = 4$;
  \item $g(x) < 3$;
  \item $g(x) \geq 3$.
\end{enumerate}
\end{exercise}

\begin{exercise}
  Considere a função $g: [0 ; 5] \to \R$ definida por: $$g(x) =
                                \begin{cases}
                                x^2-5x & \ \text{ se } \ x> 2 \\
                                x-2 & \ \text{ se } \  x \leq 2 \\
                                \end{cases}.$$
Determine as soluções de $g(x) = -6$.
\end{exercise}

\begin{exercise}
  Considere a função $f:[3;5]\to\reals$ tal que $f(x)=-x^2+4x-3$.
  \begin{enumerate}[a)]
    \item Mostre que $f$ é decrescente.
    \item $f$ possui máximo absoluto? Se sim, ocorre em qual ponto?
    \item $f$ possui mínimo absoluto? Se sim, ocorre em qual ponto?
  \end{enumerate}
\end{exercise}

\begin{exercise}
    Considere a função $f: \reals_- \to \reals^\ast_+$ tal que $f(x) = \dfrac{1}{1+x^2}$. Responda as seguintes perguntas apresentando as respectivas justificativas.
    \begin{enumerate}[a)]
	\item $f$ é monótona? Se sim, de que tipo? Se não, $f$ possui algum intervalo de monotonicidade?
	\item $f$ possui máximo absoluto?
	\item $f$ possui mínimo absoluto?
	\item $f$ é limitada?
    \end{enumerate}
\end{exercise}

\begin{exercise}
  Seja $k \in \R$. Definimos:
  \begin{enumerate}[i)]
    \item O \emph{intervalo fechado de $k$ a infinito}, denotado por $[k, + \infty)$, como sendo o seguinte subconjunto de $\mathbb R$: 
    $$ [k,+ \infty ) = { x \in \mathbb R ; \ k \leq x };$$
    \item O \emph{intervalo fechado de menos infinito a $k$}, denotado por $(- \infty, k ]$, como sendo o seguinte subconjunto de $\R$:
    $$ (- \infty, k ] = \{ x \in \mathbb R ; \  x \leq k \}.$$
  \end{enumerate}

  Considere a função real $f$ tal que $f(x) = -x^2 +2x +8$.
\begin{enumerate}[a)]
\item Mostre que $f$ é crescente no intervalo $( - \infty , 1]$;
\item Mostre que $f$ é decrescente no intervalo $[1, + \infty )$;
\item Use os itens anteriores para concluir que $1 \in \mathbb R$ é um ponto de máximo absoluto de $f$.
\end{enumerate}  
\end{exercise}

\begin{exercise}
Seja $f: \R \to \R $. Determine se as afirmações abaixo são
verdadeiras ou falsas, justificando suas respostas. As funções que
forem usadas como contraexemplo podem ser exibidas somente com o
esboço de seu gráfico.
\begin{enumerate}[a)]
  \item Se $f$ é limitada superiormente, então $f$ tem pelo menos um máximo absoluto;
  \item Se $f$ é limitada superiormente, então $f$ tem pelo menos um máximo local;
  \item Se $f$ tem um máximo local, então $f$ tem um máximo absoluto;
  \item Todo máximo local de $f$ é máximo absoluto;
  \item Todo máximo absoluto de $f$ é máximo local;
  \item Se $x_0$ é o ponto de extremo local de $f$, então é ponto de extremo local de $f^2$, onde $(f^2)(x) = f(x) \cdot f(x)$;
  \item Se $x_0$ é o ponto de extremo local de $f^2$, então é ponto de extremo local de $f$.
\end{enumerate}
\end{exercise}

\begin{exercise}
  Seja $f: \N \to \R $ e $g: \R \to \N$. Determine se as afirmações abaixo são
  verdadeiras ou falsas, justificando suas respostas. As funções que
  forem usadas como contraexemplo podem ser exibidas somente com o
  esboço de seu gráfico.
  \begin{enumerate}[a)]
    \item A função $g$ pode ser ilimitada inferiormente;
    \item $f$ é limitada superiormente ou $f$ é limitada inferiormente.
  \end{enumerate}
  \end{exercise}

\begin{exercise}
Sejam $f: \R \to \R $ e $g: \R \to \R$. Determine se as
afirmações abaixo são verdadeiras ou falsas, justificando suas
respostas. As funções que forem usadas como contraexemplo podem ser
exibidas somente com o esboço de seu gráfico.
\begin{enumerate}[a)]
  \item Se $f$ e $g$ são crescentes, então a composta $f \circ g$ é uma função crescente;
  \item Se $f$ e $g$ são crescentes, então o produto $f\cdot g$ é
  uma função crescente, onde $(f \cdot g)(x) = f(x) \cdot g(x)$;
  \item Se $f$ é crescente em $A \subset \R$ e em $B \subset \R$, então $f$ é crescente em $A \uniao B \subset \R$.
\end{enumerate}
\end{exercise}

\begin{exercise}
  Mostre que uma função real é constante se, e somente se, é não decrescente e não crescente.
\end{exercise}

\begin{exercise}
  Sejam $f : \reals \to \reals$ e $A$ e $B$ intervalos reais tais que $A \inter B$ é um intervalo não
  degenerado, ou seja, que possui pelo menos dois números. Mostre que, se $f$ é crescente
  em $A$ e em $B$, então $f$ é crescente em $A\inter B$.
\end{exercise}

\begin{exercise}
\label{exer:inversa-funcao-crescente}
Mostre que a função inversa de uma função crescente é também uma
função crescente, e que a função inversa de uma função decrescente é
decrescente.
\end{exercise}

\begin{exercise}
\textit{Observação}: este exercício requer conhecimentos de progressões
aritméticas, estudadas no Capítulo~\ref{ch:progressoes}.

  Sejam $(a_0,a_1, \dots,a_n,\dots)$ uma PA de razão positiva e a função 
  $a : \nats \to \reals$ tal que $a(i)=a_i$ para todo $i\in\nats$.
  Mostre que a função $a$ é crescente.
\end{exercise}

\begin{exercise}
  Dizemos que uma função $f: \R \to \R$ é \emph{par} quando se tem $f(-t) = f(t)$ para todo $t \in \R$. Se for o caso de $f(-t) = -f(t)$ para todo $t \in \R$, dizemos que $f$ é \emph{ímpar}.

  Considere a função real $f: \R \to \R$. Demonstre, ou refute com um contraexemplo, as afirmações abaixo:
  \begin{enumerate}[a)]
    \item Se $f$ é par e $x_0 \in \R$ é um ponto de máximo absoluto, então $-x_0 \in \R$ é também um ponto de máximo absoluto;
    \item Se $f$ é ímpar e $x_0 \in \R$ é um ponto de mínimo absoluto, então $-x_0 \in \R$ é um ponto de máximo absoluto.
    \item Se $f$ é ímpar e limitada superiormente, então $f$ é limitada inferiormente.
  \end{enumerate}
\end{exercise}
