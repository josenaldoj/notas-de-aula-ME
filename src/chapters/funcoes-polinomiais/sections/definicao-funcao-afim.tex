\subsubsection{Definição}

\begin{definition}
Uma função real chama-se \emph{afim} quando existem constantes $a, b
\in \R$ tais que $f(x) = ax +b$ para todo $x\in \R$. 
A constante $a$, em particular, é chamada de \emph{taxa de variação} ou \emph{taxa de crescimento} da função.
\end{definition}

\begin{remark}
É muito comum se referir ao coeficiente $a$ de uma função afim como coeficiente angular.
Esse termo não é apropriado, pois define-se coeficiente angular para
retas, e não para funções, mesmo que o gráfico de uma
função afim seja uma reta.
\end{remark}

\begin{example}
A função identidade $\identity \R$ (relembre a Definição \ref{def:funcao-identidade-conjunto}) é afim, 
assim como as funções reais com lei de formação $g(x) = \identity \R (x)+b = x+b$, 
cujos gráficos são translações verticais do de $\identity \R$.
\end{example}

\begin{example}
As funções lineares $f(x) = ax$ e as funções constantes $f(x) = b$ são casos especiais de funções afins.
\end{example}

\begin{example}
O preço a se pagar por uma corrida de táxi é dado por uma função
afim $f: x \mapsto ax+b$, em que $x$ é a distância percorrida
(usualmente medida em quilômetros), o valor inicial $b$ é a chamada
\emph{bandeirada} e o coeficiente $a$ é o preço de cada quilômetro
rodado.
\end{example}