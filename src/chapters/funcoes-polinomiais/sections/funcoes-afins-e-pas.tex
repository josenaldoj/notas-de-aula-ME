\subsubsection{Funções Afins e PAs}

\begin{proposition}
Seja  $f: \R \to \R$. Se $f$ é uma função afim e $\paren{x_1, x_2,
\dots , x_i, \dots}$ é uma PA, então a sequência formada pelos
pontos $y_i = f(x_i)$, $i \in \N^*$ é uma PA. Reciprocamente,
se $f$ for monótona e transformar qualquer PA $\paren{x_1, x_2,
\dots , x_i, \dots}$ numa PA com termo geral $y_i = f(x_i)$, $i \in
\N^*$, então $f$ é uma função afim.    
\end{proposition}

\begin{proof}
    Seja $f: \R \to \R$ uma função afim e $(x_i)_{i\in\N^*}$ uma PA de razão $r$. 
    Provemos que a sequência $(y_i)_{i\in\N^*}$, onde $y_i = f(x_i)$, para cada $i \in \N^*$, 
    é uma PA.
    Para tanto, seja $n \in \N^*$. 
    Note que:
    %
    \[
        y_{n+1}-y_n = f(x_{n+1})-f(x_n).
    \]

    Como $f$ é afim, existem $a,b\in\R$ tais que $f(x)=ax+b$.
    Logo:
    %
    \begin{align*}
        y_{n+1}-y_n &= a(x_{n+1})+\cancel b - (ax_n+\cancel b) \\ &= a(x_{n+1}-x_n) \\ &= ar.
    \end{align*}
    
    Assim, $(y_i)_{i\in\N^*}$ é uma PA.

    Suponha, reciprocamente, que $f$ é monótona e transforma qualquer
    PA $\paren{x_1, x_2, \dots , x_i, \dots}$ numa PA com termo geral $y_i = f(x_i)$, com $i \in
    \N^*$. 

    Para provar que $f$ é afim, é suficiente verificar a linearidade da função $g(x)$ tal que
    $f(x) = g(x) + f(0)$, uma vez que $g(x)$ seria da forma $ax$ e $f(0)$ é constante. 
    Pelo Teorema \ref{def:teorema-fundamental-proporcionalidade}, basta provar que 
    $g(nx) = ng(x)$ para todo $n \in \Z$ e todo $x \in \R$.

    Note que $g(x) = f(x) - f(0)$. Além disso, se a sequência $\prn{f(x_1), f(x_2), \dots, f(x_i), \dots}$
    é uma PA de razão $r$, a sequência $\prn{f(x_1) - f(0), f(x_2) - f(0), \dots, f(x_i) - f(0), \dots}$ também é 
    uma PA de razão $r$. Dessa forma, podemos concluir que $g$ também transforma uma PA em outra.

    Considere a PA $\prn{0, x, 2x, 3x, \dots}$. Temos que $\prn{g(0), g(x), g(2x), g(3x), \dots}$
    também é uma PA, com razão $g(x)-g(0)$. No entanto, $g(0) = f(0)-f(0) = 0$, o que implica que a PA tem
    razão $g(x)$. Pela fórmula do termo geral de uma PA, com $a_1 = g(0)$ e $a_{n+1} = g(nx)$,
    concluímos que $g(nx) = g(0) + n\cdot g(x) = n\cdot g(x)$
    para todo $n \in \N$.

    Falta mostrar que $g(nx) = n \cdot g(x)$ para cada $n \in \Z$ tal que $n < 0$.
    Para nos ajudar nessa tarefa, vamos demonstrar que $g(x) = -g(-x)$, em que $x \in \R$. 
    Primeiro, note que, para qualquer $x \in \R$, os números $-x$, $0$ e $x$
    constituem uma PA, e, portanto, $g(-x)$, $g(0)$ e $g(x)$ também estão em PA. Daí, temos que $g(x) - g(0) = 
    g(0)-g(-x)$. No entanto, $g(0) = f(0) - f(0) = 0$, o que implica que $g(x) = -g(-x)$.

    Seja, agora, $n \in \Z$ tal que $n < 0$; ou seja, $-n \in \N$. Calculemos:
    %
    \begin{alignat*}{3}
        g(nx) &= g\prn{(-n)(-x)} &&  \\
        &= (-n)g(-x) && \quad\text{($g(mx) = mg(x)$ para $m\in\N$)} \\
        &= (-n)\prn{-g(x)} && \quad\text{($g(k) = -g(-k)$ para $k \in \R$)} \\
        &= ng(x)
    \end{alignat*}

    Demonstramos, então, que $g(nx) = ng(x)$ para todo $n \in \Z$. Pelo \ref{def:teorema-fundamental-proporcionalidade},
    $g$ é linear; isso é, $g(x) = ax$ para algum $a \in \R$. Logo, $f(x) = ax + f(0)$. Como $a$ e $f(0)$ são constantes,
    temos que $f$ é afim.

    
 \end{proof}

\begin{onlineact}
    \khan{https://pt.khanacademy.org/math/cc-eighth-grade-math/cc-8th-linear-equations-functions/8th-linear-functions-modeling/e/constructing-and-interpreting-linear-functions}
    {Problemas de Modelos de Funções Lineares}.
\end{onlineact}

\begin{onlineact}
    \khan{https://pt.khanacademy.org/math/cc-eighth-grade-math/cc-8th-linear-equations-functions/constructing-linear-models-real-world/e/constructing-linear-functions-word-problems}
    {Problemas de Como Escrever Funções}.
\end{onlineact}

\begin{onlineact}
    \khan{https://pt.khanacademy.org/math/cc-eighth-grade-math/cc-8th-linear-equations-functions/linear-nonlinear-functions-tut/e/linear-non-linear-functions}
    {Funções Lineares e Não Lineares}.
\end{onlineact}