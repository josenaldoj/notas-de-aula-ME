\subsection{Definição}
\label{sec:definicao-funcoes-polinomiais}

\begin{definition}
Diz-se que $p: \R \to \R$ é uma \textdef{função polinomial} quando
existem números reais $a_0, a_1, \dots , a_n$ tais que, para todo $x
\in \R$, tem-se

\begin{equation}\label{funcpol}
p(x) = a_n x^n + a_{n-1} x^{n-1} + \dots + a_1 x + a_0.
\end{equation}

Os elementos de $p^{-1}(0)$ são chamados de \textdef{raízes de $p$}.
\end{definition}

\begin{example}
\label{exa:div-polinomio-linear}
Além das funções lineares, afins e quadráticas, a soma e o produto
de funções polinomiais são funções polinomiais. Considere a função
polinomial $p$ tal que $$p(x) = \paren{x - \alpha} \paren{x^{n-1} +
\alpha x^{n-2} + \dots + \alpha^{n-2}x+ \alpha^{n-1}}=x^n -
\alpha^n.$$ Nesse caso, dizemos que $p(x)$ é \textdef{divisível} por $x-
\alpha$.
\end{example}

\begin{proposition}
\label{prop:fatoracao-polinomios}
Se $\alpha \in \R$ é raiz da função polinomial $p(x)$ de grau $n$,
então existe uma função polinomial $q(x)$, de grau $n-1$, tal que
$$p(x) = \paren{x- \alpha}q(x).$$
Além disso, $p(x)$ não pode ter mais do que $n$ raízes.
\end{proposition}

\begin{proof}
    Seja $p(x)$ uma função polinomial de grau $n$, ou seja, $p(x)=a_n x^n  + \dots + a_1x + a_0$.
    Teremos, para $\alpha \in \R$:
    %
    \[
        p(x)-p(\alpha) = a_n(x^n-a^n) + \dots + a_1(x-\alpha) + \cancel{a_0}-\cancel{a_0}
    \]   
    %
    Colocando o termo $(x-\alpha)$ em evidência de cada parcela dos $(x^i - \alpha^i)$, teremos:
    %
    \begin{equation}
    \label{eq:px-minus-palpha}
        p(x)-p(\alpha) = (x-\alpha)q(x),
    \end{equation}
    %
    \noindent em que $q(x)$ é um polinômio de grau $n-1$ proveniente da fatoração de $(x^n-\alpha^n)$,
    pelo Exemplo \ref{exa:div-polinomio-linear}. 
    Agora, se $\alpha$ é uma raiz de $p(x)$, então $p(\alpha)=0$, e $\ref{eq:px-minus-palpha}$ tem a forma
    $p(x)=(x-\alpha)q(x)$. 
    Ademais, note que esse processo só pode ser repetido no máximo $n$ vezes, pois o grau de $q(x)$ é
    $n-1$.

\end{proof}

\begin{definition}
Uma função polinomial $p$ chama-se \emph{identicamente nula} quando
se tem $p(x) = 0$ para todo $x \in \R$.
\end{definition}

\begin{remark}
Note que uma função polinomial tem uma infinidade de raízes, 
já que todo número real é raiz de do polinômio correspondente. 
Isso não contradiz a Propisição \ref{prop:fatoracao-polinomios}, 
já que o grau de uma função polinomial não está definido para a função identicamente nula.
\end{remark}

\begin{proposition}[Fórmula de Interpolação de Lagrange]
Dados $n+1$ números reais distintos $x_0, x_1 , \dots , x_n$ e
fixados arbitrariamente $y_0, y_1, \dots, y_n$ reais, existe
um, e somente um, polinômio $p$ de grau menor ou igual a $n$ tal que
$$p(x_0) = y_0, \ \ p(x_1) = y_1,  \dots , \ \ p(x_n) = y_n.$$

$p(x)$ pode ser obtido pela fórmula:
$$p(x) = \sum_{i=0}^{n} \bracket{y_i \cdot \prod_{k \neq i} \paren{\frac {x-
x_k}{x_i-x_k}}}.$$
\end{proposition}