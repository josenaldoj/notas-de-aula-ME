\subsubsection{Gráfico}

\begin{proposition}
    O gráfico de uma função afim $f(x) = ax + b$ é uma reta.
\end{proposition}

\begin{proof}
    Sejam $f(x)=ax+b$ uma função afim e $P_1 = \prn{x_1, f(x_1)}$, $P_2 = \prn{x_2, f(x_2)}$
    e $P_3 = \prn{x_3, f(x_3)}$ pontos de seu gráfico tais que $x_1 < x_2$ e $x_2 < x_3$. 
    Provemos que esses pontos são colineares, ou seja, $\dist(P_1, P_3) = \dist(P_1, P_2) + \dist(P_2, P_3)$, onde $\dist (P_i, P_j)$ é a distância entre os pontos $P_i$ e $P_j$.
    Para isso, calculemos os valores de cada uma dessas três distâncias:
    %
    \begin{align*}
        \dist(P_1, P_2) &= \sqrt{\colche{ax_2+\cancel{b}-(ax_1+\cancel{b})}^2 + \prn{x_2-x_1}^2} \\
        &= \sqrt{\colche{a(x_2-x_1)}^2+(x_2-x_1)^2} \\
        &= \sqrt{(a^2+1)(x_2-x_1)^2}\\
        &= (x_2-x_1)\sqrt{a^2+1}
    \end{align*}
    %
    \begin{align*}
        \dist(P_2, P_3)=(x_3-x_2)\sqrt{a^2+1}
    \end{align*}
    %
    \begin{align*}
        \dist(P_1,P_3)=(x_3-x_1)\sqrt{a^2+1}
    \end{align*}
    
    Temos, de fato, que $\dist(P_1, P_2) + \dist(P_2, P_3) = (x_2-x_1)\sqrt{a^2+1}+(x_3-x_2)\sqrt{a^2+1} = (x_3-x_1)\sqrt{a^2+1} 
    = \dist(P_1,P_3)$. Portanto, o gráfico da função afim é uma reta.
\end{proof}

\begin{remark}
Para desenhar o gráfico de uma função afim, basta conhecer dois pontos, 
pois uma reta é inteiramente determinada por dois pontos.
\end{remark}