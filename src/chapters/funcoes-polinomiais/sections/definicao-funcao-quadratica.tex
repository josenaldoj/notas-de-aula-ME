\subsubsection{Definição}

\begin{definition}
Uma função $f: \R \to \R$ chama-se \emph{quadrática} quando existem
números reais $a, b, c$, com $a \neq 0$, tais que $f(x) = ax^2 +bx
+c$ para todo $x \in \R$.
\end{definition}

\begin{proposition}
Seja $f$ uma função quadrática da forma $f(x) = ax^2 + bx +c$, onde
$a >0$. Mostre que $f$ não é limitada superiormente e que o ponto
$\paren{-\frac {b} {2a}, -\frac {\Delta} {4a}}$ é o mínimo absoluto
da função. Caso tenhamos $a<0$, então $f$ não é limitada inferiormente e  o
ponto $\paren{-\frac {b} {2a}, -\frac {\Delta} {4a}}$ é o máximo
absoluto da função.
\end{proposition}

\begin{proposition}
Seja $f$ uma função quadrática da forma $f(x) = ax^2 + bx +c$. Se
$f(x_1) = f(x_2)$ para $x_1 \neq x_2$, então $x_1$ e $x_2$ são
equidistantes de $-\frac{b} {2a}$, ou seja, $\frac{x_1 +x_2} 2 =
-\frac{b}{2a}$.
\end{proposition}