\subsubsection{Definição}

\begin{definition}
Uma função $f: \R \to \R$ chama-se \emph{quadrática} quando existem
números reais $a, b, c$, com $a \neq 0$, tais que $f(x) = ax^2 +bx
+c$ para todo $x \in \R$.
\end{definition}

\begin{proposition}
Seja $f$ uma função quadrática da forma $f(x) = ax^2 + bx +c$, em que
$a >0$. Mostre que $f$ não é limitada superiormente e que o ponto
$\paren{-\frac {b} {2a}, -\frac {\Delta} {4a}}$ é o mínimo absoluto
da função. Caso tenhamos $a<0$, então $f$ não é limitada inferiormente e  o
ponto $\paren{-\frac {b} {2a}, -\frac {\Delta} {4a}}$ é o máximo
absoluto da função.
\end{proposition}

\begin{proof}
    Note que:
    %
    \begin{align*}
        f(x) &= ax^2+bx+c\\
        &= a\paren{x^2 + \frac b a x + \frac c a} \\
        &= a\colche{x^2+2x\frac b {2a} + \prn{\frac b {2a}}^2  - \prn{\frac b {2a}}^2 + \frac c a} \\
        &= a\colche{\prn{x+\frac{b}{2a}}^2 - \frac{b^2}{4a^2}+\frac c a}\\
        &= a\colche{\prn{x+\frac{b}{2a}}^2 + \frac{-b^2+4ac}{4a^2}} \\
        &= a\colche{\prn{x+\frac{b}{2a}}^2 - \frac{\Delta}{4a^2}} \\
    \end{align*}
    %
    \noindent A última expressão é chamada de \emph{forma canônica} da função quadrática.

    Note que o termo $-\frac{\Delta}{4a^2}$ não depende de $x$. 
    Assim, se $a>0$, então $f(x)$ atingirá o seu mínimo quando $\paren{x+\frac b {2a}}^2=0$, isto é, quando
    $x = -\frac b {2a}$. 
    Além disso, segue que $f(x)$ é ilimitada superiormente pois $\paren{x+\frac{b}{2a}}^2$ também o é.

    Teremos o mínimo da função como sendo:
    %
    \begin{align*}
        f\prn{\frac{-b}{2a}} &= a\prn{0^2-\frac{\Delta}{4a^2}} \\ &= -\frac{\Delta}{4a}.
    \end{align*}

    O caso em que $a<0$ é análogo.
\end{proof}

\begin{proposition}
Seja $f$ uma função quadrática da forma $f(x) = ax^2 + bx +c$. Se
$f(x_1) = f(x_2)$ para $x_1 \neq x_2$, então $x_1$ e $x_2$ são
equidistantes de $-\frac{b} {2a}$, ou seja, $\frac{x_1 +x_2} 2 =
-\frac{b}{2a}$.
\end{proposition}

\begin{proof}
    Sejam $x_1 \ne x_2$ tais que $f(x_1) = f(x_2)$. Assim:
    %
    \begin{align*}
        a\colche{ \prn{x_1+\frac{b}{2a}}^2 - \frac{\Delta}{4a^2}  } = a\colche{ \prn{x_2+\frac{b}{2a}}^2 - \frac{\Delta}{4a^2}  }
        &\iff \prn{x_1+\frac b {2a}}^2 = \prn{x_2+\frac b {2a}}^2 \\ 
        &\iff \modu{x_1 + \frac b {2a}} = \modu{x_2 + \frac b {2a}} \\ 
        &\stackrel{x_1 \ne x_2}{\iff} x_1 + \frac b {2a} = -\prn{x_2 + \frac b {2a}}  \\
        &\iff x_1 + x_2 = - 2 \cdot \frac b {2a} \\
        &\iff \frac{x_1+x_2} 2 = -\frac b {2a}
    \end{align*}

    Portanto, $x_1$ e $x_2$ são equidistantes de $-\frac b {2a}$.
\end{proof}