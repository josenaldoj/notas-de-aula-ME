\subsubsection{Definição}

\begin{definition}
    Chamamos de \textdef{função linear} uma função real com lei de formação
$f(x) = ax$ para algum $a \in \R$. Quando $a = 0$, dizemos que $f$ é \textdef{identicamente nula}.
\end{definition}

\begin{remark}
Note que, sabendo que uma função é linear, o valor de $a$ é igual a
$f(1)$.
\end{remark}

No caso das grandezas inversamente proporcionais, a função
matemática que modela tal problema é uma função $f: \R^* \to
\R^*$ tal que $f(x) = \frac a x$. 
Nesse caso, também temos a
particularidade de que $f(1) = a \in \R^*$.

A seguir, veremos um teorema de grande relevância para funções lineares.

\begin{theorem}[Teorema Fundamental da Proporcionalidade]
\label{def:teorema-fundamental-proporcionalidade}
Seja $f: \R \to \R$ uma função crescente tal que $f(1)>0$. 
As seguintes afirmações são equivalentes:
%
\begin{enumerate}[(i)]
  \item $f$ é linear;
  \item $f(x+y) = f(x) + f(y)$ para quaisquer $x, y \in \R$;
  \item $f(nx) = nf(x)$ para todo $n \in \Z$ e todo $x \in \R$.
\end{enumerate}
\end{theorem}

\begin{proof}
  Seja $f:\R \to \R$ uma função crescente.
  Para provar que os três itens são equivalentes, basta provar as implicações (i)$\implies$(ii),  
  (ii)$\implies$(iii) e (i)$\implies$(iii).

  \begin{itemize}
    \item (iii)$\implies$(i): A demonstração deste fato está fora do escopo deste texto.
    \item (i)$\implies$(ii): Suponha que $f(x)=ax$ para algum $a \in \R$. Sejam $x,y\in\R$.
    Teremos:
    %
    \begin{alignat*}{3}
        f(x+y) &= a(x+y) && \quad\text{(Definição de $f$)}\\
        &= ax+ay && \quad\text{(i)}\\
        &= f(x)+f(y) && \quad\text{(Definição de $f$)}
    \end{alignat*}

    \item (ii)$\implies$(iii): Suponha que $f(x+y)=f(x)+f(y)$ para todos $x,y\in\R$.
    Provemos, inicialmente, que $f(-x) = -f(x)$ sempre que $x\in\R$. 
    Para isso, tomemos $x\in\R$.
    Note que:
    %
    \begin{alignat*}{3}
      f(x) &= f(-x+2x)\\
      &= f(-x)+f(x+x) && \quad\text{(ii)}\\
      &= f(-x)+f(x)+f(x) && \quad\text{(ii)}\\
    \end{alignat*}
    %
    A igualdade $f(x) = f(-x)+f(x)+f(x)$ implica que $f(-x) = -f(x)$.

    Seja, agora, $n\in\Z$. Separaremos a prova de $f(nx)=nf(x)$ em casos.
    %
    \begin{itemize}
      \item Caso $n>0$: Temos que:
      %
      \begin{alignat*}{3}
        f(nx) &= f(\underbrace{x+x+\dots+x}_{\text{$n$ vezes}}) \\
        &= \underbrace{f(x)+f(x)+\dots+f(x)}_{\text{$n$ vezes}} && \quad\text{(ii)}\\
        &= nf(x)
      \end{alignat*}

      \item Caso $n<0$: Temos que:
      %
      \begin{alignat*}{3}
        f(nx)&=f\prn{-(-nx)} \\
        &= -f\prn{-nx} && \quad\text{($f(-x) = -f(x)$ para todo $x \in \R$)}\\
        &= -f\prn{(-n)x} \\
        &= -\prn{ (-n)f(x)} && \quad\text{(Caso anterior, pois $-n>0$)}\\
        &= nf(x)
      \end{alignat*}
      
      \item Caso $n=0$: Temos que:
      %
      \begin{alignat*}{3}
        f(0x) &= f(x-x) \\
        &= f(x)+f(-x) && \quad\text{(ii)}\\
        &= f(x)-f(x) && \quad\text{($f(-x) = -f(x)$ para todo $x \in \R$)}\\
        &= 0\\
        &= 0f(x)
      \end{alignat*}
    \end{itemize}

    Da verificação dos três casos, concluímos que, para quaisquer $n\in\Z$ e $x\in\R$, vale que $f(nx)=nf(x)$.
  \end{itemize}
\end{proof}

\begin{remark}
    Um resultado análogo ao do Teorema \ref{def:teorema-fundamental-proporcionalidade} é válido
    no caso de $f$ ser crescente e tal que $f(1) < 0$.
\end{remark}

A importância do Teorema \ref{def:teorema-fundamental-proporcionalidade} 
está no fato de que, se quisermos saber se $f: \R \to \R$ é uma função linear não identicamente nula,
basta verificar duas coisas:
%
\begin{enumerate}[1.]
  \item $f$ é crescente ou decrescente;
  \item $f(nx) = n f(x)$ para todo $x \in \R$ e todo $n \in \Z$. No
  caso de o domínio e o contradomínio de $f$ serem o conjunto $\R^+$, basta verificar a condição
  para $n \in \N$.
\end{enumerate}

\begin{example}
O lado de um quadrado é proporcional à sua área? Em outras palavras,
essas duas grandezas podem ser relacionadas por meio de uma função
linear?
\end{example}

\begin{solution}
  O lado de um quadrado não é proporcional à sua área.
  Para chegar a um absurdo, 
  considere que o lado e a área de um quadrado são diretamente proporcionais, ou seja, eles podem
  ser relacionados pela função linear
  %
  \begin{equation*}
  \begin{array}{cccc}
    A : & \R_+ & \to     & \R_+ \\
        &  l & \mapsto & A(l),
  \end{array}
  \end{equation*}
  %
  que mapeia um número $l$ à área $A(l)$ de um quadrado cujo lado mede $l$.
  Como $A$ é linear, então

  $$A(3) = A(1+2)=A(1)+A(2),$$

  \noindent pelo Teorema \ref{def:teorema-fundamental-proporcionalidade}. 
  No entanto, $A(3) = 9$, e $A(1)+A(2) = 1+4 = 5$, levando-nos a concluir que $9=5$, o que, evidentemente,
  é falso. Portanto, o lado e a área de um quadrado não são diretamente proporcionais.
\end{solution}