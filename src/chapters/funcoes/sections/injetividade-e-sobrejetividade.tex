\section{Injetividade e Sobrejetividade}

\begin{definition}
\label{def:inj-sobr-bij}
    Considere uma função $f: X \to Y$. Definem-se:
    %
    \begin{enumerate}[(i)]
    \item $f$ é \emph{sobrejetiva} se, para todo $y \in Y$, existe $x
    \in X$ tal que $f(x) = y$;
    \item $f$ é \emph{injetiva} se, para todos $x_1, x_2 \in X$, $x_1 \ne x_2$ implica que $f(x_1) \ne f(x_2)$;
    \item $f$ é \emph{bijetiva} se é sobrejetiva e injetiva.
    \end{enumerate}
\end{definition}

Nas Proposições \ref{prop:def-alternativa-sobr}, \ref{prop:def-alternativa-inj-1}, \ref{prop:def-alternativa-inj-2} e \ref{prop:def-alternativa-bij}, são mostradas caracterizações alternativas dos conceitos introduzidos na Definição \ref{def:inj-sobr-bij}.
Em todas elas, considere $f$ uma função com domínio $X$ e contradomínio $Y$.
%
\begin{proposition}
\label{prop:def-alternativa-sobr}
    $f$ é sobrejetiva se, e somente se, $f(X) = Y$.
\end{proposition}

\begin{proposition}
\label{prop:def-alternativa-inj-1}
    $f$ é injetiva se, e somente se, para todos $x_1, x_2 \in X$, $f(x_1) = f(x_2)$ implica $x_1 = x_2$.
\end{proposition}

\begin{proposition}
\label{prop:def-alternativa-inj-2}
    $f$ é injetiva se, e somente se, para todo $y \in f(X)$, existe um único $x \in X$ tal que $f(x) = y$.
\end{proposition}

\begin{proposition}
\label{prop:def-alternativa-bij}
    $f$ é bijetiva se, e somente se, para todo $y \in Y$, existe um único $x \in X$ tal que $f(x) = y$.
\end{proposition}

\begin{example}
\label{ex:p-q-inj-sobr-bij}
    Quais das propriedades de injetividade, sobrejetividade e bijetividade as funções $p$ e $q$ possuem?
\end{example}

\begin{solution}
    Verifiquemos, primeiro, se as funções são injetivas. $p$ não é, pois $p(-1)=1=p(1)$.
    Já $q$, por outro lado, é injetiva.
    Para provar isso, sejam $x_1, x_2 \in \R_+$ tais que $q(x_1)=q(x_2)$.
    Ora, 
    %
    \begin{align*}
        q(x_1) = q(x_2) & \implies \sqrt{x_1} = \sqrt{x_2} \\ & \implies \prn{\sqrt{x_1}}^2 = \prn{\sqrt{x_2}}^2 \\ & \implies x_1 = x_2,
    \end{align*}
    %
    o que nos permite concluir que $q$ é injetiva.

    Agora, vamos à sobrejetividade.
    $q$ não é sobrejetiva, pois não existe $x \in R_+$ tal que $\sqrt x = -1$, e $-1$ é um elemento do contradomínio dessa função.
    Já $p$, por sua vez, é sobrejetiva. 
    De fato; para todo $y \in \R_+$, $p\prn{\sqrt y} = \prn{\sqrt y}^2 = y$, e $\sqrt y$ está no domínio de $p$.

    A bijetividade não é válida para nenhuma das funções, já que $p$ não é injetiva e $p$ não é sobrejetiva.
\end{solution}

\begin{tve}
	\link{https://drive.google.com/file/d/1QqrQ5Rq2jpYwvYU8x3Q8IrCo2IxX-QeP/view?usp=sharing}{Função injetiva, sobrejetiva, bijetiva}.
\end{tve}

\begin{theorem}
\label{theo:inv-sse-bij}
    Uma função $f: X \to Y$ é invertível se, e somente se, é bijetiva.
\end{theorem}

\begin{proof}
    Seja  $f: X \to Y$.
    \begin{itemize}
        \item (\emph{Somente se}) Suponha que $f$ é invertível; isto é, existe $g : Y \to X$ tal que $f \fcomp g = \identity{Y}$ e $g \fcomp f = \identity{X}$.
        Provemos que $f$ é bijetiva. 
        \begin{itemize}
            \item (\emph{Sobrejetividade}) Seja $b \in Y$.
            Tome $g(b)\in X$.
            Note que:
            %
            \begin{align*}
                f\prn{g(b)} &= \prn{f \fcomp g}(b) \\ &= \identity{Y}(b) \\ &= b.
            \end{align*}
            %
            Assim, $f$ é sobrejetiva.
            \item (\emph{Injetividade}) Sejam $a_1, a_2 \in X$ tais que $f(a_1)=f(a_2)$.
            Logo, $g(f(a_1))=g(f(a_2))$.
            Note que:
            %
            \begin{align*}
                g(f(a_1))=g(f(a_2)) & \implies \prn{g \fcomp f}(a_1) = \prn{g \fcomp f}(a_2) \\
                & \implies \identity{X}(a_1) = \identity{X}(a_2) \\
                & \implies a_1 = a_2
            \end{align*}
            %
            Logo, $f$ é injetiva.
        \end{itemize}
        %
        Concluímos, então, que $f$ é bijetiva.

        \item (\emph{Se}) Suponha que $f$ é bijetiva.
        Ademais, defina $g: Y \to X$ com lei de formação $g(b)=a$, em que $a \in X$ é tal que $f(a)=b$.
        Ora, para cada $b \in Y$, existe tal $a$, pela sobrejetividade de $f$.
        Além disso, como $f$ é injetiva, $a$ é único. 
        Assim, pela definição alternativa de funções, $g$ é uma função.

        A igualdade $f \fcomp g = \identity Y$ é válida. De fato, além de possuírem os mesmos domínios e contradomínios, segue que para todo $y \in Y$,
        %
        \begin{align*}
            \prn{f \fcomp g}(y) & = f\prn{g(y)} & \text{(Definição de composição de funções)} \\ 
             &= y. & \text{(Definição de $g$, com $a=g(y)$ e $b=y$)}
        \end{align*}
        Logo, $f \fcomp g$ e $ \identity Y$ possuem a mesma lei de associação. Segue que  $f \fcomp g = \identity Y$.
        
        Verifiquemos, agora, que $g \fcomp f = \identity X$.
        Para tal, observe que ambas as funções possuem os mesmos domínios e contradomínios, e seja $x \in X$.
        % Suponha, para chegar a um absurdo, que $\prn{g \fcomp f}(a) \ne a$.
        % Assim,
        % %
        % \begin{equation*}
        %     \begin{array}{rlc}
        %         \prn{g \fcomp f}(a) \ne a & \implies f\prn{\prn{g \fcomp f}(a)} \ne f(a) & (\text{Injetividade de $f$}) \\
        %         & \implies \prn{f \fcomp g \fcomp f}(a) \ne f(a) & \\
        %         & \implies \prn{f \fcomp g}\prn{f(a)} \ne f(a) & \\
        %         & \implies \identity{Y}\prn{f(a)} \ne f(a) & \\
        %         & \implies f(a) \ne f(a) &
        %     \end{array}
        % \end{equation*}
        % %
        % Chegamos a um absurdo.
        % Consequentemente, $\prn{f \fcomp g}(a) = a$, o que implica que $g \fcomp f = \identity X$ e nos permite concluir que $f$ é invertível.
        Temos que:
        %
        \begin{align*}
            \prn{g \fcomp f}(x) & = g\prn{f(x)} & \text{(Definição de composição de funções)} \\ 
             &= x. & \text{(Definição de $g$, com $a=x$ e $b=f(x)$)}
        \end{align*}
        O que nos permite concluir que $g \fcomp f$ e $\identity X$ têm leis de associação iguais.
        %
        Assim, $g \fcomp f = \identity X$.

        Disso, e do fato que $f \fcomp g = \identity Y$, concluímos que $f$ é invertível.
    \end{itemize}
\end{proof}

\begin{example}
    Decorre do Teorema \ref{theo:inv-sse-bij} e do Exemplo \ref{ex:p-q-inj-sobr-bij} que as funções $p$ e $q$ não são invertíveis.
\end{example}