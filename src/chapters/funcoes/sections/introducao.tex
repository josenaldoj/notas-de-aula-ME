\section{Introdução}

De modo geral, uma \textdef{função} é uma ferramenta matemática utilizada para relacionar dois conjuntos quaisquer. 
A relação consiste em associar, a cada elemento de um dos conjuntos, \emph{exatamente um} elemento do outro conjunto.
Essa associação pode ser chamada, também, de \emph{mapeamento}.

Considere dois conjuntos $X$ e $Y$ tais que existem $x \in X$ e $y \in Y$. Sendo assim, podemos associar o objeto $x$ ao objeto $y$ através de uma função $f$. Nessa situação, também se fala que obtemos $y$ ao \emph{aplicarmos} a função $f$ em $x$. Expressamos essa afirmação através da notação $f(x) = y$. A associação pode ser representada pelo diagrama a seguir:
%
\begin{center}
     \importtikz{funcao-notacao}
\end{center}
%
No diagrama, os conjuntos $X$ e $Y$ são representados por elipses; os elementos $x$ e $y$, por pontos dentro dessas elipses, e o mapeamento $f(x) = y$, através da flecha. Note que sempre será o caso que $f(x) \in Y$, e, dessa forma, a ponta da flecha estará num ponto do conjunto $Y$.