\section{Fórmulas e Funções}

É muito importante não pensar que uma função é uma fórmula.
Considere as funções
%
$$\begin{array}{cccc}
p_1 : & \R & \to     & \R \\
     &  x & \mapsto & x^2
\end{array}
\text{\ \ \  e \ \ \ }
\begin{array}{cccc}
p_2 : & \R_+ & \to     & \R_+ \\
     &  x & \mapsto &  x^2
\end{array}.$$
%
Apesar de possuírem a mesma lei de formação -- nesse caso, uma fórmula --, elas não são iguais.
Note que os seus domínios são diferentes, e o mesmo vale para seus contradomínios.
Essas diferenças impactam, até, as propriedades que as funções satisfazem: $p_2$ é bijetiva, mas $p_1$ não.

Outra situação que refuta a ideia de que uma função é uma fórmula é quando se tem uma função que pode ser definida usando mais de uma fórmula.
Como exemplo, pode-se tomar a função $h:\R \to \R$ com a seguinte lei de formação:
%
     $$h(x) =  \begin{cases}
            0, &  \ \text{ se } x \in \R \setminus \Q \\
            1, & \ \text{ se } x \in \Q
            \end{cases} .$$