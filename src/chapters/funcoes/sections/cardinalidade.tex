\section{Funções e Cardinalidade}

As funções servem, também, como uma ferramenta para comparar conjuntos com relação às suas cardinalidades.
Na Definição \ref{def:cardinalmente-equivalente}, é mostrada a forma como isto é feito.

\begin{definition}
\label{def:cardinalmente-equivalente}
Dois conjuntos $X$ e $Y$ são ditos \emph{cardinalmente equivalentes}
(ou \emph{equipotentes}) se existe uma bijeção $f : X \to Y$.
\end{definition}

Apesar de, à primeira vista, poder parecer que bijeções e cardinalidades de conjuntos são conceitos desconexos, a relação entre eles já existia no tempo das cavernas.
Mesmo quando não existiam sistemas de representação de números, os hominídeos pré-históricos precisavam saber, por exemplo, se algum de seus animais estava faltando. 
Para tal, eles mantinham um conjunto de objetos, como pedras, com a mesma quantidade de elementos que o conjunto dos seus animais.
Assim, quando queriam saber se não havia animais faltando -- isto é, se os dois conjuntos em questão tinham a mesma cardinalidade --, bastava fazer uma associação entre os animais e os objetos de forma que cada animal correspondesse a um único objeto, e vice-versa.
Tal associação é, justamente, uma função bijetiva entre os dois conjuntos.

\begin{definition}
Dizemos que um conjunto é \emph{enumerável} quando ele é cardinalmente equivalente a algum subconjunto de $\N$.
\end{definition}

\begin{theorem}
Se existe uma injeção $f: X \to Y$, então existe uma bijeção entre $X$ e um subconjunto $Y' \subset Y$; isto é, $X$ é cardinalmente
equivalente a um subconjunto de $Y$.
\end{theorem}

\begin{theorem}
Se existe uma sobrejeção $f : X \to Y$, então existe uma bijeção entre $Y$ e um subconjunto $X' \subset X$, isto é, $Y$ é
cardinalmente equivalente a um subconjunto de $X$.
\end{theorem}

\begin{example}
	O conjunto $\Q$ é enumerável.
\end{example}

\begin{onlineact}
	\khan{https://pt.khanacademy.org/math/algebra2/manipulating-functions/invertible-functions/e/inverse-domain-range}{Determine se uma Função É Inversível}.
\end{onlineact}

\begin{onlineact}
	\khan{https://pt.khanacademy.org/math/algebra2/manipulating-functions/invertible-functions/e/restrict-the-domains-of-functions}{Restrinja os Domínios de Funções para Torná-las Inversíveis}.
\end{onlineact}