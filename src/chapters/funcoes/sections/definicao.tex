\section{Definição de Função}

\begin{definition}
Sejam $X$ e $Y$ dois conjuntos quaisquer.
Uma \textdef{função} é uma relação $f: X \to Y$ que, a cada elemento $x \in X$, associa um e somente um elemento $y \in Y$.
Definem-se, além disso, os seguintes termos:
%
\begin{enumerate}[(i)]
  \item \textdef{Domínio} e \textdef{contradomínio} de $f$ para os conjuntos $X$ e $Y$, respectivamente;
  \item \textdef{Imagem} de $f$ para o conjunto $f\prn X = \set{y \in Y \tq \exists x \in X , f \prn x =
  y} \subset Y$;
  \item \textdef{Imagem} de $x$, dado $x \in X$, para o (único) elemento $y = f(x) \in Y$.
\end{enumerate}
\end{definition}

\begin{remark}
\label{rem:def-alternativa-funcao}
Uma função pode ser vista como um terno constituído por: \textdef{domínio}, \textdef{contradomínio} e \textdef{lei de associação} (dos elementos do
domínio com os do contradomínio). 
Precisa-se desses três elementos para que uma função seja bem-definida. 
A definição alternativa a seguir também poderia ser adotada:

{\it Uma relação $f: X \to Y$ é uma \emph {função} se satisfaz as seguintes condições:
%
\begin{enumerate}[(I)]
  \item Estar bem-definida em todo elemento do domínio (existência);
  \item Não fazer corresponder mais de um elemento do contradomínio
  a cada elemento do domínio (unicidade).
\end{enumerate}}
\end{remark}

\begin{example}
\label{example:func-sq-sqrt}
Considere as funções $p$ e $q$ a seguir:
%
\begin{gather*}
\begin{array}{cccc}
p : & \R & \to     & \R_+ \\
     &  x & \mapsto & x^2;
\end{array}\\
\begin{array}{cccc}
q : & \R_+ & \to     & \R \\
     &  x & \mapsto & \sqrt x
\end{array}.
\end{gather*}
%
Qual o domínio, contradomínio e a lei de associação de $p$ e $q$?
\end{example}

\begin{solution}
\begin{itemize}
	\item[]
	\vspace{3mm}
	\item O domínio de $p$ é $\R$ e o de $q$ é $\R_+$;
	\item O contradomínio de $p$ é $\R_+$ e o de $q$ é $\R$;
	\item A lei de associação de $p$ é $p(x)=x^2$ e a de $q$ é $q(x)=\sqrt x$.
\end{itemize}
\end{solution}

\begin{definition}
Seja $\identity X : X \to X $ uma função tal que $\identity X \prn x = x$ para todo $x \in X$. Chamamos $\identity X$ de \textdef{função identidade do conjunto $X$}.
\end{definition}