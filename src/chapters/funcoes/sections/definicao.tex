\section{Definição de Função}

\begin{definition}
Sejam $X$ e $Y$ dois conjuntos quaisquer.
Uma \textdef{função} é uma relação $f: X \to Y$ (lê-se $f$ de $X$ em $Y$) que, a cada elemento $x \in X$, associa um, e somente um, elemento $y \in Y$.
Definem-se, além disso, os seguintes termos:
%
\begin{enumerate}[(i)]
\item \textdef{Domínio} e \textdef{contradomínio} de $f$ para os conjuntos $X$ e $Y$, respectivamente;
\item \textdef{Imagem} de $x$, dado qualquer $x \in X$, para o único elemento $y \in Y$ que satisfaz $y = f(x)$;
\item \textdef{Imagem} de $f$ para o conjunto $f\prn X = \set{y \in Y \tq y = f \prn x \text{ para algum } x \in X }$, ou seja, o conjunto das imagens dos elementos de $X$.

Confira o exemplo a seguir:
\begin{center}
    \importtikz{funcao-exemplo-imagem}
\end{center}
Observe que o conjunto imagem de $f$, por ser constituído de elementos de $Y$, sempre está contido no contradomínio, isto é, $f(X) \subset Y$.
\end{enumerate}
\end{definition}

%\begin{remark}
%\label{rem:def-alternativa-funcao}
De certa forma, uma função pode ser vista como um terno constituído por: \textdef{domínio}, \textdef{contradomínio} e \textdef{lei de associação} (dos elementos do domínio com os do contradomínio). 
Precisa-se desses três elementos para que uma função seja bem-definida. 

A definição alternativa a seguir também poderia ser adotada:
{\it \label{def:funcao-alternativa} Uma relação $f: X \to Y$ é uma \emph {função} se satisfaz as seguintes condições:
%
\begin{enumerate}[(I)]
  \item Estar bem-definida em todo elemento do domínio, isto é, os elementos do domínio devem corresponder à pelo menos um elemento do contradomínio;
  \item Os elementos do domínio não devem corresponder a mais de um elemento do contradomínio, ou seja, devem corresponder no máximo um elemento.
\end{enumerate}}
%\end{remark}


\begin{example}
Sejam $X = \set{x_1, x_2}$, $Y = \set{y_1, y_2}$ e a relação $f : X \to Y$ definida pelo gráfico a seguir:
\begin{center}
     \importtikz{funcao-nao-exemplo}
\end{center}
Qual(is) o(s) problema(s) com essa ``função''?
\end{example}
\begin{solution}
Temos dois problemas, o primeiro é que, pelo gráfico, temos que $f(x_1) = y_1$ e $f(x_1) = y_2$, e com isso existem dois valores possíveis para $f(x_1)$. O segundo é que $x_2$ não foi mapeado em nenhum elemento de $Y$, sendo assim não há um valor para $f(x_2)$.
\end{solution}


\begin{example}
\label{example:func-sq-sqrt}
Considere as funções $p$ e $q$ a seguir:
%
\begin{gather*}
\begin{array}{cccc}
p : & \R & \to     & \R_+ \\
     &  x & \mapsto & x^2;
\end{array}\\
\begin{array}{cccc}
q : & \R_+ & \to     & \R \\
     &  x & \mapsto & \sqrt x
\end{array}.
\end{gather*}
%
Qual o domínio, contradomínio e a lei de associação de $p$ e $q$?
\end{example}

\begin{solution}
O domínio de $p$ é $\R$ e o de $q$ é $\R_+$; 
O contradomínio de $p$ é $\R_+$ e o de $q$ é $\R$; 
A lei de associação de $p$ é $p(x)=x^2$ e a de $q$ é $q(x)=\sqrt x$.
\end{solution}

\begin{definition}[Função Identidade de um Conjunto]
Seja $\identity X : X \to X $ uma função tal que $\identity X \prn x = x$ para todo $x \in X$. Chamamos $\identity X$ de \textdef{função identidade do conjunto $X$}.
\end{definition}