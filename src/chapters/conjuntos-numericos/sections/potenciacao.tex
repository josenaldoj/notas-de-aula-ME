\section{Potenciação}

\begin{definition}
A \textdef{potência} $n \in \N^*$ de um número real $a$ é definida
como sendo a multiplicação de $a$ por ele mesmo $n$ vezes, ou seja:
%
\begin{align*}
a^n = \underbrace{a \cdot a  \dots  a}_{n \text{
vezes}}.
\end{align*}
\end{definition}

\begin{definition}
Quando $a \neq 0$, $a^0 = 1$. $0^0$ é uma indeterminação. Além disso, para $n \in \R^*_+$, tem-se que: 
\begin{align*}
&a^{-n} = \frac{1}{a^n}; \\
&a^{1/n} = \sqrt[n] a.
\end{align*}
\end{definition}

É comum definir $0^0 =1$ dependendo de como se quer abordar as potências. Saiba mais \link{https://pt.wikipedia.org/wiki/Zero_elevado_a_zero}{aqui}.

\begin{proposition}[Propriedades]
Sejam $a, b, n, m \in \R$, a menos que se diga o contrário.
\begin{enumerate}[i.]
  \item $a^m \cdot a^n = a^{m+n}$, com $a \ne 0$ ou $m \ne -n$;
  \item $\frac {a^m}{a^n} = a^{m-n}$, com $a \ne 0$;
  \item $\prn{a^m}^n = a^{m\cdot n}$, com $a \ne 0$ ou $m\cdot n \ne 0$;
  \item $a^{m^n} = a^{\overbrace{m \cdot m  \dots  m}^{n \text{
  vezes}}}$, com $n \in \N^*$, e $a \ne 0$ ou $m \ne 0$;
  \item $\prn{a \cdot b }^n= a^n \cdot b^n$, com $a \ne 0$ ou $a \cdot b \ne 0$;
  \item $\prn{\frac a b }^n = \frac {a^n} {b^n}$, com $b \ne 0$ e $n \ne 0$; 
  \item $a^{\frac m n} = \sqrt[n]{a^m}$, com $n \neq 0$, e $a \ne 0$ ou $m \ne 0$.
\end{enumerate}
\end{proposition}

\begin{remark}
Seja $a \in \R$. Temos que $\sqrt{a^2} = \modu a$. Mais geralmente, $\sqrt[n] {a^n} = \modu a$ para $n$ par. Além disso, é errado dizer que $\sqrt 4 = \pm 2$. O correto é $\sqrt 4 = 2$, mesmo que escrevas $\sqrt 4 = \sqrt{\prn {-2}^2}$. 

O erro apresentado é comum, e o fator de confusão é que responder o conjunto-solução da equação $x^2=4$ não é equivalente a responder qual a raiz de $4$, e sim responder quais números que elevados ao quadrado são iguais a $4$.
\end{remark}

\begin{onlineact}
  \khan{https://pt.khanacademy.org/math/algebra/rational-exponents-and-radicals/rational-exponents-and-the-properties-of-exponents/e/exponents_4}
  {Propriedades da potenciação (expoentes racionais)}.
\end{onlineact}

\begin{onlineact}
  \khan{https://pt.khanacademy.org/math/algebra/rational-exponents-and-radicals/alg1-simplify-square-roots/e/multiplying_radicals}
  {Simplifique raízes quadradas (variáveis)}.
\end{onlineact}

\begin{onlineact}
  \khan{https://pt.khanacademy.org/math/algebra/rational-exponents-and-radicals/alg1-simplify-square-roots/e/adding_and_subtracting_radicals}
  {Simplifique expressões de raiz quadrada}.
\end{onlineact}
