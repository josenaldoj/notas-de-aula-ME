\subsection{Gráfico}

\begin{example}
    Considere as funções logarítmicas tais que $f(x) = \log_2 x$ e $g(x)
= \log_{\frac 1 2} x$. Os gráficos de $f$ e $g$ são apresentados
abaixo.
\end{example}

\begin{remark}
    Já vimos que o crescimento exponencial supera o de qualquer
polinômio. Por ser a inversa da função exponencial, a função
logarítmica possui um crescimento muito lento. Mesmo assim, a função
logarítmica é ilimitada superiormente. Compare os gráficos abaixo:
\end{remark}

\begin{onlineact}
    \khan{https://pt.khanacademy.org/math/algebra2/exponential-and-logarithmic-functions/graphs-of-logarithmic-functions/e/graphs-of-exponentials-and-logarithms}
    {Gráficos de Funções Logarítmicas}
\end{onlineact}