\subsection{O Número $e$ e o Logaritmo Natural}

\begin{definition}
Definimos o número $e$ como sendo o número cujos valores aproximados
por falta são os números racionais da forma $$
\paren{1+ \frac 1 n}^n , n\in \N^*.$$ Em outras palavras, quanto
maior for $n \in \N^*$, melhor a aproximação de $\paren{1+ \frac
1 n}^n$ para $e$, e ela se dá na medida que desejarmos.
\end{definition}

\begin{remark}
O número $e$ é irracional. Um valor aproximado dessa importante
constante é $e = 2{,}718281828459$.
\end{remark}

Muito usado como base das funções exponenciais e logarítmicas,
principalmente no estudo dessas funções no Cálculo Infinitesimal, o
logaritmo na base $e$ é tratado de forma especial.

\begin{definition}
Denotamos $$\log_e x = \ln x$$ e o chamamos de \emph{logaritmo
natural}.
\end{definition}

\begin{onlineact}
    \khan{https://pt.khanacademy.org/math/algebra2/exponential-and-logarithmic-functions/solving-exponential-equations-with-logarithms/e/using-logarithms-to-solve-exponential-equations}
    {Solução de Equações Exponenciais Usando
Logaritmos: Base 10 e Base $e$}
\end{onlineact}

\begin{onlineact}
    \khan{https://pt.khanacademy.org/math/algebra2/exponential-and-logarithmic-functions/solving-exponential-models/e/exponential-models-word-problems}
    {Problemas com Modelos Exponenciais}
\end{onlineact}