\section{Introdução}

As funções do tipo exponenciais modelam problemas nos quais o crescimento
é calculado dependendo do valor no momento anterior, como em juros
compostos. Por que será que a expressão ``crescimento exponencial''
é sinônimo de um crescimento muito acentuado?

Além disso, a função exponencial é a única função real contínua que
transforma somas em produtos, ou seja, $$f(x+y) = f(x) \cdot f(y).$$

A função logarítmica, que será apresentada na segunda parte
desse capítulo, é a inversa da função exponencial. Por isso,
teremos que ela é a única função real contínua que transforma
produtos em somas, ou seja,
$$f(xy) = f(x) + f(y).$$
