\section{Exercícios}

\begin{exercise}
    Mostre que a função $f: \Z \to \R$ definida por $f(x)=a^x$ é
crescente se $a>1$ e decrescente se $0<a<1$.
\end{exercise}

\begin{exercise}
    Mostre que a função $f: \Q \to \R$ definida por $f(x)=a^x$ é
crescente se $a>1$ e decrescente se $0<a<1$.
\end{exercise}

\begin{exercise}
Uma alga cresce de modo que, em cada dia, ela cobre uma superfície de área
igual ao dobro da coberta no dia anterior.
%
\begin{enumerate}[a)]
  \item Defina uma função do tipo exponencial que modele o problema.
  \item Se uma alga cobre a superfície de um lago em 100 dias, 
  qual é o número de dias necessários para que três algas, da 
  mesma anterior, cubram a superfície do mesmo lago?
  \item Defina uma função que recebe a quantidade inicial de algas e que calcule
  em quantos dias o lago estará coberto pelas algas.
\end{enumerate}
\end{exercise}

\begin{exercise}
    O gordinho Jaguatirica, certo dia, fez compras em 5 lojas de um
shopping. Em cada loja, gastou metade do que possuía e pagou, na
saída, R\$ 2{,}00 de estacionamento. Se após toda essa atividade
ainda ficou com R\$ 20{,}00, que quantia ele tinha inicialmente?
\end{exercise}

\begin{exercise}
    Se $\paren{a_n}$ é uma PA, prove que $\paren{ b_n}$ definida por
$b_n = e^{a_n}$ é uma PG.
\end{exercise}

\begin{exercise}
    Existe exemplo de função crescente $f : \N \to \R_+^\ast$ tal
que, para todo $x \in \N$, a sequência $f(x)$, $f(x+1)$, $f(x+2)$,
..., $f(x+n)$, ... é uma progressão geométrica mas $f$ não é do tipo
$f(x) = b \cdot a^x$?
\end{exercise}

\begin{exercise}
  Sejam $(x_1,x_2,\dots,x_n,\dots)$ uma PA, $f:\reals \to \reals$ uma função do 
  tipo exponencial tal que $f(x)=ba^x$ e $(y_1,y_2,\dots,y_n,\dots)$ uma sequência
  tal que $f(x_i)=y_i$ para todo $i\in\pnats$. $(y_1,y_2,\dots,y_n,\dots)$ é uma PG?
\end{exercise}

\begin{exercise}
    Uma aplicação rende a \emph{juros compostos} se o rendimento diário é somado ao capital inicial para o cálculo dos juros dos dias seguintes.
    
    Edson faz uma aplicação que rende juros $j>0$ em um mês. Ou seja, se ele investiu um capital inicial $c_0$, então ao fim de 1 mês, Edson poderia resgatar $c = c_0(1+j)$, igual à situação do Exercício \ref{exercicio:juros-simples}.
    \begin{enumerate}[a)]
        \item Caso a aplicação renda juros composto, defina uma função do tipo exponencial que calcule o capital $c_c$ em função do tempo $t$ (em meses) da aplicação;
        \item Suponha que Edson precisará resgatar todo o dinheiro da aplicação em um tempo $t$ menor que um mês. É mais vantajoso aplicar com juros simples ou com juros compostos? Compare com a função criada no Exercício \ref{exercicio:juros-simples} e utilize a Desigualdade de Bernoulli enunciada abaixo.

        \emph{Desigualdade de Bernoulli:}
        Seja $a \in \R$ tal que $a \geq -1$. Seja também $b \in \R$.
        
         Caso $0 < b < 1$. Então $$(1+a)^b \leq 1 + ab.$$
        
         Caso seja $b<0$ ou $b>1$, então $$(1+a)^b \geq 1 + ab.$$
        
         Em ambos os casos, a igualdade só é satisfeita quando $a=0$.
        \item A conclusão do item anterior também é válida caso o tempo de aplicação fosse mais de 1 mês?
\end{enumerate}
\end{exercise}

\begin{exercise}
    Certa cidade teve seus casos de covid-19 perfeitamente monitorados durante $n$ dias. Observou-se que a sequência formada pela quantidade de casos acumulados a cada dia estavam em PG de razão $q>1$.
     \begin{enumerate}[a)]
        \item Se no primeiro dia foi detectado $1$ caso, expresse o termo geral da PG no $i$-ésimo dia do monitoramento;
        \item Defina uma função do tipo exponencial (podendo ser uma translação ou uma dilatação de uma função do tipo exponencial) com domínio $[1, n]$ que modele o problema. Existe alguma função que não seja do tipo exponencial, ou seja, que não se encaixe na descrição anterior, que satisfaça o termo geral do item (a) para todo $i \pertence \conjunto{1, 2, ..., n}$? Justifique.
        \item Estime em que momento do monitoramento a cidade passou a ter uma quantidade $y$ de casos.
    \end{enumerate}
\end{exercise}

\begin{exercise}
    Use as aproximações $\log 2 \approx 0,301$, $\log 3 \approx
0,477$ e $\log 5 \approx 0,699$ para obter valores aproximados para:
\begin{enumerate}[a)]
  \item $\log 9$;
  \item $\log18$;
  \item $\log 40$;
  \item $\log50$;
  \item $\log 200$;
  \item $\log 3000$;
  \item $\log 0{,}003$;
  \item $\log 0{,}81$;
  \item O(s) valor(es) de $x$ que satisfaz(em) a equação $2^x=3$;
  \item O(s) valor(es) de $x$ que satisfaz(em) a equação $5^x=2$.
\end{enumerate}
\end{exercise}

\begin{exercise}
    Mostre que $$\log_{b^n}a^n = \log_{b}a$$ para todo $a, b, n \in \R_+^\ast $ e $b\neq 1$.
\end{exercise}

\begin{exercise}
    Uma interpretação do logaritmo decimal é sua relação com a
\sub{ordem de grandeza}, isto é, com o número de algarismos na
representação decimal. As questões a seguir exploram essa relação.
\begin{enumerate}[a)]
  \item Considere o número $x = 58.932{,}1503$. Qual é a parte inteira de $\log x$?
  \item Considere $x>1$ um número real cuja parte inteira tem $k$ algarismos.
  Use que a função logarítmica é crescente para mostrar que a parte inteira
  de $\log x$ é igual a $k-1$;
  \item Generalizando o item anterior, considere o sistema de
  numeração posicional de base $b \geq 2$. Mostre que, se a
  representação de um número real $x>1$ nesse sistema tem $k$
  algarismos, então, a parte inteira de $\log_b x$ é igual a $k-1$.
\end{enumerate}
\end{exercise}

\begin{exercise}
    (UNIRIO/1994) Um explorador descobriu, na selva amazônica, uma
espécie nova de planta e, pesquisando-a durante anos, comprovou que
o seu crescimento médio variava de acordo com a fórmula $A = 40
\cdot 1{,}1^t$, onde a altura média $A$ é medida em centímetros e o
tempo $t$ em anos. Sabendo-se que $\log 2 \approx 0{,}30$ e $\log 11
\approx 1{,}04$, determine:
\begin{enumerate}[a)]
  \item A altura média, em centímetros, de uma planta dessa espécie
  aos 3 anos de vida;
  \item A idade, em anos, na qual a planta tem uma altura média de
  $1{,}6 m$.
\end{enumerate}
\end{exercise}

\begin{exercise}
    Considere $x, y \in \R$ tais que $x = 10^k y$, com $k\in \Z$.
Qual é a relação entre $\log x $ e $\log y$?
\end{exercise}

\begin{exercise}
    Se $\paren{a_n}$ é uma PG com todos os termos positivos, prove
que $\paren{ b_n}$ definida por $b_n = \ln{a_n}$ é uma PA.
\end{exercise}

\begin{exercise}
    O acidente do reator nuclear de Chernobyl, URSS, em 1986, lançou
na atmosfera grande quantidade do isótopo radioativo estrôncio-90,
cuja meia-vida (tempo necessário para que uma substância seja
reduzida à metade da quantidade inicial) é de vinte e oito anos, ou
seja, sendo $f$  a função do tipo exponencial de base $a$ que modele a
quantidade de estrôncio-90 em função do tempo, tem-se $f(28) = \dfrac
{f(0)} 2$. Supondo ser este isótopo a única contaminação
radioativa e sabendo que o local poderá ser considerado seguro
quando a quantidade de estrôncio-90 se reduzir, por desintegração, a
$\frac 1 {16}$ da quantidade inicialmente presente, em que ano o
local poderá ser habitado novamente?
\end{exercise}
