\section{Exercícios}

\begin{exercise}
    Mostre que a função $f: \Z \to \R$ definida por $f(x)=a^x$ é
crescente se $a>1$ e decrescente se $0<a<1$.
\end{exercise}

\begin{exercise}
    Mostre que a função $f: \Q \to \R$ definida por $f(x)=a^x$ é
crescente se $a>1$ e decrescente se $0<a<1$.
\end{exercise}

\begin{exercise}
    Uma alga cresce de modo que, em cada dia, ela cobre uma
superfície de área igual ao dobro da coberta no dia anterior. Se
esta alga cobre a superfície de um lago em 100 dias, qual é o número
de dias necessários para que duas algas, da mesma espécie anterior,
cubram a superfície do mesmo lago? E se forem quatro algas? Você
consegue responder esta pergunta para 3 algas?
\end{exercise}

\begin{exercise}
    O gordinho Jaguatirica, certo dia, fez compras em 5 lojas de um
shopping. Em cada loja, gastou metade do que possuia e pagou, na
saída, R\$ 2{,}00 de estacionamento. Se após toda essa atividade
ainda ficou com R\$ 20{,}00, que quantia ele tinha inicialmente?
\end{exercise}

\begin{exercise}
    Se $\paren{a_n}$ é uma PA, prove que $\paren{ b_n}$ definida por
$b_n = e^{a_n}$ é uma PG.
\end{exercise}

\begin{exercise}
    Existe exemplo de função crescente $f : \N \to \R_+^\ast$ tal
que, para todo $x \in \N$, a sequência $f(x)$, $f(x+1)$, $f(x+2)$,
..., $f(x+n)$, ... é uma progressão geométrica mas $f$ não é do tipo
$f(x) = b \cdot a^x$?
\end{exercise}

\begin{exercise}
    Use as aproximações $\log 2 \approx 0,301$, $\log 3 \approx
0,477$ e $\log 5 \approx 0,699$ para obter valores aproximados para:
\begin{enumerate}[(a)]
  \item $\log 9$;
  \item $\log 40$;
  \item $\log 200$;
  \item $\log 3000$;
  \item $\log 0{,}003$;
  \item $\log 0{,}81$.
\end{enumerate}
\end{exercise}

\begin{exercise}
    Uma interpretação do logaritmo decimal é sua relação com a
\sub{ordem de grandeza}, isto é, com o número de algarismos na
representação decimal. As questões a seguir exploram essa relação.
\begin{enumerate}[(a)]
  \item Considere o número $x = 58.932{,}1503$. Qual é a parte inteira de $\log x$?
  \item Considere $x>1$ um número real cuja parte inteira tem $k$ algarismos.
  Use que a função logarítmica é crescente para mostrar que a parte inteira
  de $\log x$ é igual a $k-1$;
  \item Generalizando o item anterior, considere o sistema de
  numeração posicional de base $b \geq 2$. Mostre que, se a
  representação de um número real $x>1$ nesse sistema tem $k$
  algarismos, então, a parte inteira de $\log_b x$ é igual a $k-1$.
\end{enumerate}
\end{exercise}

\begin{exercise}
    (UNIRIO/1994) Um explorador descobriu, na selva amazônica, uma
espécie nova de planta e, pesquisando-a durante anos, comprovou que
o seu crescimento médio variava de acordo com a fórmula $A = 40
\cdot 1{,}1^t$, onde a altura média $A$ é medida em centímetros e o
tempo $t$ em anos. Sabendo-se que $\log 2 \approx 0{,}30$ e $\log 11
\approx 1{,}04$, determine:
\begin{enumerate}[(a)]
  \item A altura média, em centímetros, de uma planta dessa espécie
  aos 3 anos de vida;
  \item A idade, em anos, na qual a planta tem uma altura média de
  $1{,}6 m$.
\end{enumerate}
\end{exercise}

\begin{exercise}
    Considere $x, y \in \R$ tais que $x = 10^k y$, com $k\in \Z$.
Qual é a relação entre $\log x $ e $\log y$?
\end{exercise}

\begin{exercise}
    Se $\paren{a_n}$ é uma PG com todos os termos positivos, prove
que $\paren{ b_n}$ definida por $b_n = \ln{a_n}$ é uma PA.
\end{exercise}

\begin{exercise}
    O acidente do reator nuclear de Chernobyl, URSS, em 1986, lançou
na atmosfera grande quantidade do isótopo radioativo estrôncio-90,
cuja meia-vida (tempo necessário para que uma substância seja
reduzida à metade da quantidade inicial) é de vinte e oito anos, ou
seja, sendo $f$  a função exponencial de base $a$ que modele a
quantidade de estrôncio-90 em função do tempo, tem-se $\log_a \frac
{f(0)} 2 = 28$. Supondo ser este isótopo a única contaminação
radioativa e sabendo que o local poderá ser considerado seguro
quando a quantidade de estrôncio-90 se reduzir, por desintegração, a
$\frac 1 {16}$ da quantidade inicialmente presente, em que ano o
local poderá ser habitado novamente?
\end{exercise}
