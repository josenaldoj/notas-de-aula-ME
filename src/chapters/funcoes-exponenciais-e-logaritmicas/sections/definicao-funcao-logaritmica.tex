\subsection{Definição}

\begin{definition}
A inversa da função exponencial de base $a$ é a \emph{função
logarítmica}
$$\log_a : \R_+^* \to \R,$$
que associa a cada número real positivo $x$ o número real $\log_a
x$, chamado \emph{logaritmo} de $x$ na base $a$. No caso de $a=10$,
escrevemos, por simplicidade, $\log_{10}x = \log x$.
\end{definition}

\begin{remark}
	Pela definição de função inversa, tem-se
$$ a^{\log_a x}=x \ \ \ \text{ e } \ \ \ \log_a \paren{a^x} = x.$$
Assim, $\log_a x $ é o expoente ao qual se deve elevar a base $a$
para obter o número $x$. Em outras palavras,
$$ y = \log_a x \iff a^y = x.$$
\end{remark}

\begin{proposition}
	Seja $f: \R_+^* \to \R$ uma função logarítmica tal que $f(x) =
\log_a x$. Os seguintes valem para quaisquer  $x, y, b \in
\R_+^*$, $b \neq 1$ e qualquer $k \in \R$:
\begin{enumerate}[(a)]
  \item $\log_a \paren{xy} = \log_a x + \log_a y$;
  \item $\log_a x^k = k\cdot \log_a x$;
  \item $\log_a 1 = 0$ e $\log_a a = 1$;
  \item $\log_a x = \frac{\log_b x}{\log_b a}$;
  \item $f$ é bijetiva com contradomínio $\R$, logo é ilimitada superiormente e inferiormente;
  \item O gráfico de $f$ é traçado por uma linha contínua;
  \item $f$ é crescente se $a>1$ e decrescente se $0<a<1$.
\end{enumerate}
\end{proposition}

\begin{proof}
	Sejam $x, y, b \in \R_+^*$, com $b \neq 1$, e $k \in \R$.
	%
	\begin{enumerate}[(a)]
		\item 
		Seja $z \in \R$. Ora,
		%
		\[
			z = \log_a x + \log_a y \iff a^z = a^{\log_a x + \log_a y} 
		\]
		%
		Note que $a^{\log_a x + \log_a y}  = a^{\log_a x} \cdot a^{\log_a y} = xy$.
		Logo, 
		%
		\begin{align*}
			z = \log_a x + \log_a y & \iff a^z = a^{\log_a x + \log_a y}\\
			& \iff a^z = xy \\
			& \iff z = \log_a(xy)
		\end{align*}
		%
		Portanto, $\log_a(xy)=\log_a x + \log_a y$.

		\item Seja $z \in \R$.
		Ora, 
		%
		\[
			z = k \log_a x \iff a^z = a^{k\log_a x}
		\]
		%
		Note que $a^{k\log_a x} = \prn{a^{\log_a x}}^k = x^k$.
		Logo:
		%
		\begin{align*}
			z = k \log_a x & \iff a^z = x^k \\ &\iff z = \log_a {x^k}
		\end{align*}
		%
		Portanto, $\log_a{x^k}=k\log_a x$

		\item Temos que $a^0 = 1$, e, portanto, $\log_a 1 = 0$.

		\item Note que $x = a^{\log_a x} = \prn{b^{\log_b a}}^{\log_a x} = b^{\log_b a \cdot \log_a x}$.
		Com isso, temos que $\log_b x = \log_b a \cdot \log_a x$. Portanto, $\log_a x = \frac{\log_b x}{\log_b a}$.

		\item A função $f$ tem inversa --- a saber, a exponencial $g(x) =  a^x$ ---, e, portanto, pelo Teorema \ref{theo:inv-sse-bij}, é bijetiva.
		Consequentemente, sua imagem é o conjunto $\reais$, o que implica que é ilimitada tanto superiormente quanto inferiormente.

		\item A explicação do porquê de o gráfico de $f$ ser traçado por uma linha contínua está fora do escopo do texto.

		\item Temos, como uma das consequências da Proposição~\ref{prop:propriedades-funcao-exponencial}, que a inversa de $f$ é crescente
		quando $a > 1$ ou decrescente quando $0 < a <1$, além de ser bijetiva.
		Logo, sua inversa, que é a própria função $f$ pela Definição~\ref{def:funcao-inversa}, também é crescente quando $a > 1$ ou decrescente
		quando $0 < a < 1$ (Exercício \ref{exer:inversa-funcao-crescente}).


	\end{enumerate}
\end{proof}

\begin{onlineact}
	\khan{https://pt.khanacademy.org/math/algebra2/exponential-and-logarithmic-functions/introduction-to-logarithms/e/logarithms_1.5}{Cálculo de Logaritmos (Avançado)}.
\end{onlineact}

\begin{onlineact}
	\khan{https://pt.khanacademy.org/math/algebra2/exponential-and-logarithmic-functions/properties-of-logarithms/e/logarithms_2}{Use as Propriedades dos Logaritmos}.
\end{onlineact}

\begin{onlineact}
	\khan{https://pt.khanacademy.org/math/algebra2/exponential-and-logarithmic-functions/change-of-base-formula-for-logarithms/e/rewrite-logarithmic-expressions-using-the-change-of-base-rule}
	{Use a Regra da Mudança de Base dos Logaritmos}
\end{onlineact}

