\subsection{Caracterização}

\begin{theorem}[Caracterização da Função Exponencial]
\label{theo:caracterizacao-funcao-exponencial}
Seja $f: \R \to \R_+^*$ uma função monótona injetiva. As
seguintes afirmações são equivalentes:
\begin{enumerate}[(i)]
  \item $f(nx) = f(x)^n$ para todo $n \in \Z$ e todo $x \in \R$;
  \item $f(x) = a^x$ para todo $x\in \R$, onde $a = f(1)$;
  \item $f(x+y) = f(x)\cdot f(y)$ para quaisquer $x, y \in \R$.
\end{enumerate}
\end{theorem}

\begin{proof}
  \item (i)$\implies$(ii): fora do escopo
  \item (ii)$\implies$(iii): Suponha que $f(x) = a^x$ para todo $x \in \reais$.
  
  Sejam $x,y \in \reais$. Note que:
  %
  \[
      f(x+y)=a^{x+y}=a^x \cdot a^y = f(x)\cdot f(y)
  \]
  \item (iii)$\implies$(i): Suponha que $f(x+y) = f(x)\cdot f(y)$ para todo $x,y\in\reais$. 
  Temos que $f(x)=f(x+0) = f(x)\cdot f(0)$. 
  Logo, $f(0)=1=\prn{f(x)}^0$.
  Além disso, para qualquer $x\in\reais$,
  \[
      1=f(0)=f(x-x)=f(x)\cdot f(-x).
  \]
  Logo, $f(-x) = \frac{1}{f(x)} = \prn{f(x)}^{-1}$.

  Sejam $n \in \nnats$ e $x \in \reais$. Note que:
  %
  \[
      f(nx)=f(\underbrace{x+x+\dots + x}_{\text{$n$ termos}}) = 
      \underbrace{f(x)\cdot f(x) \cdot \dots \cdot f(x)}_{\text{$n$ termos}} = f(x)^n
  \]

  Agora, sejam $n \in \nints$ e $x \in \reais$. Assim, $-n \in \pnats$, e
  %
  \[
      f(nx) = f\bracket{-\prn{-nx}} = \bracket{f(-nx)}^{-1} = \bracket{f(x)^{-n}}^{-1} = f(x)^n.
  \]
\end{proof}