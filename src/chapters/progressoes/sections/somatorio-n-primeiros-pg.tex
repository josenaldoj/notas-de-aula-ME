\section{Somatório dos $n$ primeiros termos de uma PG}

\begin{proposition}[Soma dos $n$ primeiros termos de uma PG]
A soma dos $n$ primeiros termos de uma PG $\prn {a_n}$ de razão $q \ne 1$ é:
\begin{equation*}
S_n = a_1 \frac{1-q^n}{1-q}.
\end{equation*}
\end{proposition}

\begin{proof}
Seja $(a_n)_{n \in \nnats}$ uma PG de razão $q$. Considere
%
\begin{equation}
\label{equation:sum-pg-1}
S_n = a_1 + a_2 + \dots + a_n.
\end{equation}
%
Logo, 
%
\begin{equation}
\label{equation:sum-pg-2}
S_n \cdot q = a_2 + a_3 + \dots + a_n + a_{n+1}.
\end{equation}
%
Calculando \ref{equation:sum-pg-1} $-$ \ref{equation:sum-pg-2}:
%
\begin{align*}
S_n - S_n \cdot q = a_1 - a_{n+1} = a_1 - a_1 \cdot q^n & \iff S_n\cdot\prn{1-q} = a_1\cdot\prn{1-q^n} \\ &\iff S_n = a_1 \cdot \frac{1-q^n}{1-q},
\end{align*}
%
\noindent o que conclui a demonstração.

\end{proof}

\begin{example}
Diz a lenda que o inventor do xadrez pediu como recompensa 1 grão de trigo pela primeira casa, 2 grãos pela segunda, 4 pela terceira e assim por diante, sempre dobrando a quantidade a cada nova casa. 
Sabendo que o tabuleiro de xadrez tem 64 casas, qual o número de grãos pedido pelo inventor do jogo?
\end{example}

\begin{solution}
Pelo seu enunciado, o problema pode ser modelado por uma PG $(a_1, a_2, \dots, a_{64})$ tal que $a_i$, $1 \le i \le n$, é o número de grãos que o inventor pediu pela $i$-ésima casa do tabuleiro.
A partir da equação para a soma dos $n$ primeiros termos de uma PG, conclui-se que o inventor pediu, no total, $1 \cdot \frac {1 - 2^{64}} {1-2} = 2^{64}-1 = 18446744073709551615$ grãos.
\end{solution}