\section{Inclusão}

\begin{definition}[Relação de Inclusão]
	\label{def:inclusao}
	Considere dois conjuntos $A$ e $B$. Se for o caso que todo elemento de $A$ é, também, elemento de $B$, diz-se que $A$ é um \emph{subconjunto} de $B$, que $A$ está \emph{contido} em $B$, ou que $A$ é \emph{parte} de $B$. Para indicar esse fato, usa-se a notação $A \contido B$.

	\label{def:naoinclusao}
	Quando $A$ \textit{não} está contido em $B$,  utiliza-se a notação  $A \naocontido B$. Na prática, isso significa que existe pelo menos um elemento que está em $A$ mas não está em $B$. Em outras palavras, existe um elemento $x$ tal que $x \em A$ mas $x \naopertence B$.
\end{definition}

\begin{remark}
	Também podemos escrever $B \contem A$ quando for o caso que $A \contido B$. Para essa situação, dizemos que $B$ \emph{contém} $A$.
\end{remark}

\begin{example}
	Considere $T$ o conjunto de todos os triângulos e $P$ o conjunto dos polígonos no plano. Como todo triângulo é um polígono podemos concluir que $T \contido P$. Observe também que $P \naocontido T$. Para poder concluir isso precisamos encontrar um elemento de $P$ que não seja um elemento de $T$. Ora, basta considerar um quadrado $q$ com lados de tamanho 1, como todo quadrado é um polígono, temos que $q \pertence P$, mas quadrados não são triangulos, então $q \naopertence T$.
\end{example}

\begin{example}
	Na Geometria, uma reta, um plano e o espaço são conjuntos. Seus elementos são pontos. Quando dizemos que uma reta $r$ está no plano $\Pi$, estamos afirmando que $r$ está contida em $\Pi$ ou, equivalentemente, que $r$ é um subconjunto de $\Pi$, pois todos os pontos que pertencem a $r$ pertencem, também, a $\Pi$. Nesse caso, deve-se escrever $r \contido \Pi$. Porém, não é correto dizer que $r$ pertence a $\Pi$, nem escrever $r \pertence \Pi$. Os elementos do conjunto $\Pi$ são pontos, não retas.
\end{example}

\begin{example}
	Considere os conjuntos $N = \conjunto{1, 2, 3, 4, 5, 6}$, $I = \conjunto{1, 3, 5}$ e $P = \conjunto{0, 2, 4, 6}$. Analisando o cenário, podemos concluir que:
	\begin{enumerate}
		\item $I \contido N$. Observe que todos os elementos de $I$ também são elementos de $N$.
		\item $P \naocontido N$. Observe que nem todos os elementos de $P$ são elementos de $N$, pois $0 \pertence P$ mas $0 \naopertence N$.
	\end{enumerate}
\end{example}

\begin{proposition}[Inclusão universal do $\vazio$]
	\label{prop:emptyset1}
	Para todo conjunto $A$, tem-se que o conjunto $\vazio$ é subconjunto de $A$ (em símbolos: $\emptyset \subset A$).
\end{proposition}

\begin{proof}
	Para chegarmos num absurdo, considere que existe um conjunto $A$ tal que $\vazio \naocontido A$. Logo, podemos concluir que existe um elemento $x$ tal que $x \pertence \vazio$ mas $x \naopertence A$, pela definição de inclusão. Mas, já sabemos que $x \naopertence \vazio$ pela definição do conjunto vazio. O que nos leva a um absurdo, pois não pode acontecer que $x \pertence \vazio$ e $x \naopertence \vazio$ ao mesmo tempo. Portanto, podemos concluir que $\vazio \contido A$.
\end{proof}

\begin{remark}
	Ao manter a arbitrariedade de um conjunto, qualquer conclusão relacionada a este conjunto valerá para todos os conjuntos.
\end{remark}

\begin{definition}[Inclusão Própria]
	Dizemos que $A \diferente \vazio$ é um \emph{subconjunto próprio} de $B$ quando $A \contido B$ mas $A \diferente B$.
\end{definition}

\begin{proposition}[Propriedades da inclusão]
	Para quaisquer conjuntos $A$, $B$ e $C$, são válidas as propriedades a seguir:
	\begin{enumerate}
		\item
			\label{inclusao:reflexividade}
			\emph{Reflexividade}: $A \contido A$;
		\item
			\label{inclusao:antissimetria}
			\emph{Antissimetria}: Se $A \contido B$ e $B \contido A$, então $A = B$;
		\item
			\label{inclusao:transitividade}
			\emph{Transitividade}: Se $A \contido B$ e $B \contido C$, então $A \contido C$.
	\end{enumerate}
\end{proposition}

\begin{proof}
	Primeiramente, demonstremos o item~\ref{inclusao:reflexividade}. Seja $x \em A$ um elemento arbitrário. Ora, como já temos que $x \pertence A$ podemos concluir que $A \contido A$.

	Segundamente, demonstremos o item~\ref{inclusao:antissimetria}. Sejam $A$ e $B$ conjuntos tais que $A \contido B$ e $B \contido A$. Suponha, por contradição, que $A \ne B$, ou seja, existe $x \em A$ tal que $x \naopertence B$ (1) ou existe $x \em B$ tal que $x \naopertence A$ (2). Ora, (1) é o mesmo que $A \naocontido B$, contradizendo $A \contido B$. Analogamente, (2) contradiz $B \contido A$. Portanto, $A = B$.

	Por fim, demonstremos que vale o item~\ref{inclusao:transitividade}. Sejam $A$, $B$ e $C$ conjuntos tais que $A \contido B$ e $B \contido C$. Agora basta demonstrar que $A \contido C$. Para isso, considere $x \pertence A$ um elemento arbitrário e mostremos que $x \pertence C$. Como temos que $A \contido B$, podemos concluir que $x \pertence B$. E, como $x \pertence B$ e $B \contido C$, segue que $x \pertence C$. Portanto, $A \contido C$.
\end{proof}

\begin{definition}[Conjunto das Partes]
	\label{def:powerset} % substituir label para def:conjunto-das-partes
	Dado um conjunto $A$, chamamos de \emph{conjunto das partes} de $A$ o conjunto formado por todos os seus subconjuntos, e denotamo-lo $\partes A$. Em símbolos:
	\[
		\partes A = \conjunto{ X \taisque X \contido A}.
	\]
\end{definition}

\begin{example}
	\label{exem-powerset-basico}
	Dado $A = \conjunto{1, 2, 3}$, determine $\partes A$.
\end{example}

\begin{solution}
	$\partes A = \conjunto{\emptyset, \unitario{1}, \unitario{2}, \unitario{3}, \conjunto{1,2}, \conjunto{2,3}, \conjunto{1,3}, A}$.
\end{solution}

\begin{remark}
	Como o Exemplo~\ref{exem-powerset-basico} ilustra, o conjunto das partes de um determinado conjunto tem a particularidade de que todos os seus elementos também são conjuntos.
\end{remark}
