\section{Introdução}
\label{sec:intro}

Um \emph{conjunto} é definido por seus elementos, e nada mais.
Desse fato decorre, imediatamente, que dois conjuntos são \emph{iguais} quando possuem os mesmos elementos, e apenas nessa situação. 

Dados um conjunto $A$ e um objeto qualquer $x$, é natural perguntar se $x$ é um elemento do conjunto $A$. Tal questionamento admite apenas \entreaspas{sim} ou \entreaspas{não} como candidatos a resposta. Isso se dá porque, na Matemática, qualquer afirmação é verdadeira ou é falsa, não havendo possibilidade de ser as duas coisas simultaneamente nem de, ainda, haver uma terceira opção. 

\caixote{Quando a Matemática \entreaspas{falha}}{
    O caráter binário e exclusivo do valor-verdade das afirmações faz parecer que a Matemática é infalível se usada corretamente, o que não se verifica de fato. O matemático austríaco Kurt Gödel provou, em 1931, que todo sistema formal que inclua a aritmética é falho pois possui verdades que não podem ser provadas -- os chamados paradoxos. Antes de assistir ao vídeo \href{https://youtu.be/UI1xR_AECrU}{\texttt{Este vídeo está mentindo}}, reflita se você vai acreditar nele ou não.
}

A descrição dos elementos de um conjunto -- necessária para defini-lo, conforme já comentado -- pode ser feita textualmente. Contudo, nestas notas serão utilizadas duas formas matemáticas comuns de se especificar tal descrição, apresentadas nos Exemplos~\ref{ex-vogais} e~\ref{ex-primos-pares}. 

\begin{example}
    \label{ex-vogais}
    Se quisermos expressar qual seria o conjunto de todas as vogais do nosso alfabeto, precisaríamos de alguma notação para representá-lo. Temos $V = \conjunto{\texttt a, \texttt e, \texttt i, \texttt o, \texttt u}$ como sendo o conjunto das vogais. Tal notação lista explicitamente os membros de seu conjunto, colocando-os entre chaves e separados por vírgula.
\end{example}

\begin{example}
    \label{ex-primos-pares}
    O conjunto $P$ dos números primos pares pode ser representado por $P = \conjunto{ x \taisque \text{$x$ é primo e par}} = \unitario{2}$. Tal notação é lida por \entreaspas{o conjunto de todos os $x$'s tais que $x$ é primo e par}. Entre chaves é colocada a representação para o elemento e sua descrição separados por \entreaspas{;}, \entreaspas{\contrabarra}, ou \entreaspas{|}.  Não é correto escrever $P = \conjunto{\text{ números primos pares }}$, por exemplo.
\end{example}
