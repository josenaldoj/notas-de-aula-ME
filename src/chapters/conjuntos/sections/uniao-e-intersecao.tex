\subsection{União e Interseção}

\begin{definition}[União e Interseção]
Dados os conjuntos $A$ e $B$, definem-se:
\begin{enumerate}
    \item
        A \textdef{união} $A \cup B$ como sendo o conjunto formado pelos elementos que pertencem a pelo menos um dos conjuntos $A$ e $B$. Ou seja,
    $$ A \cup B = \set{x \tq x \in A \text{ ou } x \in B} . $$
    
    \item
        A \textdef{interseção} $A \cap B$ como sendo o conjunto formado pelos elementos que pertencem a ambos $A$ e $B$. Ou seja,
        $$ A \cap B = \set{x \tq x \in A \text{ e } x \in B}. $$
\end{enumerate}
\end{definition}

\begin{example}
Sejam $A = \set{1, 2, 3}$ e $ B = \set{2,5}$. Determine $A \cup B$ e $A \cap B$.
\end{example}

\begin{solution}
\begin{align*}
    A \cup B &= \set{1,2,3,5};\\
    A \cap B &= \set{2}.
\end{align*}
\end{solution}

\begin{proposition}[Propriedades da união e interseção]
\label{prop:uniao-e-intersecao}
Para quaisquer conjuntos $A$, $B$ e $C$, tem-se:
\begin{enumerate}[i)]
    \item
        \textdef{Comutatividade}:
        \begin{enumerate}[a)]
            \item $A \cup B = B \cup A$;
            \item $A \cap B = B \cap A$.
        \end{enumerate}

    \item
        \textdef{Associatividade}:
        \begin{enumerate}[a)]
            \item $\left(A \cup B \right) \cup C = A \cup \left( B \cup C \right)$;
            \item $\left(A \cap B \right) \cap C = A \cap \left( B \cap C \right)$.
        \end{enumerate}

    \item
        \begin{enumerate}[a)]
            \label{prop:uniao-e-intersecao-inclusao}
            \item
                \label{prop:uniao-inclusao}
                $A \subset \prn{A \cup B}$;
            \item
                \label{prop:intersecao-inclusao}
                $\prn{A \cap B} \subset A$.
        \end{enumerate}

    \item
        \textdef{Distributividade}, de uma em relação à outra:
        \begin{enumerate}[a)]
            \item $A \cap \left( B \cup C \right) = \left(A \cap B \right) \cup \left( A \cap C \right)$;
            \item $A \cup \left( B \cap C \right) = \left(A \cup B \right) \cap \left( A \cup C  \right)$.
        \end{enumerate}

    \item
        \label{prop:demorgan}
        \textdef{Leis de DeMorgan}:
        \begin{enumerate}[a)]
            \item $\left( A \cup B \right)^C = A^C \cap B^C$;
            \item $\left(A \cap B \right)^C = A^C \cup B^C$.
        \end{enumerate}
\end{enumerate}
\end{proposition}

\begin{proof}
Abaixo, seguem as demonstrações de cada propriedade.
\begin{enumerate}[i)]
    \item Exercício.

    \item
        \begin{enumerate}[a)]
            \item Exercício.
            \item
                Provemos que $A \cap (B \cap C) \subset (A \cap B) \cap C$. Seja $x \in A \cap (B \cap C)$, isso é, $x \in A$ e $x \in B \cap C$. De $x \in B \cap C$, temos $x \in B$ e $x \in C$. Como $x \in A$ e $x \in B$, segue que $x \in A \cap B$. Além disso, $x \in C$. Então, $x \in A \cap (B \cap C)$. Logo, $A \cap (B \cap C) \subset (A \cap B) \cap C$. \\
                A prova de que $A \cap (B \cap C) \supset (A \cap B) \cap C$, necessária para concluir a igualdade desejada, fica como exercício para o leitor.
        \end{enumerate}

    \item 	
        \begin{enumerate}[a)]
            \item
                Seja $x \in A$. Pela definição de união, segue que $x \in A \cup B$. Portanto, $A \subset \prn{A \cup B}$;
            \item
                Seja $x \in \prn{A \cap B}$. Pela definição de interseção, segue que $x\in A$ e $x \in B$. Em particular, já temos que $x \in A$. Portanto, $\prn{A \cap B} \subset A$.
        \end{enumerate}
        Em decorrência dessa propriedade, vamos tratar como imediatos que: se $x \in A$, então $x \in \prn{A \cup B}$; e se $x \in \prn{A \cap B}$, então $x\in A$ (ou, caso convenha, $x\in B$).

    \item Exercício.

    \item Abaixo, seguem as demonstrações das leis de De Morgan.
        \begin{enumerate}[a)]
            \item
                Inicialmente, demonstremos que $(A \cup B)^C \subset A^C \cap B^C$. Do item \ref{prop:uniao-inclusao} da propriedade \ref{prop:uniao-e-intersecao-inclusao}, temos que $A \subset A \cup B$. Segue do item \ref{prop:complementar:contrapositiva} da proposição \ref{prop:complementar} que $(A \cup B)^C \subset A^C$ (i). De forma análoga, também temos que $(A \cup B)^C \subset B^C$ (ii). Seja agora $x \in (A \cup B)^C$. Por (i), temos $x \in A^C$. Por (ii), temos $x \in B^C$. Logo, $x \in A^C \cap B^C$. Portanto, $(A \cup B)^C \subset A^C \cap B^C$. \\
                Finalmente, demonstremos que $A^C \cap B^C \subset (A \cup B)^C$. Seja $x \in A^C \cap B^C$, ou seja, $x \in A^C$ e $x \in B^C$. Dessa forma, $x \not\in A$ e $x \not\in B$. Suponha, por contradição, que $x \in A \cup B$. Dessa maneira, teríamos $x \in A$, o que contradiz $x \not\in A$, ou $x \in B$, o que contradiz $x \not\in B$. Logo, $x \not\in A \cup B$, ou seja, $x \in (A \cup B)^C$. Assim, $A^C \cap B^C \subset (A \cup B)^C$. Portanto, $(A \cup B)^C = A^C \cap B^C$.
            \item
                Primeiramente, demonstremos que $A^C \cup B^C \subset (A \cap B)^C$. Observe que, pelo item \ref{prop:intersecao-inclusao} da propriedade \ref{prop:uniao-e-intersecao-inclusao}, temos $A \cap B \subset A$, e, pelo item \ref{prop:complementar:contrapositiva} da proposição \ref{prop:complementar}, temos $A^C \subset (A \cap B)^C$ (i). De forma análoga, temos $B^C \subset (A \cap B)^C$ (ii). Então, seja $x \in A^C \cup B^C$, isto é, $x \in A^C$ ou $x \in B^C$. Caso que $x \in A^C$, podemos afirmar que $x \in (A \cap B)^C$ por (i). E, caso $x \in B^C $, por (ii), também podemos afirmar que $x \in (A \cap B)^C$. Portanto, temos que $A^C \cup B^C \subset (A \cap B)^C$. \\
                Finalmente, demonstremos que $(A \cap B)^C \subset A^C \cup B^C$. Suponha, para chegar num absurdo, que $(A \cap B)^C \not\subset A^C \cup B^C$. Então, existe $x \in (A \cap B)^C$ tal que $x \not\in A^C \cup B^C$. Logo, $x \not\in A \cap B$. Temos 4 possibilidades para $x$: \\
                Caso $x \in A$ e $x \in B$: Então, $x \in A \cap B$, o que contradiz $x \not\in A \cap B$. \\
                Caso $x \not\in A$ e $x \in B$: Então, $x \in A^C$. Daí, $x \in A^C \cup B^C$, o que contradiz $x \not\in A^C \cup B^C$.\\
                Caso $x \in A$ e $x \not\in B$: Então, $x \in B^C$. Daí, $x \in A^C \cup B^C$, o que contradiz $x \not\in A^C \cup B^C$.\\
                Caso $x \not\in A$ e $x \not\in B$: De $x \not\in $, temos $x \in A^C$. Daí, $x \in A^C \cup B^C$, o que contradiz $x \not\in A^C \cup B^C$.\\
                Sendo assim, como os 4 casos levam a um absurdo, podemos concluir que $(A \cap B)^C \subset A^C \cup B^C$.
        \end{enumerate}
\end{enumerate}
\end{proof}

\begin{onlineact}
\khan{https://pt.khanacademy.org/math/statistics-probability/probability-library/basic-set-ops/e/basic_set_notation}{Notação Básica de Conjunto}.
\end{onlineact}
