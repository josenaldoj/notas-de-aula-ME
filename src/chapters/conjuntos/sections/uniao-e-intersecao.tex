\subsection{União e Interseção}

\begin{definition}[União e Interseção]
    Dados os conjuntos $A$ e $B$, definem-se:
    \begin{enumerate}
        \item
        A \emph{união} $A \uniao B$ como sendo o conjunto formado pelos elementos que pertencem a pelo menos um dos conjuntos $A$ e $B$. Ou seja,
        \[
            A \uniao B = \conjunto{ x \tq x \em A \ou x \em B}.
        \]
    
        \item
        A \emph{interseção} $A \inter B$ como sendo o conjunto formado pelos elementos que pertencem a ambos $A$ e $B$. Ou seja,
        \[
            A \inter B = \conjunto{x \tq x \em A \e x \em B}.
        \]
    \end{enumerate}
\end{definition}

\begin{example}
    Sejam $A = \conjunto{1, 2, 3}$ e $ B = \conjunto{2,5}$. Determine $A \uniao B$ e $A \inter B$.
\end{example}

\begin{solution}
    \begin{align*}
        A \uniao B &= \conjunto{1,2,3,5};\\
        A \inter B &= \conjunto{2}.
    \end{align*}
\end{solution}

\begin{proposition}[Propriedades da união e interseção]
    \label{prop:uniao-e-intersecao}
    Para quaisquer conjuntos $A$, $B$ e $C$, e fixado um universo $\U$, tem-se:
    \begin{enumerate}
        \item
            \begin{enumerate}
                \label{prop:uniao-e-intersecao-inclusao}
                \item
                    $A \contido {A \uniao B}$;
                \item
                    ${A \inter B} \contido A$.
            \end{enumerate}

        \item
            União/interseção com o universo:
            \begin{enumerate}
                \item $A \uniao \universo = \universo$;
                \item $A \inter \universo = A$.
            \end{enumerate}

        \item
            \emph{Comutatividade}:
            \begin{enumerate}
                \item $A \uniao B = B \uniao A$;
                \item $A \inter B = B \inter A$.
            \end{enumerate}

        \item
            \emph{Associatividade}:
            \begin{enumerate}
                \item $\prn{A \uniao B} \uniao C = A \uniao \prn{ B \uniao C}$;
                \item $\prn{A \inter B} \inter C = A \inter \prn{ B \inter C}$.
            \end{enumerate}

        \item
            \emph{Distributividade}, de uma em relação à outra:
            \begin{enumerate}
                \item $A \inter (B \uniao C) = (A \inter B) \uniao (A \inter C)$;
                \item $A \uniao (B \inter C) = (A \uniao B) \inter (A \uniao C)$.
            \end{enumerate}

        \item
            \label{prop:demorgan}
            \emph{Leis de DeMorgan}:
            \begin{enumerate}
                \item $(A \uniao B)^C = A^C \inter B^C$;
                \item $(A \inter B)^C = A^C \uniao B^C$.
            \end{enumerate}
    \end{enumerate}
\end{proposition}

\begin{proof}
    Abaixo, seguem as demonstrações de cada propriedade.
    \begin{enumerate}
        \item 	
            \begin{enumerate}
                \item
                    Seja $x \em A$. Pela definição de união, segue que $x \pertence A \uniao B$. Portanto, $A \contido A \uniao B$;
                \item
                    Seja $x \em A \inter B$. Pela definição de interseção, segue que $x \pertence A$ e $x \pertence B$. Em particular, já temos que $x \pertence A$. Portanto, $A \inter B \contido A$.
            \end{enumerate}
            Em decorrência dessa propriedade, vamos tratar como imediatos que: se $x \pertence A$, então $x \pertence A \uniao B$ (ou, de maneira análoga, que $x \pertence B \uniao A$); e se $x \pertence A \inter B$, então $x \pertence A$ (ou, caso convenha, $x \pertence B$).

        \item 
            \begin{enumerate}
                \item
                    Exercício.

                \item
                    Seja $x \em A$. Temos que $x \pertence \U$, logo, $x \pertence A \inter \U$. Segue, assim, que $A \contido A\inter \U$. A inclusão $A \inter \U \contido A$, por sua vez, é imediata pela propriedade~\ref{prop:uniao-e-intersecao-inclusao} da Proposição~\ref{prop:uniao-e-intersecao}. Portanto, $A \inter \U = A$.
            \end{enumerate}
        
        \item Exercício.

        \item
            \begin{enumerate}
                \item
                    Exercício.

                \item
                    Provemos que $A \inter (B \inter C) \contido (A \inter B) \inter C$. Seja $x \em A \inter (B \inter C)$, isto é, $x \pertence A$ e $x \pertence B \inter C$. De $x \pertence B \inter C$, temos $x \pertence B$ e $x \pertence C$. Como $x \pertence A$ e $x \pertence B$, segue que $x \pertence A \inter B$. Além disso, $x \pertence C$. Então, $x \pertence (A \inter B) \inter C$. Logo, $A \inter (B \inter C) \contido (A \inter B) \inter C$.

                    A prova de que $A \inter (B \inter C) \contem (A \inter B) \inter C$, necessária para concluir a igualdade desejada, fica como exercício para o leitor.
            \end{enumerate}

        \item
            Exercício.

        \item
            Abaixo, seguem as demonstrações das leis de De Morgan.
            \begin{enumerate}
                \item
                    Inicialmente, demonstremos que $(A \uniao B)^C \contido A^C \inter B^C$. Do item~\ref{prop:uniao-e-intersecao-inclusao} da Proposição~\ref{prop:uniao-e-intersecao}, temos que $A \contido A \uniao B$. Segue do item \ref{prop:complementar:contrapositiva} da Proposição~\ref{prop:complementar} que $(A \uniao B)^C \contido A^C$ (I). De forma análoga, também temos que $(A \uniao B)^C \contido B^C$ (II). Seja agora $x \em (A \uniao B)^C$. Por (I), temos $x \em A^C$. Por (II), temos $x \em B^C$. Logo, $x \pertence A^C \inter B^C$. Portanto, $(A \uniao B)^C \contido A^C \inter B^C$.

                    Finalmente, demonstremos que $A^C \inter B^C \contido (A \uniao B)^C$. Seja $x \em A^C \inter B^C$, ou seja, $x \pertence A^C$ e $x \pertence B^C$. Dessa forma, $x \naopertence A$ e $x \naopertence B$. Suponha, por contradição, que $x \pertence A \uniao B$. Dessa maneira, teríamos $x \pertence A$, o que contradiz $x \naopertence A$, ou $x \pertence B$, o que contradiz $x \naopertence B$. Logo, $x \naopertence A \uniao B$, ou seja, $x \pertence (A \uniao B)^C$. Assim, $A^C \inter B^C \contido (A \uniao B)^C$. Portanto, $(A \uniao B)^C = A^C \inter B^C$.
                \item
                    Primeiramente, demonstremos que $A^C \uniao B^C \contido (A \inter B)^C$. Observe que, pela propriedade~\ref{prop:uniao-e-intersecao-inclusao}, temos $A \inter B \contido A$, e, pelo item \ref{prop:complementar:contrapositiva} da proposição \ref{prop:complementar}, temos $A^C \contido (A \inter B)^C$ (I). De forma análoga, temos $B^C \contido (A \inter B)^C$ (II). Então, seja $x \in A^C \uniao B^C$, isto é, $x \in A^C$ ou $x \in B^C$. Caso que $x \in A^C$, podemos afirmar que $x \in (A \inter B)^C$ por (I). E, caso $x \in B^C $, por (II), também podemos afirmar que $x \in (A \inter B)^C$. Portanto, temos que $A^C \uniao B^C \contido (A \inter B)^C$. \\
                    Finalmente, demonstremos que $(A \inter B)^C \contido A^C \uniao B^C$. Suponha, para chegar num absurdo, que $(A \inter B)^C \not\contido A^C \uniao B^C$. Então, existe $x \in (A \inter B)^C$ tal que $x \naopertence A^C \uniao B^C$. Logo, $x \naopertence A \inter B$. Temos 4 possibilidades para $x$:
                    
                    Caso $x \in A$ e $x \in B$: Então, $x \in A \inter B$, o que contradiz $x \naopertence A \inter B$.

                    Caso $x \naopertence A$ e $x \pertence B$: Então, $x \in A^C$. Daí, $x \in A^C \uniao B^C$, o que contradiz $x \naopertence A^C \uniao B^C$.

                    Caso $x \pertence A$ e $x \naopertence B$: Então, $x \in B^C$. Daí, $x \in A^C \uniao B^C$, o que contradiz $x \naopertence A^C \uniao B^C$.

                    Caso $x \naopertence A$ e $x \naopertence B$: De $x \naopertence A$, temos $x \in A^C$. Daí, $x \in A^C \uniao B^C$, o que contradiz $x \naopertence A^C \uniao B^C$.\\
                    Sendo assim, como os 4 casos levam a um absurdo, podemos concluir que $(A \inter B)^C \contido A^C \uniao B^C$.
            \end{enumerate}
    \end{enumerate}
\end{proof}

\begin{onlineact}
    \khan{https://pt.khanacademy.org/math/statistics-probability/probability-library/basic-set-ops/e/basic_set_notation}{Notação Básica de Conjunto}.
\end{onlineact}
