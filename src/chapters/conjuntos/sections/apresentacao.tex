\section{Apresentação}

Conjuntos estão por toda parte, ainda que, em um primeiro momento, isso não seja tão evidente. Utilizamos seus conceitos mais frequentemente do que imaginamos, mas não costumamos parar para pensar sobre o quão importantes eles são.
Para se ter ideia, praticamente toda a Matemática atual é formulada na linguagem de conjuntos, mesmo sendo a mais simples das ideias matemáticas.

Comecemos com a mais simples das noções. Reflita sobre qual seria a resposta para a pergunta \abreaspas O que é um conjunto?\fechaaspas

\reticencias Refletiu?

Intuitivamente, o que nos vem a cabeça é que conjuntos são uma espécie de coleção, agrupamento -- ou qualquer outro sinônimo -- de objetos.
No entanto, esse é um conceito bastante abstrato; isto é, não há exatamente uma resposta para o questionamento proposto, tendo em vista que ela depende da interpretação de cada um.

No decorrer destas notas de aula, trataremos os conceitos de conjuntos de uma maneira mais palpável para que se tornem mais praticáveis.
Logo, precisaremos ir um pouco mais além para entender melhor e ver o que essa ferramenta tão poderosa pode nos oferecer.
