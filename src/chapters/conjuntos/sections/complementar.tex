\subsection{Complementar}
A noção de complementar de um conjunto só faz sentido quando fixamos um \textdef{conjunto universo}, que denotaremos por $\U$.
Uma vez fixado $\U$, todos os elementos considerados pertencerão a $\U$ e todos os conjuntos serão subconjuntos de $\U$.
Dessa forma, teremos bem esclarecidos quais objetos podem estar presentes nos conjuntos.

\begin{example}
Na geometria plana, $\U$ é o plano onde os elementos são pontos, e todos os conjuntos são constituídos por pontos desse plano.
As retas servem como exemplos desses conjuntos; portanto, são subconjuntos de $\U$ (não elementos!).
\end{example}

\begin{definition}[Complementar]
\label{def:complementar}
Dado um conjunto $A$ (isto é, um subconjunto de $\U$), chama-se \textdef{complementar} de $A$ o conjunto $\compl{A}$ formado pelos elementos de $\U$ que não pertencem a $A$.
Em outras palavras:
$$
\compl{A} = \set{x \tq x \notin A}
$$
\end{definition}

\begin{proposition}
\label{prop:def-complementar}
Tem-se, para todo conjunto $A$, que:
$$
x \not \in \compl{A} \sse x \in A.
$$
\end{proposition}

\begin{proof}
Exercício.
\end{proof}

\begin{example}
Seja $\U$ o conjunto dos triângulos.
Qual o complementar do conjunto dos triângulos escalenos?
\end{example}

\begin{solution}
	Quando classificamos um triângulo pelo comprimento de seus lados, ele é, necessariamente, equilátero, isósceles ou escaleno.
	Portanto, o complementar do conjunto dos triângulos escalenos é o conjunto dos triângulos que são equiláteros ou isósceles.
\end{solution}

\begin{proposition}[Propriedades do complementar]
\label{prop:complementar}
Fixado um conjunto universo $\U$, valem para todos conjuntos $A$ e $B$:
\begin{enumerate}[i)]
	\item
		\label{prop:complementar:universo}
		$\compl{\U}=\emptyset$ e $\compl{\emptyset} = \U$;
	\item
		\label{prop:complementar:complementar-do-complementar}
		$\compl{\prn{\compl{A}}} = A$ (todo conjunto é complementar do seu complementar);
	\item
		\label{prop:complementar:contrapositiva}
		Se $A \subset B$, então $\compl{B} \subset \compl{A}$ (se um conjunto está contido em outro, seu complementar contém o complementar desse outro). 
\end{enumerate}
\end{proposition}

\begin{proof}
\begin{enumerate}[i)]
\item[]
\item
A inclusão $\emptyset \subset \compl{\U}$ é imediata pois $\emptyset$ é subconjunto de qualquer conjunto. 
Provemos, agora, que $\compl{\U} \subset \emptyset$.
Suponha, por absurdo, que $\compl{\U} \notsubset \emptyset$; ou seja, existe $x \in \compl{\U}$ tal que $x \notin \emptyset$.
Ora, se $x \in \compl{\U}$, então $x \notin \U$, o que é um absurdo.
Logo, $\compl{\U} \subset \emptyset$.
Dos fatos de que $\emptyset \subset \compl{\U}$ e $\compl{\U} \subset \emptyset$, conclui-se que $\compl{\U} = \emptyset$.
A demonstração de que $\compl{\emptyset} = \U$ fica como exercício para o leitor.

\item
Provemos, primeiro, que $A \subset \compl{\prn{\compl{A}}}$.
Para tal, seja $x \in A$.
Logo, $x \notin \compl{A}$, pela Proposição \ref{prop:def-complementar}.
Assim, $x \in \compl{\prn{\compl{A}}}$, usando, agora, a própria \complementref{definição de complementar}.
Então, $A \subset \compl{\prn{\compl{A}}}$.
A prova de que $\compl{\prn{\compl{A}}} \subset A$, item restante para podermos concluir que a igualdade desejada é válida, fica como exercício para o leitor.

\item
Sejam $A$, $B$ conjuntos tais que $A \subset B$.
Além disso, seja $x \in \compl{B}$, o que implica em $x \notin B$.
Temos duas possibilidades para a presença de $x$ no conjunto $A$, a saber, $x \in A$ e $x \notin A$.
Se $x \in A$, teríamos $x \in B$ também pois $A \subset B$; um absurdo visto que já temos $x \notin B$.
Logo, concluímos que $x \notin A$, ou seja, $x \in \compl{A}$.
Segue, então, que $\compl{B} \subset \compl{A}$.
\end{enumerate}
\end{proof}