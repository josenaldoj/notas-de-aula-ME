\subsection{Complementar}
A noção de complementar de um conjunto só faz sentido quando fixamos um \emph{conjunto universo}, que denotaremos por $\U$. Uma vez fixado $\universo$, todos os elementos considerados pertencerão a $\U$ e todos os conjuntos serão subconjuntos de $\U$. Dessa forma, teremos bem esclarecidos quais objetos podem estar presentes nos conjuntos.

\begin{example}
	Na geometria plana, $\universo$ é o plano onde os elementos são pontos, e todos os conjuntos são constituídos por pontos desse plano. As retas servem como exemplos desses conjuntos; portanto, são subconjuntos de $\U$ (não elementos!).
\end{example}

\begin{definition}[Complementar]
	\label{def:complementar}
	Dado um conjunto $A$ (isto é, um subconjunto de $\U$), chama-se \emph{complementar} de $A$ o conjunto $\compl A$ formado pelos elementos de $\U$ que não pertencem a $A$.
	Em outras palavras:
	\[
		A\complementar = \conjunto{ x \taisque x \naopertence A}.
	\]
\end{definition}

%\begin{proposition}
%	\label{prop:def-complementar}
%	Tem-se, para todo conjunto $A$, que:
%	\begin{center}
%		$x \naopertence \compl A$ se, e somente se, $x \pertence A$.
%	\end{center}
%\end{proposition}

\begin{proof}
	Exercício.
\end{proof}

\begin{example}
	Seja $\U$ o conjunto dos triângulos. Qual o complementar do conjunto dos triângulos escalenos?
\end{example}

\begin{solution}
	Quando classificamos um triângulo pelo comprimento de seus lados, ele é, necessariamente, equilátero, isósceles ou escaleno. Portanto, o complementar do conjunto dos triângulos escalenos é o conjunto dos triângulos que são equiláteros ou isósceles.
\end{solution}

\begin{proposition}[Propriedades do complementar]
	\label{prop:complementar}
	Fixado um conjunto universo $\U$, valem para todos conjuntos $A$ e $B$:
	\begin{enumerate}
		\item
			\label{prop:complementar:universo}
			$\compl\U = \vazio$ e $\compl\vazio = \U$;
		\item
			\label{prop:complementar:complementar-do-complementar}
			$\prn{A\complementar}\complementar = A$ (todo conjunto é complementar do seu complementar);
		\item
			\label{prop:complementar:contrapositiva}
			Se $A \contido B$, então $\compl{B} \contido \compl{A}$ (se um conjunto está contido em outro, seu complementar contém o complementar desse outro). 
	\end{enumerate}
\end{proposition}

\begin{proof}
	Primeiramente, demonstremos que vale o item~\ref{prop:complementar:universo}. A inclusão $\vazio \contido \compl{\U}$ é imediata pois $\vazio$ é subconjunto de qualquer conjunto.  Provemos, agora, que $\compl{\U} \contido \vazio$. Suponha, por absurdo, que $\compl{\U} \naocontido \vazio$; ou seja, existe $x \em \compl{\U}$ tal que $x \naopertence \vazio$. Ora, se $x \pertence \compl{\U}$, então $x \naopertence \U$, o que é um absurdo. Logo, $\compl{\U} \contido \vazio$. Dos fatos de que $\vazio \contido \compl{\U}$ e $\compl{\U} \contido \vazio$, conclui-se que $\compl{\U} = \vazio$. A demonstração de que $\compl{\vazio} = \U$ fica como exercício para o leitor.

	Em segundo lugar, demonstremos que vale o item~\ref{prop:complementar:complementar-do-complementar}. Para isso, provemos que $A \contido \compl{\prn{\compl{A}}}$ primeiro. Para tal, seja $x \pertence A$. Temos duas possibilidades para a presença de $x$ no conjunto $\compl{A}$, a saber, $x \pertence \compl{A}$ e $x \naopertence \compl{A}$. Se $x \pertence \compl{A}$, segue que $x \naopertence A$, um absurdo, pois $x \pertence A$ por hipótese. Logo, $x \naopertence \compl{A}$. Assim, $x \pertence \compl{\prn{\compl{A}}}$, usando, agora, a própria definição de complementar. Então, $A \contido \compl{\prn{\compl{A}}}$. A prova de que $\compl{\prn{\compl{A}}} \contido A$, item restante para podermos concluir que a igualdade desejada é válida, fica como exercício para o leitor.

	Em decorrência dessa propriedade, vamos tratar como imediato que, para todo conjunto $A$, $x \naopertence \compl A$ se, e somente se, $x \pertence A$.

	Por fim, demonstremos que vale o item~\ref{prop:complementar:contrapositiva}. Para isso, sejam $A$, $B$ conjuntos tais que $A \contido B$. Além disso, seja $x \em \compl{B}$, o que implica em $x \naopertence B$. Temos duas possibilidades para a presença de $x$ no conjunto $A$, a saber, $x \pertence A$ e $x \naopertence A$. Se $x \pertence A$, teríamos $x \em B$ também pois $A \contido B$; um absurdo visto que já temos $x \naopertence B$. Logo, concluímos que $x \naopertence A$, ou seja, $x \pertence \compl{A}$. Segue, então, que $\compl{B} \contido \compl{A}$.
\end{proof}