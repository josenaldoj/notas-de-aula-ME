\section{Conjuntos e Lógica}

Em Matemática, a Teoria de Conjuntos está intimamente relacionada à Lógica. Como evidência disso, existem diversas equivalências entre relações e operadores de conjuntos e conectivos lógicos. Apresentar-se-ão quatro delas, mas antes vamos entender como usamos os símbolos $\implica$ e $\sse$ no nosso curso.

Na Lógica Clássica Proposicional, a implicação $(\implicalogico)$ é um conectivo lógico. Ela constrói, a partir de duas proposições $P$ e $Q$, uma terceira proposição $P \implicalogico Q$.O valor-verdade de $P \implicalogico Q$ é completamente determinado quando se conhecem os valores-verdade de $P$ e $Q$.A relação exata pode ser consultada na Tabela \ref{tbl:implicacao}, denominada tabela-verdade.

\begin{table}[h]
	\centering
	\begin{tabular}{cc|c}
		$P$		& $Q$		& $P \implica Q$ \\ \hline
		V		& V			& V			     \\
		V		& F			& F			     \\
		F		& V			& V			     \\
		F		& F			& V			     \\	
	\end{tabular}
	\caption{Tabela-verdade da implicação.}
	\label{tbl:implicacao}
\end{table}

A implicação aparece textualmente de algumas maneiras; a mais comum delas sendo \abreaspas Se $P$, então $Q$\fechaaspas.
No caso de a afirmação anterior ser, por exemplo, um problema, interpreta-se o $P$ como a sua hipótese, e $Q$, como a sua tese. Costuma-se escrever, também, nessa situação, $P \implica Q$ e reservar o $P \implicalogico Q$ para contextos puramente lógicos.

Para provar que uma implicação é verdadeira, o primeiro passo é considerar a hipótese $P$ como uma propriedade verdadeira.
Consequentemente, nossa atenção se restringe somente às duas primeiras linhas da tabela-verdade de $P \implica Q$.
Assim, pela tabela, constatamos que, para alcançarmos nosso objetivo de provar que a afirmação \entreaspas{Se $P$, então $Q$} é verdadeira, precisamos mostrar que a tese $Q$ também é verdadeira.

Quando já está provado que \entreaspas{Se $P$, então $Q$} é verdadeira, utilizamos esse resultado para concluir que $Q$ é verdadeira quando temos, também, que $P$ é verdadeira. Isso pode, mais uma vez, ser verificado pela tabela-verdade de $P \implica Q$: na única linha em que $P \implica Q$ e $P$ são verdadeiras, $Q$ também é.

Outro conectivo lógico é o $\sselogico$, chamado de bicondicional ou bi-implicação.
Da mesma forma que a implicação -- e os demais conectivos lógicos --, o valor-verdade da expressão $P \sselogico Q$, dadas duas proposições $P$ e $Q$, é totalmente determinado pelos valores-verdade dessas proposições.
A relação se encontra na Tabela \ref{tbl:bi-implicacao}.

\begin{table}[h]
	\centering
	\begin{tabular}{cc|c}
		$P$		& $Q$		& $P \sselogico Q$	\\ \hline
		V		& V			& V			\\
		V		& F			& F			\\
		F		& V			& F			\\
		F		& F			& V			\\	
	\end{tabular}
	\caption{Tabela-verdade da bi-implicação.}
	\label{tbl:bi-implicacao}
\end{table}

Pela tabela, pode-se notar que a expressão $P \sselogico Q$ é verdadeira quando $P$ e $Q$ têm o mesmo valor-verdade, e apenas nessa situação.

De forma similar ao que ocorre na implicação, reserva-se o símbolo $\sselogico$ para contextos lógicos, utilizando o \entreaspas{$\sse$} apenas em cenários menos formais. Textualmente, uma afirmação do tipo $P \sse Q$ costuma se manifestar como ``$P$ se, e somente se, $Q$''. Isso começa a evidenciar um importante fato sobre a bi-implicação: ela pode ser definida como a conjunção de duas implicações. A frase ``$P$ se $Q$'' é equivalente a ``se $Q$, então $P$'', o que, conforme já visto, pode ser representado como $Q \implica P$. Ademais, ``$P$ somente se $Q$'' diz que $P$ só pode ser verdade se $Q$ também for, ou seja, $P \implica Q$. As frases, quando juntas, expressam a noção de bi-condicionalidade.

Outra versão textual comum do $\sse$ é a sentença ``Para $P$, é necessário e suficiente que $Q$''. Assim como no caso do ``se, e somente se'', essa expressão carrega duas informações consigo. A frase ``Para $P$, é necessário que $Q$'' significa que a validade de $P$ só é possível na presença da validade de $Q$, isto é, $P \implica Q$. Por outro lado, ``Para $P$, é suficiente que $Q$'' indica que basta $Q$ ser verdade para $P$ também ser, ou seja, $Q \implica P$.

Uma terceira forma de se apresentar a informação $P \sse Q$ é dizer que ``$P$ é equivalente a $Q$''. Pode-se justificar essa construção pela tabela-verdade de $P \sselogico Q$. Conforme mencionado, $P \sselogico Q$ é verdade precisamente quando $P$ e $Q$ tem o mesmo valor-verdade. Em caso positivo, não existe uma diferença prática entre $P$ e $Q$. Assim, usa-se o termo ``equivalência'', que também indica que, quando se tem uma dessas proposições, se tem a outra, já que elas possuem o mesmo valor-verdade.

Em todas as formas textuais vistas, o $\sse$ pôde ser desmembrado em dois $\implica$. Isso reflete diretamente na maneira como se prova afirmações do tipo $P \sse Q$. O método consiste em demonstrar, de forma independente, que $P \implica Q$ e que $Q \implica P$. Aqui, vale a discussão acerca de como provar afirmações com $\implica$. A diferença é que sempre se fazem necessárias duas provas desse tipo.

Explicados os usos do $\implica$  e do $\sse$ neste texto, podemos voltar a atenção à equivalência entre operadores e relações de conjuntos e conectivos lógicos. No restante desta seção, considere $P$ e $Q$ propriedades aplicáveis aos elementos de $\U$. Considere, também, $A = \conjunto{ x \tq \text{$x$ satisfaz $P$}}$ e $B = \conjunto{ x \tq \text{$x$ satisfaz $Q$}}$.

\begin{proposition}[Inclusão e implicação] 
	$A \contido B$ é equivalente a $P \implica Q$.
\end{proposition}

\begin{proposition}[Igualdade e bi-implicação] 
	$A = B$ é equivalente a $P \sse Q$.
\end{proposition}

\begin{example}
	Analise as implicações abaixo:
	\begin{align*}
        x^2+1=0 &\implica (x^2+1)(x^2-1) = 0 \vezes (x^2-1) \\
                &\implica x^4 - 1 = 0 \\
                &\implica x^4 = 1 \\
                &\implica x \pertence \conjunto {-1, 1}
	\end{align*}
	Isso quer dizer que o conjunto solução de $x^2 +1 = 0$ é $\conjunto{-1, 1}$?
\end{example}

\begin{solution}
	O conjunto-solução de $x^2 + 1 = 0$ é $S = \conjunto{ x \nos \reais \taisque x^2 + 1 = 0} = \vazio$, o que implica que $S \diferente \conjunto{-1, 1}$.
\end{solution}

\begin{proposition}[Complementar e negação] 
	$A^C$ é equivalente a $\naologico P$.
\end{proposition}

Podemos combinar os itens \ref{prop:complementar:complementar-do-complementar} e \ref{prop:complementar:contrapositiva} da Proposição \ref{prop:complementar} e obter que:
\begin{center}
    $P \implica Q$ se, e somente se, $\naologico Q \implica \naologico P$.
\end{center}

Chamamos $Q \implica P$ de \emph{recíproca} de $P \implica Q$, e $P \elogico \naologico Q$ de \emph{negação} de $P \implica Q$. É dado um exemplo no Exercício \ref{exe:escrever-reciprocas}.

\begin{example}
	Observe as afirmações:
	
	\begin{itemize}
		\item Todo número primo maior do que 2 é ímpar;
		\item Todo número par maior do que 2 é composto.
	\end{itemize}

	Essas afirmações dizem exatamente a mesma coisa, ou seja, exprimem a mesma ideia; só que com diferentes termos. Podemos reescrevê-las na forma de implicações vendo claramente que uma é contrapositiva da outra, e todas estão sob a hipótese de que $n \in \N$, com $n > 2$:    
    \begin{align*}
        \text{$n$ primo}                   &\implica \text{$n$ ímpar} \\
        \text{$\naologico$( $n$ ímpar )}   &\implica \text{$\naologico$( $n$ primo )} \\
        \text{$n$ par}                     &\implica \text{$n$ composto}
    \end{align*}
\end{example}

\begin{proposition}[União e disjunção]
	$A \uniao B$ é equivalente a $P \lor Q$ ($P \ou Q$).
\end{proposition}

\begin{proposition}[Interseção e conjunção]
	$A \inter B$ é equivalente a $P \elogico Q$ ($P \e Q$).
\end{proposition}

\begin{remark}
    O conectivo lógico \textit{ou} tem significado diferente do usado normalmente no português. Na linguagem coloquial, usamos $P$ \textit{ou} $Q$ sem permitir que sejam as duas coisas ao mesmo tempo. Analise a seguinte história:

    \emph{Um obstetra que também era matemático acabara de realizar um parto quando o pai perguntou: ``É menino ou menina, doutor?''. E ele respondeu: ``sim''.}
\end{remark}

As equivalências entre as relações e os operadores da Teoria dos Conjuntos e conectivos da Lógica são resumidas na Tabela~\ref{tbl:equiv-conj-logc}.
\begin{table}[h]
	\centering
    \begin{tabular}{|c|c|}
        \hline
        Operação/relação em Conjuntos & Fórmula de Lógica \\ \hline
        $A = B$                 	  & $P \sse Q$        \\ \hline
        $A \contido B$            	  & $P \implica Q$    \\ \hline
        $A\complementar$        	  & $\naologico P$    \\ \hline
        $A \uniao B$              	  & $P \lor Q$        \\ \hline
        $A \inter B$              	  & $P \elogico Q$    \\
        \hline
    \end{tabular}
	\caption{Equivalências entre as relações e operadores de conjuntos e conectivos lógicos.}
	\label{tbl:equiv-conj-logc}
\end{table}

\caixote{Um problema de lógica}{
	A polícia prende quatro homens, um dos quais cometeu um furto. Eles fazem as seguintes declarações:
	\begin{itemize}
		\item Arnaldo: Bernaldo fez o furto.
		\item Bernaldo: Cernaldo fez o furto.
		\item Dernaldo: eu não fiz o furto.
		\item Cernaldo: Bernaldo mente ao dizer que eu fiz o furto.
	\end{itemize}
	Se sabemos que só uma destas declarações é a verdadeira, quem é culpado pelo furto?
}