\section{Exercícios}

\begin{exercise}
\label{exe:vazio-notacao}
	De que outras formas podemos representar o \emptysetref{conjunto vazio} utilizando as duas notações de definição de conjuntos que conhecemos?
\end{exercise}


\begin{exercise}
\label{exe:vazios-tricky}
	Decida quais das afirmações a seguir estão corretas. Justifique suas respostas.
	\begin{enumerate}[a.]
		\item $\emptyset \in \emptyset$;
		\item $\emptyset \subset \emptyset$;
		\item $\emptyset \in \set{\emptyset}$;
		\item $\emptyset \subset \set{\emptyset}$.
	\end{enumerate}
\end{exercise}

\begin{exercise}
%\label{exe:} @TODO linkar com as propriedades
Complete as demonstrações em \nameref{prop:uniao-e-intersecao} que não foram feitas em sala de aula.
\end{exercise}

\begin{exercise}
%\label{exe:}
Demonstre que os seguintes itens são equivalentes:
	\begin{enumerate}[a.]
		\item $A \uniao B = B$;
		\item $A \subset B$;
		\item $A \inter B = A$;
	\end{enumerate}
	\begin{hint}
		Para tanto, é preciso provar \textbf{a. $\iff$ b.} e \textbf{b. $\iff$ c.}.
		Outra maneira é provar \textbf{a. $\implies$ b.}, \textbf{b. $\implies$ c.} e por fim, \textbf{c. $\implies$ a.}.
	\end{hint}
\end{exercise}

\begin{exercise}
O diagrama de Venn para os conjuntos $X$, $Y$, $Z$ decompõe o
plano em oito regiões. Numere essas regiões e exprima cada um dos
conjuntos abaixo como reunião de algumas dessas regiões. (Por
exemplo: $X \inter Y = 1 \uniao 2$.)
\begin{enumerate}[a.]
  \item $\left(X^C \uniao Y \right)^C$;
  \item $\left(X^C \uniao Y \right) \uniao Z^C$;
  \item $\left(X^C \inter Y \right) \uniao \left(X \inter Z^C \right)$;
  \item $\left(X \uniao Y \right)^C \inter Z$.
\end{enumerate}
\end{exercise}

\begin{exercise}
Exprimindo cada membro como reunião de regiões numeradas, prove
as igualdades:
\begin{enumerate}[a.]
  \item $\left(X \uniao Y \right)\inter Z = \left(X \inter Z \right) \uniao \left(Y \inter Z
  \right)$;
  \item $X \uniao \left(Y \inter Z \right)^C = X \uniao Y^C \uniao Z^C$.
\end{enumerate}
\end{exercise}

\begin{exercise}
Sejam $A$, $B$ e $C$ conjuntos. Determine uma condição necessária e
suficiente para que se tenha 
	$$A \uniao \left( B \inter C \right) = \left(A \uniao B \right) \inter C$$
\end{exercise}

\begin{exercise}
Considere $A$, $A^\prime$, $B$ e $B^\prime$ conjuntos tais que $A  \subset A^\prime$ e 
$B \subset B^\prime$. Prove que, se $A^\prime \inter B^\prime=\emptyset$, então
$A\inter B = \emptyset$.  
\end{exercise}

\begin{exercise}
Recorde a definição da diferença entre conjuntos:
  $$B \setminus A = \set {x \tq x \in B \text { e } x \notin A}.$$
  Mostre que
    \begin{enumerate}[a.]
      \item $B \setminus A = B \inter A^C$;
      \item $(B \setminus A)\uniao A = B \uniao A$;
      \item $\prn{B \setminus A}^C = B^C\uniao A$;
      \item $B \setminus A = \emptyset$ se, e somente se, $B \subset
      A$;
      \item $B \setminus A = B$ se, e somente se, $A \inter B =
      \emptyset$;
      \item Vale a igualdade $B \setminus A = A \setminus B$ se, e
      somente se, $A = B$;
      \item Determine uma condição necessária e suficiente para que
      se tenha $$A \setminus \left(B \setminus C \right) = \left(A
      \setminus B \right) \setminus C.$$
      %\sub{Dica}: Use o diagrama de Venn para enxergar tal condição
      %necessária e suficiente antes de demonstrar a igualdade.
    \end{enumerate}
\end{exercise}

\begin{exercise}
Dê exemplos de implicações, envolvendo conteúdos de ensino
  médio, que sejam: verdadeiras com recíproca verdadeira;
  verdadeiras com recíproca falsa; falsas, com recíproca verdadeira;
  falsas, com recíproca falsa.
\end{exercise}

\begin{exercise}
Considere $P$, $Q$ e $R$ condições aplicáveis aos elementos
de um conjunto universo $\U$, e $A$, $B$ e $C$ os subconjuntos de
$\U$ dos elementos que satisfazem $P$, $Q$ e $R$, respectivamente.
Expresse, em termos de implicações entre $P$, $Q$ e $R$, as
seguintes relações entre os conjuntos $A$, $B$ e $C$.
\begin{enumerate}[a.]
\item $A \inter B^C \subset C$;
\item $A^C \uniao B^C \subset C$;
\item $A^C \uniao B \subset C^C$;
\item $A^C \subset B^C \uniao C$;
\item $A \subset B^C \uniao C^C$.
\end{enumerate}
\end{exercise}

\begin{exercise}
Considere as seguintes (aparentes) equivalências lógicas:
\begin{align*}
x=1 & \iff x^2 -2x +1 = 0 \\
& \iff x^2 -2 \cdot 1 +1 =0 \\
& \iff x^2 - 1 =0 \\
& \iff x = \pm 1
\end{align*}
Conclusão (?): $x=1 \iff x= \pm 1$. Onde está o erro?
\end{exercise}

\begin{exercise}
\label{exe:escrever-reciprocas}
Escreva as recíprocas, contrapositivas e negações
matemáticas das seguintes afirmações:
\begin{enumerate}[a.]
  \item Todos os gatos têm rabo; $\left(G \implies R \right)$\\
  \sub{Recíproca:} Se têm rabo então é gato; $\left(R \implies G \right)$\\
  \sub{Contrapositiva:} Se não tem rabo então não é gato; $\left(\sim R \implies \sim G \right)$\\
  \sub{Negação:} Existe um gato que não tem rabo. $\left(G \land \sim R \right)$
  \item Sempre que chove, eu saio de guarda-chuva ou fico em casa;
  \item Todas as bolas de ping pong são redondas e brancas;
  \item Sempre que é terça-feira e o dia do mês é um número primo,
  eu vou ao cinema;
  \item Todas as camisas amarelas ou vermelhas têm manga comprida;
  \item Todas as coisas quadradas ou redondas são amarelas e
  vermelhas.
\end{enumerate}
\end{exercise}

\begin{exercise}
Considere os conjuntos: $F$ composto por todos os filósofos;
$M$ por todos os matemáticos; $C$ por todos os cientistas; e $P$ por
todos os professores.
\begin{enumerate}[a.]
  \item Exprima cada uma das afirmativas abaixo usando a linguagem
  de conjuntos: \\
  (i) Todos os matemáticos são cientistas; (ii) Alguns matemáticos
  são professores; (iii) Alguns cientistas são filósofos; (iv) Todos
  os filósofos são cientistas ou professores; (v) Nem todo professor
  é cientista.
  \item Faça o mesmo com as afirmativas abaixo: \\
  (vi) Alguns matemáticos são filósofos; (vii) Nem todo filósofo é
  cientista; (viii) Alguns filósofos são professores; (ix) Se um
  filósofo não é matemático, ele é professor; (x) Alguns filósofos
  são matemáticos.
  \item Tomando as cinco primeiras afirmativas como hipóteses,
  verifique quais das afirmativas do segundo grupo são
  necessariamente verdadeiras.
\end{enumerate}
\end{exercise}

\begin{exercise}
Considere um grupo de 4 cartões, que possuem uma letra escrita
em um dos lados e um número do outro. Suponha que seja feita, sobre
esses cartões, a seguinte afirmação: \emph{Todo cartão com uma vogal
de um lado tem um número ímpar do outro}. Quais dos cartões abaixo
você precisaria virar para verificar se essa afirmativa é verdadeira
ou falsa?
\begin{center}
\begin{tabular}{|c|c|c|c|c|c|c|}
  \cline{1-1} \cline{3-3} \cline{5-5} \cline{7-7}
  % after \\: \hline or \cline{col1-col2} \cline{col3-col4} ...
  A & $\empty$ & 1 & $\empty$ & B & $\empty$ & 4 \\
  \cline{1-1} \cline{3-3} \cline{5-5} \cline{7-7}
\end{tabular}
\end{center}
\end{exercise}

\begin{exercise}
  Numa mesa há cinco cartas:
  %
  \begin{center}
  \begin{tabular}{|c|c|c|c|c|c|c|c|c|}
    \cline{1-1} \cline{3-3} \cline{5-5} \cline{7-7} \cline{9-9}
    Q & $\empty$ & T & $\empty$ & 3 & $\empty$ & 4 & $\empty$ & 6 \\
    \cline{1-1} \cline{3-3} \cline{5-5} \cline{7-7} \cline{9-9}
  \end{tabular}
  \end{center}
  %
  Cada carta tem um número natural de um lado e uma letra de outro lado. 
  Nico afirma: ``Qualquer carta que tenha uma vogal tem um número par do outro lado''.
  Jorel provou que Nico mente virando somente uma das cartas. Qual das cinco cartas
  Jorel teve que virar para provar que Nico mentiu?
\end{exercise}