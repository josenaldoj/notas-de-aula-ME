\section{Pertinência}

Na Seção \ref{sec:intro}, entendemos o que é um conjunto e como podemos descrevê-lo.
Agora, teremos o poder de relacionar objetos a conjuntos perguntando se um dado objeto está em um conjunto.
Como já falamos sobre, só temos uma resposta possível dentre sim e não. O que nos permite a seguinte definição:
\begin{definition}[Relação de Pertinência]
    \label{def:pertinencia}
    Dados um objeto $x$ e um conjunto $A$, se for o caso de $x$ ser um elemento de $A$, dizemos que $x$ \emph{pertence} a $A$. Para denotar esse fato, escrevemos $x \pertence A$.

    \label{def:naopertinencia}
    Quando $x$ não é um elemento de $A$ dizemos que $x$ \emph{não pertence} a $A$, o que denotamos por $x \naopertence A$.
\end{definition}

Não é correto dizer que $x$ está \entreaspas{contido} ou é \entreaspas{parte} de $A$, esses termos são usados em matemática com outro sentido, como veremos na Definição~\ref{def:inclusao}.

\begin{example}
    Considere $P$ e $V$ conforme definido nos Exemplos~\ref{ex-vogais} e~\ref{ex-primos-pares}, respectivamente. Sendo assim, temos que $\texttt e \pertence V$ e $3 \naopertence P$.
\end{example}

\begin{example}
    Considere $L$ como sendo o conjunto de todas as letras do alfabeto latino. Sabendo que as vogais do conjunto $V$ fazem parte desse alfabeto, podemos concluir que $\texttt a, \texttt e, \texttt i , \texttt o , \texttt u  \pertence L$. Também sabemos que a letra $\texttt j$ faz parte do alfabeto, isto é, $\texttt j \pertence L$. Contudo, como $\texttt j$ não é uma vogal, podemos concluir que $\texttt j \naopertence V$.
\end{example}

Ainda na Seção~\ref{sec:intro}, vimos que dois conjuntos iguais $A$ e $B$ possuem, necessariamente, os mesmos elementos, o que implica que são diferentes quando existe pelo menos um elemento que pertence a apenas um dos dois conjuntos. Com a notação de pertinência, isto é o mesmo que dizer que existe $x \em A$ tal que $x \naopertence B$ ou que existe $x \em B$ tal que $x \naopertence A$.

\begin{example}
    Os conjuntos $\N$ (números naturais) e $\Z$ (números inteiros), abordados com mais detalhes no Capítulo \ref{cap:conjuntos-numericos}, são diferentes. Note que $-1 \pertence \Z$ mas $-1 \naopertence \N$, por exemplo, o que prova que esses conjuntos são diferentes.
\end{example}

As noções de conjuntos e elementos não são mutuamente exclusivas como, por exemplo, a paridade nos números naturais. Isso quer dizer que, dependendo do contexto, um conjunto pode ser um elemento de outro conjunto, como veremos nos exemplos a seguir.

\begin{example}
    \label{exe:conjuntos-de-conjuntos-explicito}
    Considere o conjunto $A = \conjunto{\conjunto{1, 2},\unitario{2}, 1}$. Observe que existem elementos em $A$ que são conjuntos. Note que:
    \begin{enumerate}
        \item $\conjunto{1, 2} \pertence A$
        \item $\unitario{2} \pertence A$
        \item $1 \pertence A$
        \item $2 \naopertence A$
    \end{enumerate}
\end{example}

\begin{example}
    \label{exe:conjuntos-de-conjuntos-implicito}
    Os habitantes de um país podem ser vistos como um conjunto de pessoas. Por sua vez, cada pessoa pode ser vista como um conjunto de células. Mais tarde, na Definição~\ref{def:powerset}, veremos um famoso conjunto formado por conjuntos.
\end{example}

Para dizer se um dado objeto deve receber o título de \entreaspas{elemento}, é preciso olhar para o conjunto com o qual ele está se relacionando. No caso do Exemplo~\ref{exe:conjuntos-de-conjuntos-explicito}, o conjunto $\conjunto{1, 2}$ é elemento do conjunto $A$, mas quando o interesse se volta à sua relação com o número $2$, não faz sentido se referir a $\conjunto{1, 2}$ como \entreaspas{elemento}.

A Definição~\ref{def:emptyset} irá introduzir outro importante tipo de conjunto.

\begin{definition}[O Conjunto Vazio]
    \label{def:emptyset} % trocar label para def:vazio
    O conjunto que não possui elementos é chamado de \emph{conjunto vazio} e é representado por $\vazio$. Então, para qualquer que seja o objeto $x$, temos que $x \naopertence \vazio$.
\end{definition}

Agora, com a Definição~\ref{def:emptyset}, o Exercício~\homeworkref{exe:vazio-notacao} já pode ser feito.

\begin{example}
    Quais outros conjuntos você conhece? Que tal pensar sobre o conjunto $A = \conjunto{ x \talque x \naopertence A}$?
\end{example}

Esta \entreaspas{definição} de conjunto não faz sentido, resultando num paradoxo, apesar de muito importante para o desenvolvimento da teoria de conjuntos. Sua descoberta é atribuída ao matemático britâncio Bertrand Russel (1872--1970) e é conhecido por \emph{Paradoxo de Russel} ou \emph{Paradoxo do barbeiro}, que você pode rever no vídeo \href{https://youtu.be/UI1xR_AECrU}{\texttt{Este vídeo está mentindo}}.