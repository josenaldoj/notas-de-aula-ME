\subsubsection{Propriedades da Função Tangente}

\begin{proposition}
    Valem as seguintes propriedades acerca da função tangente:

\begin{itemize}
  \item Embora não seja definida para todo número real, a função
  tangente pode ser considerada uma função periódica de período
  $\pi$ em todo o seu domínio, pois $\tan \paren{x+\pi} = \tan x$;
  \item Para todo par de pontos  $(x_1, y_1)$ e $(x_2, y_2)$ em uma reta não vertical, com $x_1 \neq x_2$, se
  $\alpha$ é o ângulo formado pela reta e o eixo $x$, então $$\tan
  \alpha = \frac {y_2 - y_1} {x_2 - x_1}.$$
  \item Ao definirmos $tan: \paren{- \frac {\pi} 2, \frac {\pi} 2} \to \reais$,
obtemos uma bijeção. Assim, o intervalo aberto $\paren{- \frac {\pi}
2, \frac {\pi} 2}$ tem a mesma cardinalidade que $\reais$.
\end{itemize}
\end{proposition}

\begin{proof}
  \item %TODO: decide what will happen with this (trello)
  \item Considere a Imagem . %TODO: update ref %TODO: better text about the image (include the m factor)
  %
  % TODO: insert image here
  %
  No triângulo retângulo mostrado na imagem, temos:
  %
  \begin{align*}
    \tan \alpha &= \frac {\sen\alpha}{\cos\alpha} \\ 
    &= \frac{ \frac{\text{cateto oposto}}{\text{hipotenusa}}   }{\frac{\text{cateto adjacente}}{\text{hipotenusa}}} \\
    &= \frac{\text{cateto oposto}}{\text{cateto adjacente}}\\
    &= \frac{y_2 - y_1}{x_2 - x_1} \\ &= m
  \end{align*}
  \item %TODO: decide what will happen with this (trello)
\end{proof}