\subsubsection{Funções Tangente, Cotangente, Secante e Cossecante}

\begin{definition}
\label{def:outras-funcoes-trigonometricas}
    Definem-se, através das funções seno e cosseno, as funções
trigonométricas com as seguintes leis de formação:
\begin{itemize}
  \item $\tan x = \frac {\sen x}{\cos x}$, \emph{tangente};
  \item $\cot x = \frac {\cos x}{\sen x}$, \emph{cotangente};
  \item $\sec x = \frac {1}{\cos x}$, \emph{secante};
  \item $\csc x = \frac {1}{\sen x}$, \emph{cossecante}.
\end{itemize}
\end{definition}

\begin{remark}
    Os domínios das funções introduzidas na Definição~\ref{def:outras-funcoes-trigonometricas} não contêm o conjunto dos valores de $x$
que zeram seus respectivos denominadores. Assim, temos, por exemplo, que o maior subconjunto dos reais no qual podemos definir
as funções tangente e secante é $$\bigcup_{k \in \Z} \paren{k \pi -
\frac {\pi} 2, k \pi + \frac {\pi} 2}.$$
\end{remark}

\begin{tve}
    \link{https://drive.google.com/file/d/1s8mcd4VT2FF1kRKK7fmKUn8TvczowO6s/view?usp=sharing}{Funções trigonométricas - parte 1}
\end{tve}

\begin{tve}
    \link{https://drive.google.com/file/d/1s92aj3NvUDDrFm8vBuGvhcmIWK9W214Z/view?usp=sharing}{Funções trigonométricas - parte 2}
\end{tve}