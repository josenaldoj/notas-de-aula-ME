\section{Leis dos Cossenos e dos Senos}

\begin{theorem}[Lei dos Cossenos]
    Seja $ABC$ um triângulo com $a = \seg BC$, $b = \seg AC$ e $c = \seg
AB$. Então
$$b^2 = a^2 + c^2 - 2 ac \cdot \cos \widehat B.$$
\end{theorem}

\begin{remark}
    A Lei dos Cossenos é uma generalização do Teorema de Pitágoras. Note
que, se $\widehat B$ é um ângulo reto, então $\cos \widehat B = 0$ e
$b$ será a hipotenusa do triângulo.
\end{remark}

\begin{theorem}[Lei dos Senos]
    Seja $ABC$ um triângulo com $a = \seg BC$, $b = \seg AC$ e $c = \seg
AB$. Então
$$\frac a {\sen \widehat A} = \frac b {\sen \widehat B} = \frac c {\sen \widehat C}$$
\end{theorem}

As leis dos cossenos e dos senos permitem obter os seis elementos de
um triângulo quando são dados três deles, desde que um seja lado,
conforme os casos clássicos de congruência de triângulos.

\begin{onlineact}
    \khan{https://pt.khanacademy.org/math/trigonometry/trig-with-general-triangles/solving-general-triangles/e/law-of-sines-and-cosines-word-problems}
    {Problemas com Triângulos Gerais}.
\end{onlineact}