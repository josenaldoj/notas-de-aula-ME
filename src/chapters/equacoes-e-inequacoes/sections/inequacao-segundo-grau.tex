\section{Inequação do 2° grau}

\begin{definition}
    Uma \textdef{inequação do segundo grau} é uma relação de uma das formas
    a seguir:
    \begin{align*}
        ax^2 + bx + c &< 0;   \\
        ax^2 + bx + c &> 0;   \\
        ax^2 + bx + c &\le 0; \\
        ax^2 + bx + c &\ge 0;
    \end{align*}
    onde $a, b, c \nos \reais$, com $a \diferente 0$.
\end{definition}

\begin{example}
    Resolva as seguintes inequações:
    \begin{enumerate}[a)]
        \item $x^2 -3x +2 > 0$;
        \item $x^2 -3x +2 \le 0$.
    \end{enumerate}
\end{example}

\begin{solution}
    Observe que: 
    \[
        x^2 - 3x + 2 = (x-2)(x-1) > 0.
    \]
    Logo, teremos:
    \begin{figure}[H]
        \centering
        \importtikz{inequacao-2grau-exemplo1}
        \caption{}
    \end{figure}

    Assim, $S = \conjunto{ x \nos \reais \taisque x < 1 \ou x > 2 }$. Ademais, a solução $S\linha$ para a inequação $x^2 - 3x + 2 \menorigual 0$ pode ser obtida da seguinte forma:
    \begin{align*}
        S\linha &= S\complementar \\ 
                &= \conjunto{ x \nos \reais \taisque x < 1 \ou x > 2 }\complementar \\ 
                &= \conjunto{ x \nos \reais \taisque x \maiorigual 1 \e x \menorigual 2 } \\ 
                &= \conjunto{ x \nos \reais \taisque 1 \menorigual x \menorigual 2 }.
    \end{align*}
\end{solution}

\begin{example}
    Prove que a soma de um número positivo com seu inverso multiplicativo é sempre maior ou igual a 2.
\end{example}

\begin{proof}
    Queremos demonstrar que $x + \dfrac 1 x \maiorigual 2$, para todo $x > 0$. De fato, seja $x \nos \reais$. Observe que:
    \[
        x + \dfrac 1 x \maiorigual 2 \sse x + \dfrac 1 x - 2 \maiorigual 0.
    \]
    Sendo assim, basta demonstrar que $x + \dfrac 1 x - 2 \maiorigual 0$. Calculando o termo do lado esquerdo obtemos:
    \begin{align*}
        x + \dfrac 1 x - 2 &= \dfrac {x \vezes x} x + \dfrac 1 x -  \dfrac {2x} x \\
                           &= \dfrac {x^2 - 2x + 1} x \\
                           &= \dfrac {(x - 1)^2} x.
    \end{align*}

    Com isso, podemos realizar o estudo de sinal da função $\dfrac {(x - 1)^2} x$ da seguinte forma:
    \begin{figure}[H]
        \centering
        \importtikz{inequacao-2grau-exemplo2}
        \caption{}
    \end{figure}

    Portanto, para que $x + \dfrac 1 x \maiorigual 2$, é necessário e suficiente que $x > 0$ como queríamos demonstrar.
\end{proof}