\section{Introdução}
\label{eq:intro}
Como você responderia se te perguntassem: Qual o número cujo dobro somado com sua quinta parte é igual a 121? 

Uma outra charada matemática popular segue passos como os descritos abaixo:

\begin{enumerate}[label=\textbf{\arabic*}.]
  \item Escolha um número;
  \item Multiplique esse número por 6;
  \item Some 12;
  \item Divida por 3;
  \item Subtraia o dobro do número que você escolheu; 
  \item O resultado é igual a 4.
\end{enumerate}

Esses dois exemplos tem em comum o fato de que há um número desconhecido em ambos. No primeiro exemplo, esse número é a resposta para a pergunta. E, provavelmente, apenas um número irá satisfazer as condições necessárias para ser a resposta correta. Já no segundo exemplo esse número é o escolhido pelo leitor/interlocutor. E como esse número vai mudar de acordo com a pessoa que estiver lendo esse texto, é importante que qualquer número se encaixe nas condições necessárias para que a charada funcione. 

Equações são ferramentas básicas que possibilitam a tradução de problemas como esses para linguagem matemática. Consistem de igualdades onde há uma ou mais incógnitas, que por sua vez são valores desconhecidos. 

Resolver uma equação consiste em encontrar os valores das incógnitas que tornam a igualdade verdadeira. Devemos, então, responder as perguntas: "Quais são os números que satisfazem essa igualdade? Quais valores tornam a igualdade verdadeira?"

Semelhantemente, uma inequação é uma expressão matemática que envolve uma desigualdade (maior do que, menor do que) e pelo menos uma incógnita. Resolver uma inequação é o equivalente a encontrar os valores das icógnitas que tornam a desigualdade verdadeira. 

Ao longo deste capítulo vamos discorrer sobre como encontrar uma equação ou inequação que descreva uma situação-problema e como encontrar os valores das incógnitas que tornam essas expressões verdadeiras. 

