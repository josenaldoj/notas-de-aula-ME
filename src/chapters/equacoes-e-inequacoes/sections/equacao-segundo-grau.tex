\section{Equação do 2° grau}

\begin{definition}
	\label{def:eq-2-grau}
	A \emph{equação do segundo grau} com coeficientes $a$, $b$ e $c$ é uma equação da forma 
	\[
		ax^2 + bx + c = 0,
	\]
	onde $a, b, c \nos \reais$, $a \diferente 0$ e $x$ é a incógnita cujo valor real deve ser determinado.
\end{definition}

\begin{example}
	\label{ex:sol-2-grau}
	Encontre as soluções de uma equação do segundo grau.
\end{example}

\begin{solution}
	Da Definição \ref{def:eq-2-grau}, sabemos que uma equação do segundo grau tem a seguinte forma:
	\[
		ax^2 + bx + c = 0,
	\]
	onde $a,b,c \in \R$, com $a \ne 0$. Manipulemos a equação para encontrar o valor de $x$:
	\begin{align*}
		ax^2 + bx + c = 0 & \sse x^2+\frac{b}{a}x +\frac{c}{a}=0 \\ 
						  & \sse x^2+2\frac{b}{2a}x+\prn{\frac{b}{2a}}^2-\prn{\frac{b}{2a}}^2+\frac{c}{a}=0\\ 
						  & \sse \prn{x+\frac{b}{2a}}^2=\frac{b^2}{4a^2}-\frac{c}{a}=\frac{b^2 -4ac}{4a^2}\\ 
						  & \sse \abs{x+\frac{b}{2a}} = \sqrt{\frac{b^2-4ac}{4a^2}}\\ 
						  & \sse x+\frac{b}{2a}=\pm\sqrt{\frac{b^2-4ac}{4a^2}}=\frac{\pm\sqrt{b^2-4ac}}{\pm2a}=\pm\frac{\sqrt{b^2-4ac}}{2a}\\ 
						  & \sse x=\frac{-b\pm\sqrt{b^2-4ac}}{2a}\\ 
	\end{align*}
	Portanto, quando $b^2-4ac \ge 0$, o conjunto-solução $S$ da equação será:
	\begin{align*}
		S = \conjunto{\dfrac {-b - \raiz{b^2 - 4ac}} {2a}, \dfrac{-b + \raiz{b^2 - 4ac}} {2a}}
	\end{align*}
\end{solution}

\begin{remark}
	As soluções de uma equação também são chamadas de \emph{raízes da equação}.
\end{remark}

\begin{tve}
	\link{https://drive.google.com/file/d/1jlhxAGjMz9ctCqoe7jaNW9EOd7KpJj2e/view?usp=sharing}{Fórmula de Bhaskara}
\end{tve}

\begin{onlineact}
	\khan{https://pt.khanacademy.org/math/algebra/quadratics/quadratics-square-root/e/quadratics-by-taking-square-roots-with-steps}{Equações do segundo grau com cálculo de raízes quadradas: com etapas}.
\end{onlineact}

\begin{onlineact}
	\khan{https://pt.khanacademy.org/math/algebra/quadratics/solving-quadratics-by-completing-the-square/e/completing_the_square_2}{Método de completar quadrados}.
\end{onlineact}

\begin{definition}
	Chamamos de \emph{discriminante} da equação do segundo grau a expressão $b^2 - 4ac$ e denotamos pela letra grega maiúscula $\Delta$ (lê-se delta).
\end{definition}

\begin{remark}
	O número de soluções de uma equação do segundo grau é totalmente determinado pelo sinal do seu discriminante, de forma tal que:
	\begin{itemize}
		\item Se $\Delta > 0$, existem duas soluções reais;
		\item Se $\Delta = 0 $, existe uma solução real ($x_1 = x_2 = \frac{-b}{2a})$;
		\item Se $\Delta < 0$, não existe solução real.
	\end{itemize}
\end{remark}

\begin{example}
	Sabendo que $x$ é um número real que satisfaz a equação:
	\[
		x = 1 + \frac 1 {1 + \frac 1 x},
	\]
	determine os valores possíveis de $x$.
\end{example}

\begin{solution}
	Manipulemos a equação:
	\begin{align*}
		x = 1 + \dfrac 1 {1 + \frac 1 x} & \sse x \prn{1 + \dfrac 1 x } = \prn{ 1 + \dfrac 1 x }+1 \\
										 & \sse x + 1 = \dfrac 1 x + 2 \\
										 & \sse x^2 + x = 1 + 2x \\
										 & \sse x^2 - x - 1 = 0 \\
										 & \sse x = \dfrac{1 \pm\raiz 5} 2.
	\end{align*}
	Logo, as soluções são $\dfrac {1 - \raiz 5} 2$ e $\dfrac { 1 + \raiz 5 } 2$.
\end{solution}

\begin{remark}
	O número $\phi = \frac {1 + \raiz 5} 2$ é conhecido como razão áurea, número de ouro, proporção divina, entre outras denominações.
	Veja o episódio A Proporção Divina \link{https://www.youtube.com/watch?v=mfL6-g5mQw4}{parte 01} e \link{https://www.youtube.com/watch?v=xtsTXAwWF20&}{parte 02} do programa português Isto É Matemática.
\end{remark}

\begin{onlineact}
	\khan{https://pt.khanacademy.org/math/algebra/quadratics/solving-quadratics-using-the-quadratic-formula/e/quadratic_equation}{Fórmula de Bhaskara}.
\end{onlineact}

\begin{theorem}
	Os números $\alpha$ e $\beta$ são as raízes da equação do segundo grau:
	\[
		ax^2 + bx + c = 0
	\]
	se, e somente se,
	\[
		\alpha + \beta = \dfrac {-b} a \; \e \; \alpha \beta = \dfrac c a.
	\]
\end{theorem}

\begin{proof}
	Sejam $\alpha$ e $\beta$ raízes da equação do 2\tdeg{} grau $ax^2 + bx + c = 0$. Do Exemplo~\ref{ex:sol-2-grau}, calculando $\alpha+\beta$, obtemos:
	\[
		\alpha+\beta = \dfrac{-b+\sqrt\Delta}{2a}+\dfrac{-b-\sqrt\Delta}{2a} = \dfrac{-2b}{2a}=\dfrac{-b}{a}.
	\]
	A demonstração de $\alpha\beta = \dfrac c a$ fica como exercício para o leitor.

    Reciprocamente, se $\alpha+\beta = \dfrac {-b} a$ e $\alpha\beta = \dfrac c a$, então, da fatoração de $ax^2 + bx + c$, segue que:
    \[
        ax^2 + bx + c = a\paren{x^2 + \frac {b} a x+ \frac c a} = a\colche{x^2  -\paren{\alpha+\beta} x+ \alpha\beta} = a\paren{x - \alpha}\paren{x - \beta} = 0.
    \]
    Logo, $\alpha$ e $\beta$ são raízes da equação do segundo grau.
\end{proof}