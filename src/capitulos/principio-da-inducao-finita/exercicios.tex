\section{Exercícios}

\begin{exercise}
Demonstre, por indução, que para qualquer $n \in \N^\ast$ é válida a igualdade:
$$1+2+ 3+ \dots + n = \frac{n \paren {n+1}} 2.$$
\end{exercise}

\begin{exercise}
Demonstre, por indução, que para qualquer $n \in \N^\ast$ é
válida a igualdade:
$$1^2 +2^2 + 3^2+ \dots + n^2 = \frac{n \paren {n+1} \paren{2n+1}} 6.$$
\end{exercise}

\begin{exercise}
Prove que $3^{n-1} < 2^{n^2}$ para todo $n \in \N^\ast$.
\end{exercise}

\begin{exercise}
Mostre, por indução, que
$$\paren{\frac {n+1}{n}}^n \leq n,$$
para todo $n \in \N^\ast$ tal que $n \geq 3$.

\dica Mostre que $\frac{k+2}{k+1} \leq \frac{k+1} k$ para todo $k \in
\N^\ast$. Depois, eleve tudo à potência $k+1$.
\end{exercise}

\begin{exercise}
Prove que
$$\frac 1 {\sqrt 1} +\frac 1 {\sqrt 2} +\frac 1 {\sqrt 3} + \dots + \frac 1 {\sqrt n} \geq \sqrt n,$$
para todo $ n \in \N^\ast$.
\end{exercise}

\begin{exercise}
Um subconjunto do plano é \emph{convexo} se o segmento ligando quaisquer dois de seus pontos está totalmente nele contido.
Os exemplos mais simples de conjuntos convexos são o próprio plano e qualquer semi-plano.
Mostre que, para qualquer $n \in \N^\ast$, a interseção $n$ de conjuntos convexos é ainda um conjunto convexo.
\end{exercise}