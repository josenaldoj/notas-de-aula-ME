\section{Conjuntos e Lógica}



Em Matemática, a Teoria de Conjuntos está intimamente relacionada à Lógica.
Como evidência disso, existem diversas equivalências entre relações e operadores de conjuntos e conectivos lógicos.
Apresentar-se-ão quatro delas, mas antes vamos entender como usamos os símbolos $ \implies$ e $\iff$ no nosso curso.

Na Lógica Proposicional, a implicação $\prn \to $ é um conectivo lógico. 
Ela constrói, a partir de duas proposições $P$ e $Q$, uma terceira proposição $P \to Q$.
O valor-verdade de $P \to Q$ é completamente determinado quando se conhecem os valores-verdade de $P$ e $Q$.
A relação exata pode ser consultada na Tabela \ref{tbl:implicacao}, denominada tabela-verdade.

\begin{table}[h]
\caption{Tabela-verdade da implicação:}
\label{tbl:implicacao}
\centering
\begin{tabular}{cc|c}
$P$		& $Q$		& $P \to Q$	\\ \hline
V		& V			& V			\\
V		& F			& F			\\
F		& V			& V			\\
F		& F			& V			\\	
\end{tabular}
\end{table}

A implicação aparece textualmente de algumas maneiras; a mais comum delas sendo ``Se $P$, então $Q$''.
No caso de a afirmação anterior ser, por exemplo, um problema, interpreta-se o $P$ como a sua hipótese, e $Q$, como a sua tese.
Costuma-se escrever, também, nessa situação, $P \implies Q$ e reservar o $P \to Q$ para contextos puramente lógicos.

Para provar que uma implicação é verdadeira, o primeiro passo é considerar a hipótese $P$ como uma propriedade verdadeira.
Consequentemente, nossa atenção se restringe somente às duas primeiras linhas da tabela-verdade de $P \to Q$.
Assim, pela tabela, constatamos que, para alcançarmos nosso objetivo de provar que a afirmação ``Se $P$, então $Q$'' é verdadeira,
precisamos mostrar que a tese $Q$ também é verdadeira.

Quando já está provado que ``Se $P$, então $Q$'' é verdadeira, 
utilizamos esse resultado para concluir que $Q$ é verdadeira quando temos, também, que $P$ é verdadeira.
Isso pode, mais uma vez, ser verificado pela tabela-verdade de $P \to Q$:
na única linha em que $P \to Q$ e $P$ são verdadeiras, $Q$ também é.

Outro conectivo lógico é o $\liff$, chamado de bicondicional ou bi-implicação.
Da mesma forma que a implicação -- e os demais conectivos lógicos --, o valor-verdade da expressão $P \liff Q$, dadas duas proposições $P$ e $Q$, é totalmente determinado pelos valores-verdade dessas proposições.
A relação se encontra na Tabela \ref{tbl:bi-implicacao}.

\begin{table}[h]
\caption{Tabela-verdade da bi-implicação.}
\label{tbl:bi-implicacao}
\centering
\begin{tabular}{cc|c}
$P$		& $Q$		& $P \liff Q$	\\ \hline
V		& V			& V			\\
V		& F			& F			\\
F		& V			& F			\\
F		& F			& V			\\	
\end{tabular}
\end{table}
%
\noindent Pela tabela, pode-se notar que a expressão $P \liff Q$ é verdadeira quando $P$ e $Q$ têm o mesmo valor-verdade, e apenas nessa situação.

De forma similar ao que ocorre na implicação, reserva-se o símbolo $\liff$ para contextos lógicos, utilizando o $\iff$ apenas em cenários menos formais.
Textualmente, uma afirmação do tipo $P \iff Q$ costuma se manifestar como ``$P$ se, e somente se, $Q$''. 
Isso começa a evidenciar um importante fato sobre a bi-implicação: ela pode ser definida como a conjunção de duas implicações.
A frase ``$P$ se $Q$'' é equivalente a ``se $Q$, então $P$'', o que, conforme já visto, pode ser representado como $Q \implies P$.
Ademais, ``$P$ somente se $Q$'' diz que $P$ só pode ser verdade se $Q$ também for, ou seja, $P \implies Q$. 
As frases, quando juntas, expressam a noção de bi-condicionalidade.

Outra versão textual comum do $\iff$ é a sentença ``Para $P$, é necessário e suficiente que $Q$''.
Assim como no caso do ``se, e somente se'', essa expressão carrega duas informações consigo.
A frase ``Para $P$, é necessário que $Q$'' significa que a validade de $P$ só é possível na presença da validade de $Q$, isso é, $P \implies Q$.
Por outro lado, ``Para $P$, é suficiente que $Q$'' indica que basta $Q$ ser verdade para $P$ também ser, ou seja, $Q \implies P$.

Uma terceira forma de se apresentar a informação $P \iff Q$ é dizer que ``$P$ é equivalente a $Q$''.
Pode-se justificar essa construção pela tabela-verdade de $P \liff Q$. 
Conforme mencionado, $P \liff Q$ é verdade precisamente quando $P$ e $Q$ tem o mesmo valor-verdade.
Em caso positivo, não existe uma diferença prática entre $P$ e $Q$.
Assim, usa-se o termo ``equivalência'', que também indica que, quando se tem uma dessas proposições, se tem a outra, já que elas possuem o mesmo valor-verdade.

Em todas as formas textuais vistas, o $\iff$ pôde ser desmembrado em dois $\implies$.
Isso reflete diretamente na maneira como se prova afirmações do tipo $P \iff Q$.
O método consiste em demonstrar, de forma independente, que $P \implies Q$ e que $Q \implies P$.
Aqui, vale a discussão acerca de como provar afirmações com $\implies$.
A diferença é que sempre se fazem necessárias duas provas desse tipo.

Explicados os usos do $\implies$  e do $\iff$ neste texto, podemos voltar a atenção à equivalência entre operadores e relações de conjuntos e conectivos lógicos.
No restante desta seção, considere $P$ e $Q$ propriedades aplicáveis aos elementos de $\U$.
Considere, também, $A = \set {x \tq x \text{ satisfaz } P}$ e $B= \set {x \tq x \text{ satisfaz } Q}$.

\begin{equivalence}[Inclusão e implicação] 
$$ A \subset B \text{ é equivalente a } P \implies Q \text{.} $$
\end{equivalence}

\begin{equivalence}[Igualdade e bi-implicação] 
$$ A = B \text{ é equivalente a } P \iff Q \text{.} $$
\end{equivalence}

\begin{example}
Analise as implicações abaixo:
\begin{equation*}
\begin{aligned}
x^2+1=0 & \implies \left(x^2 +1 \right) \left( x^2-1 \right) = 0
\cdot \left( x^2-1 \right) \\
& \implies x^4 - 1 = 0 \\
& \implies x^4 = 1 \\
& \implies x \in \set {-1, 1}
\end{aligned}
\end{equation*}
%
Isso quer dizer que o conjunto solução de $x^2 +1 = 0$ é $\set{-1, 1}$?
\end{example}

\begin{solution}
O conjunto-solução de $x^2 + 1 = 0$ é $S = \set{x \in \R \tq x^2 + 1 = 0} = \emptyset$, o que implica que $S \ne \set{-1,1}$.
\end{solution}

\begin{equivalence}[Complementar e negação] 
$A^C$ é equivalente a $\neg P$.
\end{equivalence}
Podemos combinar os itens \ref{prop:complementar:complementar-do-complementar} e \ref{prop:complementar:contrapositiva} da Proposição \ref{prop:complementar} e obter que:
%
$$ P \implies Q \sse \neg Q \implies \neg P \text{.} $$
%
Chamamos $Q \implies P$ de \textdef{recíproca} de $P \implies Q$, e $P \land \neg Q$ de \textdef{negação}\newline de $P \implies Q$. É dado um exemplo no Exercício \ref{exe:escrever-reciprocas}.

\begin{example}
Observe as afirmações:

\begin{enumerate}
	\item Todo número primo maior do que 2 é ímpar;
	\item Todo número par maior do que 2 é composto.
\end{enumerate}
%
Essas afirmações dizem exatamente a mesma coisa, ou seja, exprimem a mesma ideia; só que com diferentes termos. Podemos reescrevê-las na forma de implicações vendo claramente que uma é contrapositiva da outra, e todas estão sob a hipótese de que $n \in \N$, com $n > 2$:
%
\begin{nofleqn}{align*}
n \text{ primo 	} & \implies n \text{ ímpar}\\
\neg (\text{ }n \text{ ímpar } )& \implies \neg (\text{ }n \text{ primo } )\\
n \text{ par } & \implies n \text{ composto}
\end{nofleqn}
\end{example}

\begin{equivalence}[União e disjunção]
$A \cup B$ é equivalente a $P \lor Q$ ($P \text{ ou } Q$).
\end{equivalence}

\begin{equivalence}[Interseção e conjunção]
$A \cap B$ é equivalente a $P \land Q$ ($P \text{ e } Q$).
\end{equivalence}

\begin{remark}
O conectivo lógico \textit{ou} tem significado diferente do usado normalmente no português. Na linguagem coloquial, usamos $P$ \textit{ou} $Q$ sem permitir que sejam as duas coisas ao mesmo tempo. Analise a seguinte história:\newline

\textit{Um obstetra que também era matemático acabara de realizar um parto quando o pai perguntou: ``É menino ou menina, doutor?''. E ele respondeu: ``sim''}.
\end{remark}

As equivalências entre as relações e os operadores da Teoria dos Conjuntos e conectivos da Lógica são resumidas na Tabela \ref{tbl:equiv-conj-logc}.

\begin{table}[h]
\begin{center}
\caption{Equivalências entre as relações e operadores de conjuntos e conectivos lógicos.}
\label{tbl:equiv-conj-logc}
\begin{tabular}{|c|c|}
	\hline
	Operação/relação em Conjuntos & Fórmula de Lógica \\\hline
	$A=B$ & $P \iff Q$ \\ \hline
	$A \subset B$ & $P \implies Q$ \\ \hline
	$A^C$ & $\neg P$ \\ \hline
	$A \cup B$ & $P \lor Q$ \\ \hline
	$A \cap B$ & $P \land Q$ \\
	\hline
\end{tabular}
\end{center}
\end{table}

\begin{problem}
A polícia prende quatro homens, um dos quais cometeu um furto. Eles fazem as seguintes declarações:
%
\begin{itemize}[leftmargin=*]
  \item Arnaldo: Bernaldo fez o furto.
  \item Bernaldo: Cernaldo fez o furto.
  \item Dernaldo: eu não fiz o furto.
  \item Cernaldo: Bernaldo mente ao dizer que eu fiz o furto.
\end{itemize}
%
Se sabemos que só uma destas declarações é a verdadeira, quem é culpado pelo furto?
\end{problem}
